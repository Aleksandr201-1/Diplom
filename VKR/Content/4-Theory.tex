\section{Теория}

Текст теоретического обзора.

\subsection{Актуальность темы}

Обыкновенные дифференциальные уравнения (ОДУ) и системы дифференциальных уравнений (СДУ) широко используются для математического моделирования процессов и явлений в различных областях науки и техники. Переходные процессы в радиотехнике, динамика биологических популяций, модели экономического развития, движение космических объектов и так далее исследуются с помощью ОДУ и СДУ. Помимо этого, с помощью ОДУ и СДУ моделируются химические процессы (расчёт термодинамического равновесия, моделирование химических превращений и так далее).

\subsubsection{Движение космических объектов}

\subsubsection{Баллистика}

\subsubsection{Механика твердого тела (маятники с одной осью, многоосные)}

\subsubsection{Механика жидкости и газа (течение газа, горение газа)}

\subsubsection{Физика высоких скоростей}

%\paragraph{paragraph}

\subsection{Программные комплексы для решения ОДУ}

В настоящее время для решения систем ОДУ создано большое количество ПО и библиотек. Все они обладают как своими плюсами так и недостатками. Здесь будет перечислены наиболее распространённые программы и библиотеки. 


\subsubsection{MatLab}

MatLab ~--- пакет прикладных программ для решения задач технических вычислений. Пакет используют более миллиона инженерных и научных работников, он работает на большинстве современных операционных систем, включая Linux, macOS, Solaris (начиная с версии R2010b поддержка Solaris прекращена) и Windows.

Для решения ОДУ MatLab предлагает следующие методы:
\begin{itemize}
    \item ode23() — метод решения не жёстких дифференциальных уравнений; метод низкого порядка,
    \item ode45() — метод решения не жёстких дифференциальных уравнений; метод среднего порядка,
    \item ode113() — метод решения не жёстких дифференциальных уравнений; метод переменного порядка,
    \item ode15s() — метод решения жёстких дифференциальных уравнени1 и DAE; метод переменного порядка,
    \item ode23s() — метод решения жёстких дифференциальных уравнений; метод низкого порядка.
\end{itemize}

\subsubsection{Intel ODE Solvers Library}

Библиотека для решения ОДУ. Содержит следующие методы:
\begin{itemize}
    \item rkm9st() — специализированная программа для решения нежестких и средне-жестких систем ODE с использованием явного метода, который основан на методе Мерсона 4-го порядка и многоступенчатом методе 1-го порядка, включающем до 9 этапов с контролем устойчивости.
    \item mk52lfn() — специализированная процедура для решения жестких систем ODE с использованием неявного метода, основанного на L-стабильном (5,2)-методе с числовой матрицей Якоби, которая вычисляется с помощью процедуры.
    \item mk52lfa() — специализированная программа для решения жестких систем ODE с использованием неявного метода, основанного на L-stable (5,2)-методе с численным или аналитическим вычислением матрицы Якоби. Пользователь должен предоставить процедуру для этого вычисления.
    \item rkm9mkn() — специализированная процедура для решения систем ODE с переменной или априори неизвестной жесткостью; автоматически выбирает явную или неявную схему на каждом шаге и при необходимости вычисляет числовую матрицу Якоби.
    \item rkm9mka() — специализированная подпрограмма для решения систем ODE с переменной или априори неизвестной жесткостью; автоматически выбирает явную или неявную схему на каждом шаге. Пользователь должен предоставить процедуру для численного или аналитического вычисления матрицы Якоби.
\end{itemize}

Библиотека написана на C со всеми вытекающими отсюда зависимостями. Доступны 32- и 64-разрядные версии библиотеки (libiode\_ia32.lib и libiode\_intel64.lib).

\subsubsection{GNU Scientific Library (GSL)}

GNU Scientific Library (или GSL) ~--- это библиотека, написанная на языке программирования C для численных вычислений в прикладной математике и науке. GSL является частью проекта GNU и распространяется на условиях лицензии GPL.

GSL используется, в частности, в таком программном обеспечении, как PSPP и Perl Data Language.

Явные методы:
\begin{itemize}
    \item rk2() — явный встроенный метод Рунге-Кутты (2, 3).
    \item rk4() — явный 4-й порядок (классический) Рунге-Кутты. Оценка погрешности осуществляется методом удвоения шага.
    \item rkf45() — явный встроенный метод Рунге-Кутты-Фельберга (4, 5).
    \item rkck() — явный встроенный метод Рунге-Кутты Кэш-Карпа (4, 5).
    \item rk8pd() — явный встроенный метод Рунге-Кутты Принца-Дорманда (8, 9).
\end{itemize}

Неявные методы:
\begin{itemize}
    \item rk1imp() — неявный гауссовский метод Рунге-Кутты первого порядка. Также известен как неявный метод Эйлера или обратный метод Эйлера. Оценка погрешности осуществляется методом удвоения шага. Для этого алгоритма требуется якобиан.
    \item rk2imp() — неявный гауссовский метод Рунге-Кутты второго порядка. Также известно как неявное правило средней точки. Оценка погрешности осуществляется методом удвоения шага. Для этого шагового двигателя требуется якобиан.
    \item rk4imp() — неявный гауссов Рунге-Кутта 4-го порядка. Оценка погрешности осуществляется методом удвоения шага. Для этого алгоритма требуется якобиан.
    \item bsimp() — неявный метод Булирша-Стоера Бадера и Дейфлхарда. Этот метод, как правило, подходит для сложных задач. Для этого шагового двигателя требуется якобиан.
    \item msadams() — линейный многоступенчатый метод Адамса с переменным коэффициентом в форме Nordsieck. Этот шаговый процессор использует явные методы Адамса-Башфорта (предсказатель) и неявные методы Адамса-Моултона (корректор) в режиме функциональной итерации $P(EC) ^m$. Порядок методов динамически варьируется от 1 до 12.
    \item msbdf() — метод линейной многоступенчатой формулы обратного дифференцирования с переменным коэффициентом (BDF) в форме Nordsieck. Этот шаговый преобразователь использует явную формулу BDF в качестве предиктора и неявную формулу BDF в качестве корректора. Для решения системы нелинейных уравнений используется модифицированный итерационный метод Ньютона. Порядок методов динамически варьируется от 1 до 5. Этот метод, как правило, подходит для сложных задач. Для этого шагового двигателя требуется якобиан.
\end{itemize}

\subsubsection{DotNumerics}

Решатели для нежёстких систем:
\begin{itemize}
    \item AdamsMoulton() — решает начальную задачу для нелинейных обыкновенных дифференциальных уравнений с использованием метода Адамса-Моултона.
    \item ExplicitRK45() — решает задачу с начальными значениями для нелинейных обыкновенных дифференциальных уравнений, используя явный метод Рунге-Кутты порядка (4)5.
\end{itemize}

Решатели для жестких систем:
\begin{itemize}
    \item ImplicitRK5() — решает начальную задачу для жестких обыкновенных дифференциальных уравнений с использованием неявного метода Рунге-Кутты порядка 5.
    \item GearsBDF() — решает начальную задачу для жестких обыкновенных дифференциальных уравнений с использованием метода BDF Gear.
\end{itemize}

\subsubsection{SMath Studio}

SMath Studio ~--- программа для вычисления математических выражений и построения графиков функций. Работа с интерфейсом программы напоминает работу с обычным листом бумаги, так как все математические выражения в ней записываются не в строчку текстом, а в графическом, удобном для человека, виде (по аналогии с системой MathCad). Примечательно, что данная программа является отечественной.

Для решения ОДУ в SMath Studio реализованы следующие методы:

\begin{itemize}
    \item метод Эйлера (2-го порядка)
    \item метод Рунге-Кутты (классический 4-го порядка)
\end{itemize}

