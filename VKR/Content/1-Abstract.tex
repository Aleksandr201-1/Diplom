\vkrAbstract
%\maketitle\newpage\chapter*{РЕФЕРАТ}

Отчёт 1 с., 1 кн., 1 рис., 1 табл., 1 источн., 1 прил.

ЧИСЛЕННЫЕ МЕТОДЫ, МЕТОД РУНГЕ-КУТТУ, ЖЁСТКИЕ СИСТЕМЫ, СДУ, ОДУ, ХИМИЧЕСКАЯ КИНЕТИКА

Объектом разработки является 

Цель работы ~--- разработка программы, позволяющей решать системы дифференциальных уравнений (СДУ) и .

В процессе работы были использованы явные и неявные методы Рунге-Кутты, метод QR и LU разложения матрицы, методы итераций, методы Зейделя и Ньютона для решения СЛАУ.

В результате работы была создана программа, позволяющая  решать СДУ при помощи большой коллекции методов и в частности уравнения химической кинетики.

Обыкновенные дифференциальные уравнения (ОДУ) и системы дифференциальных уравнений (СДУ) широко используются для математического моделирования процессов и явлений в различных областях науки и техники. Переходные процессы в радиотехнике, динамика биологических популяций, модели экономического развития, движение космических объектов и так далее исследуются с помощью ОДУ и СДУ.

Данное ПО можно использовать для исследований химической кинетики. Помимо этого её можно использовать в учебных целях для решения простых ОДУ.

Преимуществами моей работы по сравнению с аналогами являются большое число методов на выбор, возможность добавления своего метода решения, быстрота вычисления результата, 

В дальнейшем программу можно улучшить путём разработки модулей для работы с динамикой биологических популяций, моделями экономического развития, движением космических объектов и так далее, потому что все вышеперечисленные задачи решаются с помощью СДУ.