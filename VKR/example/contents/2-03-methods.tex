\section{КЛАССИЧЕСКИЕ МЕТОДЫ ДЛЯ РЕШЕНИЯ ОДУ}

\subsection{Постановка задачи}

Программа, над которой ведётся работа, предназначена для решения дифференциальных уравнений или систем дифференциальных 
уравнений с начальными значениями, то есть задачи Коши ~--- классической математической постановки. 
Сама по себе задача достаточно сложная и изучалась уже более ста лет.

Конкретная прикладная задача сводится к решению дифференциального уравнения произвольного порядка \ref{eq-koshi}:
\begin{equation}
    \begin{cases}
        y^{(n)} = f(x, y, y', y'', ..., y^{(n - 1)})\\
        y(x_0) = y_0\\
        y'(x_0) = y_1\\
        y''(x_0) = y_2\\
        ...\\
        y^{(n - 1)}(x_0) = y_{n - 1}
    \end{cases}
    \label{eq-koshi}
\end{equation}

Данное уравнение произвольного порядка $n$ может быть преобразовано в систему из $n$ дифференциальных уравнений первого порядка путём
замены переменных. Пример \ref{eq-koshi-system} демонстрирует преобразование задачи Коши второго порядка в систему из 2-х уравнений
первого порядка, путём замены $y'$ на $z$:
\begin{equation}
    \begin{cases}
        z' = f(x, y, y', y'')\\
        y' = z\\
        y(x_0) = y_0\\
        z(x_0) = y_1
    \end{cases}
    \label{eq-koshi-system}
\end{equation}

Из курса дифференциальных уравнений известно, что задача с начальными условиями при непрерывных правых частях, удовлетворяющих условию
Липшица по всем переменным, имеет единственное решение.

Методы решения дифференциальных уравнений можно классифицировать на точные, приближенные и численные. Точные методы, которые изучаются
в курсе дифференциальных уравнений и могут быть применены к очень ограниченному кругу уравнений, позволяют выразить решение
дифференциальных уравнений либо через элементарные функции, либо с помощью квадратур от элементарных функций. К приближенным методам
относятся приемы, в которых решение дифференциального уравнения получается, как предел некоторой последовательности, элементы которой
построены с помощью элементарных функций. Численные методы представляют собой алгоритмы вычисления приближенных значений искомой
функции в узлах. %ссылку бы сюда

Такие задачи могут отличаться между собой сложностью решения. Так одни задачи можно решать при помощи группы явных методов и получать
достаточно точное решение. Другие же задачи, в которых присутствует резкий скачок градиента функции, решать приходится с использованием
неявных схем. Такие задачи называются жёсткими.

\subsection{Жёсткость}

Будем считать линейную систему обыкновенных дифференциальных уравнений $u' = Au$ ($A$ ~--- постоянная матрица $n \times n$)
жёсткой, если выполняются следующие требования:

\begin{enumerate}
    \item все собственные числа $\lambda_i$ матрицы $A$ имеют отрицательную действительную часть, т. е. $Re\lambda_i < 0$, 
    $i = 1, 2, ..., n$;
    \item число
        \begin{equation}
            S = \dfrac{\max\limits_{1 \leq k \leq n}|Re\lambda_k|}{\min\limits_{1 \leq k \leq n}|Re\lambda_k|}
            \label{eq:tough}
        \end{equation}
        велико
\end{enumerate}

Число \ref{eq:tough} называется жёсткостью задачи.

Для жёстких задач число жёсткости должно быть намного больше единицы, однако чёткой границы между жёсткой и нежёсткой задачей нет. Так
модельные уравнения с числом жёсткости более $100$ уже дают небольшие скачки погрешности решения. В задачах химической кинетики число
жёсткости может быть более $10^6$.

\subsection{Явные схемы для решения ОДУ}

Для решения жёстких и нежёстких задач можно использовать различные семейства методов. В данной работе будет рассмотрено семейство
одношаговых методов Рунге-Кутты.

В семейство методов Рунге-Кутты входит огромное число схем, как явных, так и неявных. Все эти методы представлены в виде таблиц
Бутчера. Общий вид явных схем представлен на рисунке \ref{fig:Runge}.

Явные методы обладают нижней диагональной формой таблицы Бутчера и позволяют решать задачи обычными маршевыми методами. В связи с тем,
что явные методы являются условно устойчивыми, работа с ними сильно зависит от размера шага итегрирования и для достижения заданной
точности требуют достаточно мелкий шаг интегрирования и повторного вычисления для повышения порядка точности процедурой Рунге-Ромберга.

%Общий вид явных схем:

% \begin{equation}
%     \begin{cases}
%         y_{k + 1} = y_k + \Delta y_k\\
%         \Delta y_k = \mathlarger{\sum}\limits_{i = 1}^{p}b_iK_i^k\\
%         K_i^k = hf\underset{i = 2, 3, ..., p}{(x_k + c_i h, y_k + h \mathlarger{\sum}\limits_{j = 1}^{i - 1}a_{ij}K_j^k)}
%     \end{cases}
%     \label{eq:explicit}
% \end{equation}

\begin{figure}
    \includegraphics[width=15cm]{2-03-runge}
    \caption{Общий вид явных схем}
    \label{fig:Runge}
\end{figure}

Примеры таблиц Бутчера \ref{tab:RungeKutta4}, \ref{tab:RungeKutta6} для явных методов:

\begin{table}    
    \caption{Таблица Бутчера для метода явного метода Рунге-Кутты 4-го порядка}
    \begin{tabular}{|c|c|c|c|c|}
    \hline
    $0$ & $0$ & $0$ & $0$ & $0$\\
    \hline
    $\frac{1}{2}$ & $\frac{1}{2}$ & $0$ & $0$ & $0$\\
    \hline
    $\frac{1}{2}$ & $0$ & $\frac{1}{2}$ & $0$ & $0$\\
    \hline
    $1$ & $0$ & $0$ & $1$ & $0$\\
    \hline
    $0$ & \cellcolor{lightgray} $\frac{1}{6}$ & \cellcolor{lightgray} $\frac{1}{3}$ & \cellcolor{lightgray} $\frac{1}{3}$ & \cellcolor{lightgray} $\frac{1}{6}$\\
    \hline
    \end{tabular}
    \label{tab:RungeKutta4}
\end{table}

\begin{table}    
    \caption{Таблица Бутчера для метода явного метода Рунге-Кутты 6-го порядка}
    \begin{tabular}{|c|c|c|c|c|c|c|}
    \hline
    $0$ & $0$ & $0$ & $0$ & $0$ & $0$ & $0$\\
    \hline
    $\frac{1}{4}$ & $\frac{1}{4}$ & $0$ & $0$ & $0$ & $0$ & $0$\\
    \hline
    $\frac{1}{2}$ & $\frac{1}{2}$ & $0$ & $0$ & $0$ & $0$ & $0$\\
    \hline
    $\frac{1}{2}$ & $\frac{1}{7}$ & $\frac{2}{7}$ & $\frac{1}{14}$ & $0$ & $0$ & $0$\\
    \hline
    $\frac{3}{4}$ & $\frac{3}{8}$ & $0$ & $-\frac{1}{2}$ & $\frac{7}{8}$ & $0$ & $0$\\
    \hline
    $1$ & $-\frac{4}{7}$ & $\frac{12}{7}$ & $-\frac{2}{7}$ & $-1$ & $7$ & $0$\\
    \hline
    $0$ & \cellcolor{lightgray} $\frac{7}{90}$ & \cellcolor{lightgray} $\frac{16}{45}$ & \cellcolor{lightgray} $-\frac{1}{3}$ & \cellcolor{lightgray} $\frac{7}{15}$ & \cellcolor{lightgray} $\frac{16}{45}$ & \cellcolor{lightgray} $\frac{7}{90}$\\
    \hline
    \end{tabular}
    \label{tab:RungeKutta6}
\end{table}

\subsection{Явные вложенные схемы для решения ОДУ}

В связи с перечисленными недостатками обычных явных схем, есть смысл применить явные вложенные схемы, которые базируются так же на
нижней треугольной матрице Бутчера, обладают маршевым
методом решения и позволяют на базе одних и тех же поправочных коэффициентов моделировать решение с разным порядком точности и тем
самым либо увеличивать шаг интегрирования, либо уменьшать.

%(примеры таблиц Дормана-Принца и Фалберга 2)
%(формулы в общем виде)

\begin{figure}
    \includegraphics[width=15cm]{2-03-falberg}
    \caption{Общий вид явных вложенных схем}
    \label{fig:Falberg}
\end{figure}

Примеры таблиц Бутчера для вложенных методов:

\begin{table}
    \caption{Таблица Бутчера для метода Фалберга 2-го порядка}
    \begin{tabular}{|c|c|c|c|}
    \hline
    $0$ & $0$ & $0$ & $0$\\
    \hline
    $1$ & $1$ & $0$ & $0$\\
    \hline
    $\frac{1}{2}$ & $\frac{1}{4}$ & $\frac{1}{4}$ & $0$\\
    \hline
    $0$ & \cellcolor{lightgray} $\frac{1}{2}$ & \cellcolor{lightgray} $\frac{1}{2}$ & \cellcolor{lightgray} $0$\\
    \hline
    $0$ & \cellcolor{lightgray} $\frac{1}{6}$ & \cellcolor{lightgray} $\frac{1}{6}$ & \cellcolor{lightgray} $\frac{4}{6}$\\
    \hline
    \end{tabular}
    \label{tab:Falberg2}
\end{table}

\begin{table}    
    \caption{Таблица Бутчера для метода Дормана-Принца 4-го порядка}
    \begin{tabular}{|c|c|c|c|c|c|c|c|}
    \hline
    $0$ & $0$ & $0$ & $0$ & $0$ & $0$ & $0$ & $0$\\
    \hline
    $\frac{1}{5}$ & $\frac{1}{5}$ & $0$ & $0$ & $0$ & $0$ & $0$ & $0$\\
    \hline
    $\frac{3}{10}$ & $\frac{3}{40}$ & $\frac{9}{40}$ & $0$ & $0$ & $0$ & $0$ & $0$\\
    \hline
    $\frac{4}{5}$ & $\frac{44}{45}$ & $-\frac{56}{15}$ & $\frac{32}{9}$ & $0$ & $0$ & $0$ & $0$\\
    \hline
    $\frac{8}{9}$ & $\frac{19372}{6561}$ & $-\frac{25360}{2187}$ & $\frac{64448}{6561}$ & $-\frac{212}{729}$ & $0$ & $0$ & $0$\\
    \hline
    $1$ & $\frac{9017}{3168}$ & $-\frac{355}{33}$ & $\frac{46732}{5247}$ & $\frac{49}{176}$ & $-\frac{5103}{18656}$ & $0$ & $0$\\
    \hline
    $1$ & $\frac{35}{384}$ & $0$ & $\frac{500}{1113}$ & $\frac{125}{192}$ & $-\frac{2187}{6784}$ & $\frac{11}{84}$ & $0$\\
    \hline
    $0$ & \cellcolor{lightgray} $\frac{35}{384}$ & \cellcolor{lightgray} $0$ & \cellcolor{lightgray} $\frac{500}{1113}$ & \cellcolor{lightgray} $\frac{125}{192}$ & \cellcolor{lightgray} $-\frac{2187}{6784}$ & \cellcolor{lightgray} $\frac{11}{84}$ & \cellcolor{lightgray} $0$\\
    \hline
    $0$ & \cellcolor{lightgray} $\frac{5179}{57600}$ & \cellcolor{lightgray} $0$ & \cellcolor{lightgray} $\frac{7571}{16695}$ & \cellcolor{lightgray} $\frac{393}{640}$ & \cellcolor{lightgray} $-\frac{92097}{339200}$ & \cellcolor{lightgray} $\frac{187}{2100}$ & \cellcolor{lightgray} $\frac{1}{40}$\\
    \hline
    \end{tabular}
    \label{tab:DormanPrince4}
\end{table}

\subsection{Неявные схемы для решения ОДУ}

К наиболее сложным методам можно отнести группу неявных схем. Плотно заполненная таблица Бутчера не позволяет использовать маршевые
методы и принуждает решать систему алгебраических уравнений на каждом шаге интегрирования, что вызывет определённые сложности у
разработчиков.

\begin{figure}
    \includegraphics[width=15cm]{2-03-gauss}
    \caption{Общий вид неявных схем}
    \label{fig:Gauss}
\end{figure}

Примеры таблиц Бутчера для неявных методов:

\begin{table}    
    \caption{Таблица Бутчера для метода неявного метода Гаусса 4-го порядка}
    \begin{tabular}{|c|c|c|}
    \hline
    $\frac{1}{2} - \frac{\sqrt{3}}{6}$ & $\frac{1}{4}$ & $\frac{1}{4} - \frac{\sqrt{3}}{6}$\\
    \hline
    $\frac{1}{2} + \frac{\sqrt{3}}{6}$ & $\frac{1}{4} + \frac{\sqrt{3}}{6}$ & $\frac{1}{4}$\\
    \hline
    $0$ & \cellcolor{lightgray} $\frac{1}{2}$ & \cellcolor{lightgray} $\frac{1}{2}$\\
    \hline
    \end{tabular}
    \label{tab:Gauss4}
\end{table}

\begin{table}    
    \caption{Таблица Бутчера для метода неявного метода Гаусса 6-го порядка}
    \begin{tabular}{|c|c|c|c|}
    \hline
    $\frac{1}{2} - \frac{\sqrt{15}}{10}$ & $\frac{5}{36}$ & $\frac{2}{9} - \frac{\sqrt{15}}{15}$ & $\frac{5}{36} - \frac{\sqrt{15}}{30}$\\
    \hline
    $\frac{1}{2}$ & $\frac{5}{36} + \frac{\sqrt{15}}{24}$ & $\frac{2}{9}$ & $\frac{5}{36} - \frac{\sqrt{15}}{24}$\\
    \hline
    $\frac{1}{2} + \frac{\sqrt{15}}{10}$ & $\frac{5}{36} + \frac{\sqrt{15}}{30}$ & $\frac{2}{9} + \frac{\sqrt{15}}{15}$ & $\frac{5}{36}$\\
    \hline
    $0$ & \cellcolor{lightgray} $\frac{5}{18}$ & \cellcolor{lightgray} $\frac{4}{9}$ & \cellcolor{lightgray} $\frac{5}{18}$\\
    \hline
    \end{tabular}
    \label{tab:Gauss6}
\end{table}

Помимо перечисленных групп методов, существуют так же диагональные неявные методы, неявные вложенные, неявные методы без одной строки
или столбца, но все они являются подгруппами неявных методов и используют тот же вид общей схемы.

Преимущество явных методов заключается в производительности, так как им не нужно решать на каждом шаге системы алгебраических
уравнений. Отсюда следует, что использование явных методов даёт больший выигрыш по времени, чем использование более производительного
оборудования и распараллеливания. Главным преимуществом неявных методов является наличие устойчивости (А-устойчивости, L-устойчивости и
так далее), в результате чего полученное ими решение является гарантированно устойчивым, в отличие от явных.

\subsection{Устойчивость}

Текст

Однако некоторые СДУ, описывающие химические процессы можно решить и явными методами с требуемой точностью, потому что свойства А и
L-устойчивости являются
лишь достаточным, но не необходимым условием эффективности решения, поэтому нельзя использовать только те или иные методы. Выбрать
оптимальный метод можно путём целевого тестирования.


%Цитирование источника 4 \cite{Wikipedia4} \cite{cite_1_2} \cite{cite_1_15} \cite{cite_1_16}.
