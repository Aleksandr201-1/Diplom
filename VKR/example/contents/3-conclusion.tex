\conclusion

Выпускная квалификационная работа бакалавра представляет собой программу для работы с СДУ при помощи множества методов семейства Рунге-Кутты. В работе
реализовано 62 схемы со 2 по 6 порядок точности. Из них 18 явных со 2 по 6 порядок точности, 9 вложенных, включая схему Дормана-Принса 4(5) порядка,
22 неявных, в том числе
схемы Радо, Гаусса и Лобатто для полных и неполных матриц. Помимо этого, протестирован один L-стабильный диагональный метод. Для
неявных схем используются схемы решения САУ первого порядка (простой итерации, Зейделя) и второго порядка (метод Ньютона), причём
для обращения
матрицы применялся метод LU-разложения. Для дифференцирования функции при построении матрицы Якоби для метода Ньютоны использовались
формулы с 4 порядком точности. При необходимости можно использовать формулы с меньшим порядком.

Для того чтобы определить, какие методы можно использовать, был реализован алгоритм вычисления числа жёсткости СДУ, использующий
QR-разложение матрицы системы для поиска всех собственных чисел. При высокой жёсткости для расчёта применяютя только жёсткие
схемы.

Для удобного отображения результатов вычислений был создан генератор pdf-отчётов, в которых представлена информация о решаемой задаче,
метод её решения и таблица Бутчера для этого метода, график, отображающий решение численными методами и аналитическое (при наличии) и
время, затраченное на работу программы. Если задача решалась при помощи жёстких схем, то дополнительно выводится информация о
количестве итераций. Так же, при необходимости, может быть построен приближающий полином. Кроме этого, реализована возможность простого
добавления новых методов решения на случай, если пользователю нужно решение каким-либо специфическим методом, которого нет в программе.
При помощи пользовательского интерфейса можно использовать разработанную программу как для решения ОДУ, так и в целях обучения.

Программа тестировалась как на задачах химической кинетики, так и на модельных уравнениях и дала удовлетворительные результаты.
В дальнейшем на базе данной программы планируется реализация решений краевых задач.

%недостатки (парсер)