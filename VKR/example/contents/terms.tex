\newglossaryentry{id1}{ % Нужны разные id, можно ставить просто последовательно
    name={Жёсткая система},
    description={ОДУ, численное решение которого явными методами является неудовлетворительным из-за резкого увеличения
    числа вычислений или из-за резкого возрастания погрешности при недостаточно малом шаге.} 
}
\newglossaryentry{id2}{
    name={Методы Рунге-Кутты},
    description={большой класс численных методов решения задачи Коши для обыкновенных дифференциальных уравнений и их систем.} 
}
\newglossaryentry{id3}{
    name={Химическая кинетика},
    description={раздел физической химии, изучающий закономерности протекания химических реакций во времени, зависимости этих
    закономерностей от внешних условий, а также механизмы химических превращений.} 
}
\newglossaryentry{id4}{
    name={Условие химического равновесия},
    description={равенство полных химических потенциалов исходных веществ и продуктов.} 
}
%термины численных методов

%термины прикладной задачи

%Добавить химию
%Матрица Бутчера
%Условие химического равновесия
%Методы рунге-кутты
%явные методы
%неявные
%вложенные явные
%вложенные неявные
%Жёсткость
