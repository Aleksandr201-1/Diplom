\section{РЕШЕНИЕ ЗАДАЧ ХИМИЧЕСКОЙ КИНЕТИКИ}

\subsection{Модели химической кинетики}

Для описания химической реакции необходимо знать закономерности её протекания во времени, а именно, скорость и механизм. Скорость и
механизм химических превращений изучает раздел химии ~--- химическая кинетика \cite{Article5, book7, book8}.

Будем рассматривать многокомпонентную систему переменного состава из $N$ веществ, в которой протекает $N_R$ реакций вида:

\begin{equation}
    \begin{gathered}
    \sum\limits_{i = 1}^{N}\overrightarrow{\nu_i^{(r)}}M_i \xleftrightarrow[\overrightarrow{W^{(r)}}]{\overleftarrow{W^{(r)}}} \sum\limits_{i = 1}^{N}\overleftarrow{\nu_i^{(r)}}M_i,\\
    \overleftrightarrow{q^{(r)}} = \sum\limits_{i = 1}^{N}\overleftrightarrow{\nu_i^{(r)}},\\
    r = 1, ..., N_R,\\
    i = 1, ..., N
    \end{gathered}
    \label{eq:ChemCommon}
\end{equation}

Здесь $r$ ~--- порядковый номер реакции, $i$ ~--- порядковый номер вещества, $\overleftrightarrow{\nu_i}$ ~--- стехиометрические
коэффициенты
(коэффициенты, стоящие перед молекулами веществ в химических уравнениях), $\overleftrightarrow{q^{(r)}}$ ~--- молекулярность
соответствующих элементарных реакций (число частиц, которые участвуют в элементарном акте химического взаимодействия).

В записи каждой реакции фигурирует $W^{(r)}$ ~--- скорость химической реакции. Она прямо пропорциональна произведению объёмных концентраций
участвующих в ней компонентов и так называемой константы скорости реакции $\overleftrightarrow{K^{r}}(T)$, зависящей от температуры
(в общем случае и от давления).

Общий вид формул скоростей химических реакций:

\begin{equation}
    \overleftrightarrow{W^{r}} = \overleftrightarrow{K^{r}}(T)\prod\limits_i(\rho\gamma_i)^{\overleftrightarrow{\nu^{r}}}
    \label{eq:Warrow}
\end{equation}

Константы скорости реакции $\overleftrightarrow{K^{r}}(T)$ расчитываются для прямого и оборатного хода реакции по следующим формулам:

\begin{equation}
    \overrightarrow{K^{r}}(T) = AT^n\exp{-\dfrac{E}{T}}
    \label{eq:Kright}
\end{equation}

\begin{equation}
    \overleftarrow{K^{r}}(T) = \overrightarrow{K^{r}}(T)\exp{(\sum\limits_{i=1}^N(\overrightarrow{\nu_i^{(r)}} - \overleftarrow{\nu_i^{(r)}})(\dfrac{G_i^0(T)}{RT} + \ln{\dfrac{RT}{p_0}}))}
    \label{eq:Kleft}
\end{equation}

Здесь $A, n, E$ ~--- Аррениусовские константы, $r = 1, ..., N_R$ ~--- порядковый номер реакции, $G_i^0(T)$ ~--- стандартный молярный потенциал
Гиббса. Для вычисления $G_i^0(T)$ используются полиномиальные аппроксимационные формулы:

\begin{equation}
    G_i^0(T) = \Delta_fH^0(T_0) - [H^0(T_0) - H^0(0)] - T\text{Ф}(T_0),
    \label{eq:Gi}
\end{equation}

где $H^0(0)$ ~--- стандартная энтальпия при абсолютном нуле, $H^0(T_0)$ ~--- стандартная энтальпия при $T_0$. Для задания $\text{Ф}^0(T)$
применяют полиномы.

\begin{equation}
    \text{Ф}^0(T) = \phi_0 + \phi_{ln}\ln x + \phi_{-2}x^{-2} + \phi_{-1}x^{-1} + \phi_{1}x + \phi_{2}x^{2} + \phi_{3}x^{3}
    \label{eq:F_i}
\end{equation}

Здесь $\phi_0, \phi_{ln}, \phi_{-2}, ..., \phi_3$ ~--- числовые коэффициенты, индивидуальные для каждого вещества.

Используя скорости $W^{(r)}$, можно составить уравнения изменения мольно-массовых концентраций по времени,
имеющих следующий вид:

\begin{equation}
    \begin{cases}
        %\rho \dfrac{d\gamma_i}{dt} = \overrightarrow{W_i}(\rho, T, \gamma_1, ..., \gamma_N) - \overleftarrow{W_i}(\rho, T, \gamma_1, ..., \gamma_N)\\
        \rho \dfrac{d\gamma_i}{dt} = W_i(\rho, T, \gamma_1, ..., \gamma_N)\\
        \gamma_i(t_0) = \gamma_i^0,\\
        i = 1, ..., N
    \end{cases}
    \label{eq:GammaI}
\end{equation}

где $W_i$ ~--- скорость образования i-го вещества. Вычисляется $W_i$ по формуле (\ref{eq:Wi}):

\begin{equation}
    W_i = \sum\limits_{r = 1}^{N_R}(\overrightarrow{\nu_i^{(r)}} - \overleftarrow{\nu_i^{(r)}})(\overrightarrow{W^{(r)}} - \overleftarrow{W^{(r)}})
    \label{eq:Wi}
\end{equation}

В формулах (\ref{eq:GammaI}) помимо $\gamma_i$ фигурируют плотность $\rho$ и температура $T$. Их значения могут быть как константами по
времи вычисления, так и переменными. В работе реализовано моделирование случая, когда плотность меняется по закону 
(\ref{eq:Rho}), а температура константна, и случая, когда температура меняется, а плотность считается константой.

\begin{equation}
    \rho = \dfrac{P}{RT\sum\limits_{i = 1}^{N}\gamma_i}
    \label{eq:Rho}
\end{equation}

Второй случай, связанный с изменением температуры немного сложнее в реализации, так как температуру приходится находить итерационными
методами из уравнения (\ref{eq:TempFind}).

%GFunc[j](T) - T * derivative(GFunc[j], T, 0.001, DiffConfig::POINTS4_ORDER1_WAY1) - R * T
\begin{equation}
    U = \sum\limits_{i = 1}^{N}(G_i^0(T) - T\dfrac{\partial G_i^0(T)}{\partial T} - RT)\gamma_i,
    \label{eq:TempFind}
\end{equation}

где $U$ ~--- полная внутренняя энергия системы, которая считается перед решением и сохраняется постоянной до конца расчётов. Если
представить данное уравнение в виде (\ref{eq:TempFind2}), то можно использовать, к примеру, метод Ньютона для поиска корня уравнения.

\begin{equation}
    F(T) = U - \sum\limits_{i = 1}^{N}(G_i^0(T) - T\dfrac{\partial G_i^0(T)}{\partial T} - RT)\gamma_i
    \label{eq:TempFind2}
\end{equation}

Так как $U$, является константой, то решение $F(T) = 0$ будет соответствовать значению температуры на следующем шаге интегрирования. 

\subsection{Примеры расчёта химического состава смеси}

В задачах химической кинетики на вход программе поступают не дифференциальные уравнения, а список химических уравнений, начальные
температура, давление, описание примесей, при их наличии, и начальные концентрации веществ.

\textit{Задача №4}

В данном примере на рисунке \ref{fig:chem1} показано моделирование пяти реакций с начальными концентрациями $\mu_{H_2} = 0.5$ и $\mu_{O_2} = 0.5$, нормальным
атмосферным давлением $P = 101325\text{Па}$ и температурой $T = 2300\text{К}$.

\begin{equation}
    \begin{cases}
        H_2 + O_2 \Longleftrightarrow 2OH\\
        H + O_2 \Longleftrightarrow OH + O\\
        H_2 + OH \Longleftrightarrow H_2O + H\\
        H_2 + O \Longleftrightarrow OH + H\\
        2OH \Longleftrightarrow H_2O + O
    \end{cases}
\label{eq:ChemTask1}
\end{equation}

\begin{figure}
    \begin{tikzpicture}
\begin{axis}[
	xlabel={$t, \text{с}$},
	ylabel={$\gamma, \text{моль}/\text{кг}$},
	xmin=0, xmax=2e-06,
	xtick={0,4e-07,8e-07,1.2e-06,1.6e-06,2e-06},
    legend style={at={(1,1.75)},
                  anchor=north east},
	%legend pos=outer north east,
	ymajorgrids=true,
	grid style=dashed,
]

\addplot[
	color=blue,
	mark=square,
	mark size=0.5pt
]
coordinates {
(0,29.3991)(1e-08,29.3986)(2e-08,29.398)(3e-08,29.3971)(4e-08,29.3962)(5e-08,29.395)(6e-08,29.3937)(7e-08,29.3923)(8e-08,29.3906)(9e-08,29.3887)(1e-07,29.3866)(1.1e-07,29.3841)(1.2e-07,29.3814)(1.3e-07,29.3782)(1.4e-07,29.3747)(1.5e-07,29.3707)(1.6e-07,29.3661)(1.7e-07,29.3609)(1.8e-07,29.355)(1.9e-07,29.3484)(2e-07,29.3408)(2.1e-07,29.3323)(2.2e-07,29.3226)(2.3e-07,29.3116)(2.4e-07,29.2991)(2.5e-07,29.2849)(2.6e-07,29.2689)(2.7e-07,29.2507)(2.8e-07,29.2302)(2.9e-07,29.2069)(3e-07,29.1805)(3.1e-07,29.1506)(3.2e-07,29.1167)(3.3e-07,29.0785)(3.4e-07,29.0351)(3.5e-07,28.9862)(3.6e-07,28.9308)(3.7e-07,28.8683)(3.8e-07,28.7978)(3.9e-07,28.7182)(4e-07,28.6284)(4.1e-07,28.5273)(4.2e-07,28.4135)(4.3e-07,28.2856)(4.4e-07,28.142)(4.5e-07,27.981)(4.6e-07,27.8008)(4.7e-07,27.5993)(4.8e-07,27.3746)(4.9e-07,27.1244)(5e-07,26.8466)(5.1e-07,26.5389)(5.2e-07,26.1991)(5.3e-07,25.8251)(5.4e-07,25.4149)(5.5e-07,24.9668)(5.6e-07,24.4795)(5.7e-07,23.952)(5.8e-07,23.3839)(5.9e-07,22.7756)(6e-07,22.1282)(6.1e-07,21.4436)(6.2e-07,20.7247)(6.3e-07,19.9752)(6.4e-07,19.1999)(6.5e-07,18.4043)(6.6e-07,17.5947)(6.7e-07,16.7778)(6.8e-07,15.9607)(6.9e-07,15.1505)(7e-07,14.3541)(7.1e-07,13.5782)(7.2e-07,12.8286)(7.3e-07,12.1103)(7.4e-07,11.4275)(7.5e-07,10.7834)(7.6e-07,10.18)(7.7e-07,9.61847)(7.8e-07,9.09921)(7.9e-07,8.6217)(8e-07,8.18483)(8.1e-07,7.78699)(8.2e-07,7.42617)(8.3e-07,7.10012)(8.4e-07,6.80644)(8.5e-07,6.54264)(8.6e-07,6.30626)(8.7e-07,6.09488)(8.8e-07,5.90618)(8.9e-07,5.73798)(9e-07,5.58823)(9.1e-07,5.45505)(9.2e-07,5.33669)(9.3e-07,5.23159)(9.4e-07,5.13829)(9.5e-07,5.05551)(9.6e-07,4.98209)(9.7e-07,4.91699)(9.8e-07,4.85926)(9.9e-07,4.80809)(1e-06,4.76273)(1.01e-06,4.72252)(1.02e-06,4.68688)(1.03e-06,4.65528)(1.04e-06,4.62728)(1.05e-06,4.60245)(1.06e-06,4.58044)(1.07e-06,4.56093)(1.08e-06,4.54363)(1.09e-06,4.52829)(1.1e-06,4.51469)(1.11e-06,4.50263)(1.12e-06,4.49194)(1.13e-06,4.48245)(1.14e-06,4.47404)(1.15e-06,4.46658)(1.16e-06,4.45997)(1.17e-06,4.4541)(1.18e-06,4.4489)(1.19e-06,4.44428)(1.2e-06,4.44018)(1.21e-06,4.43655)(1.22e-06,4.43333)(1.23e-06,4.43047)(1.24e-06,4.42793)(1.25e-06,4.42568)(1.26e-06,4.42369)(1.27e-06,4.42192)(1.28e-06,4.42035)(1.29e-06,4.41895)(1.3e-06,4.41772)(1.31e-06,4.41662)(1.32e-06,4.41565)(1.33e-06,4.41478)(1.34e-06,4.41402)(1.35e-06,4.41334)(1.36e-06,4.41274)(1.37e-06,4.4122)(1.38e-06,4.41173)(1.39e-06,4.41131)(1.4e-06,4.41093)(1.41e-06,4.4106)(1.42e-06,4.41031)(1.43e-06,4.41005)(1.44e-06,4.40981)(1.45e-06,4.40961)(1.46e-06,4.40943)(1.47e-06,4.40926)(1.48e-06,4.40912)(1.49e-06,4.40899)(1.5e-06,4.40888)(1.51e-06,4.40878)(1.52e-06,4.40869)(1.53e-06,4.40861)(1.54e-06,4.40854)(1.55e-06,4.40848)(1.56e-06,4.40843)(1.57e-06,4.40838)(1.58e-06,4.40833)(1.59e-06,4.40829)(1.6e-06,4.40826)(1.61e-06,4.40823)(1.62e-06,4.4082)(1.63e-06,4.40818)(1.64e-06,4.40816)(1.65e-06,4.40814)(1.66e-06,4.40812)(1.67e-06,4.40811)(1.68e-06,4.4081)(1.69e-06,4.40808)(1.7e-06,4.40807)(1.71e-06,4.40806)(1.72e-06,4.40806)(1.73e-06,4.40805)(1.74e-06,4.40804)(1.75e-06,4.40804)(1.76e-06,4.40803)(1.77e-06,4.40803)(1.78e-06,4.40802)(1.79e-06,4.40802)(1.8e-06,4.40802)(1.81e-06,4.40801)(1.82e-06,4.40801)(1.83e-06,4.40801)(1.84e-06,4.40801)(1.85e-06,4.40801)(1.86e-06,4.408)(1.87e-06,4.408)(1.88e-06,4.408)(1.89e-06,4.408)(1.9e-06,4.408)(1.91e-06,4.408)(1.92e-06,4.408)(1.93e-06,4.408)(1.94e-06,4.408)(1.95e-06,4.408)(1.96e-06,4.408)(1.97e-06,4.408)(1.98e-06,4.40799)(1.99e-06,4.40799)(2e-06,4.40799)
};
\addlegendentry{Приближённое решение H2}

\addplot[
	color=purple,
	mark=square,
	mark size=0.5pt
]
coordinates {
(0,0)(1e-08,0.000659696)(2e-08,0.00116715)(3e-08,0.00158627)(4e-08,0.0019614)(5e-08,0.00232354)(6e-08,0.00269483)(7e-08,0.00309175)(8e-08,0.00352736)(9e-08,0.0040128)(1e-07,0.00455842)(1.1e-07,0.0051745)(1.2e-07,0.00587177)(1.3e-07,0.00666189)(1.4e-07,0.00755773)(1.5e-07,0.00857368)(1.6e-07,0.00972597)(1.7e-07,0.0110329)(1.8e-07,0.0125152)(1.9e-07,0.0141962)(2e-07,0.0161026)(2.1e-07,0.0182644)(2.2e-07,0.0207156)(2.3e-07,0.0234948)(2.4e-07,0.0266454)(2.5e-07,0.0302167)(2.6e-07,0.0342644)(2.7e-07,0.0388514)(2.8e-07,0.0440486)(2.9e-07,0.0499363)(3e-07,0.0566047)(3.1e-07,0.0641557)(3.2e-07,0.0727036)(3.3e-07,0.0823773)(3.4e-07,0.0933213)(3.5e-07,0.105697)(3.6e-07,0.119687)(3.7e-07,0.135493)(3.8e-07,0.153341)(3.9e-07,0.173482)(4e-07,0.196194)(4.1e-07,0.221785)(4.2e-07,0.250593)(4.3e-07,0.282989)(4.4e-07,0.319379)(4.5e-07,0.360201)(4.6e-07,0.405927)(4.7e-07,0.457062)(4.8e-07,0.514139)(4.9e-07,0.577716)(5e-07,0.64837)(5.1e-07,0.726681)(5.2e-07,0.81323)(5.3e-07,0.908571)(5.4e-07,1.01322)(5.5e-07,1.12764)(5.6e-07,1.25218)(5.7e-07,1.38708)(5.8e-07,1.53244)(5.9e-07,1.68815)(6e-07,1.85391)(6.1e-07,2.02916)(6.2e-07,2.2131)(6.3e-07,2.40463)(6.4e-07,2.60242)(6.5e-07,2.80486)(6.6e-07,3.01015)(6.7e-07,3.21633)(6.8e-07,3.42135)(6.9e-07,3.62314)(7e-07,3.81971)(7.1e-07,4.00921)(7.2e-07,4.19)(7.3e-07,4.36073)(7.4e-07,4.52036)(7.5e-07,4.66816)(7.6e-07,4.80378)(7.7e-07,4.92712)(7.8e-07,5.0384)(7.9e-07,5.13804)(8e-07,5.22666)(8.1e-07,5.305)(8.2e-07,5.37388)(8.3e-07,5.43415)(8.4e-07,5.4867)(8.5e-07,5.53235)(8.6e-07,5.57192)(8.7e-07,5.60614)(8.8e-07,5.6357)(8.9e-07,5.66121)(9e-07,5.68322)(9.1e-07,5.70221)(9.2e-07,5.71859)(9.3e-07,5.73274)(9.4e-07,5.74498)(9.5e-07,5.75556)(9.6e-07,5.76473)(9.7e-07,5.77269)(9.8e-07,5.7796)(9.9e-07,5.78562)(1e-06,5.79086)(1.01e-06,5.79543)(1.02e-06,5.79942)(1.03e-06,5.80292)(1.04e-06,5.80598)(1.05e-06,5.80866)(1.06e-06,5.81102)(1.07e-06,5.81309)(1.08e-06,5.81491)(1.09e-06,5.81651)(1.1e-06,5.81793)(1.11e-06,5.81917)(1.12e-06,5.82027)(1.13e-06,5.82124)(1.14e-06,5.8221)(1.15e-06,5.82286)(1.16e-06,5.82352)(1.17e-06,5.82412)(1.18e-06,5.82464)(1.19e-06,5.8251)(1.2e-06,5.82551)(1.21e-06,5.82588)(1.22e-06,5.8262)(1.23e-06,5.82648)(1.24e-06,5.82674)(1.25e-06,5.82696)(1.26e-06,5.82716)(1.27e-06,5.82733)(1.28e-06,5.82749)(1.29e-06,5.82763)(1.3e-06,5.82775)(1.31e-06,5.82786)(1.32e-06,5.82795)(1.33e-06,5.82804)(1.34e-06,5.82812)(1.35e-06,5.82818)(1.36e-06,5.82824)(1.37e-06,5.8283)(1.38e-06,5.82834)(1.39e-06,5.82838)(1.4e-06,5.82842)(1.41e-06,5.82845)(1.42e-06,5.82848)(1.43e-06,5.82851)(1.44e-06,5.82853)(1.45e-06,5.82855)(1.46e-06,5.82857)(1.47e-06,5.82859)(1.48e-06,5.8286)(1.49e-06,5.82861)(1.5e-06,5.82862)(1.51e-06,5.82863)(1.52e-06,5.82864)(1.53e-06,5.82865)(1.54e-06,5.82866)(1.55e-06,5.82866)(1.56e-06,5.82867)(1.57e-06,5.82867)(1.58e-06,5.82868)(1.59e-06,5.82868)(1.6e-06,5.82868)(1.61e-06,5.82869)(1.62e-06,5.82869)(1.63e-06,5.82869)(1.64e-06,5.82869)(1.65e-06,5.8287)(1.66e-06,5.8287)(1.67e-06,5.8287)(1.68e-06,5.8287)(1.69e-06,5.8287)(1.7e-06,5.8287)(1.71e-06,5.8287)(1.72e-06,5.8287)(1.73e-06,5.8287)(1.74e-06,5.82871)(1.75e-06,5.82871)(1.76e-06,5.82871)(1.77e-06,5.82871)(1.78e-06,5.82871)(1.79e-06,5.82871)(1.8e-06,5.82871)(1.81e-06,5.82871)(1.82e-06,5.82871)(1.83e-06,5.82871)(1.84e-06,5.82871)(1.85e-06,5.82871)(1.86e-06,5.82871)(1.87e-06,5.82871)(1.88e-06,5.82871)(1.89e-06,5.82871)(1.9e-06,5.82871)(1.91e-06,5.82871)(1.92e-06,5.82871)(1.93e-06,5.82871)(1.94e-06,5.82871)(1.95e-06,5.82871)(1.96e-06,5.82871)(1.97e-06,5.82871)(1.98e-06,5.82871)(1.99e-06,5.82871)(2e-06,5.82871)
};
\addlegendentry{Приближённое решение OH}

\addplot[
	color=red,
	mark=square,
	mark size=0.5pt
]
coordinates {
(0,0)(1e-08,0.000109069)(2e-08,0.000400144)(3e-08,0.000835904)(4e-08,0.0013959)(5e-08,0.00207143)(6e-08,0.00286205)(7e-08,0.00377336)(8e-08,0.00481554)(9e-08,0.00600255)(1e-07,0.00735171)(1.1e-07,0.00888356)(1.2e-07,0.0106219)(1.3e-07,0.0125942)(1.4e-07,0.0148314)(1.5e-07,0.0173692)(1.6e-07,0.0202475)(1.7e-07,0.0235121)(1.8e-07,0.0272147)(1.9e-07,0.0314137)(2e-07,0.0361754)(2.1e-07,0.0415749)(2.2e-07,0.0476972)(2.3e-07,0.0546384)(2.4e-07,0.0625072)(2.5e-07,0.0714268)(2.6e-07,0.081536)(2.7e-07,0.092992)(2.8e-07,0.105972)(2.9e-07,0.120676)(3e-07,0.137329)(3.1e-07,0.156187)(3.2e-07,0.177533)(3.3e-07,0.20169)(3.4e-07,0.229019)(3.5e-07,0.259923)(3.6e-07,0.294855)(3.7e-07,0.334321)(3.8e-07,0.378884)(3.9e-07,0.42917)(4e-07,0.485874)(4.1e-07,0.549762)(4.2e-07,0.62168)(4.3e-07,0.702554)(4.4e-07,0.793394)(4.5e-07,0.895296)(4.6e-07,1.00944)(4.7e-07,1.13709)(4.8e-07,1.27958)(4.9e-07,1.43831)(5e-07,1.61474)(5.1e-07,1.81033)(5.2e-07,2.02657)(5.3e-07,2.26488)(5.4e-07,2.52661)(5.5e-07,2.813)(5.6e-07,3.12506)(5.7e-07,3.46356)(5.8e-07,3.82893)(5.9e-07,4.22124)(6e-07,4.64004)(6.1e-07,5.08443)(6.2e-07,5.55289)(6.3e-07,6.04335)(6.4e-07,6.55317)(6.5e-07,7.07916)(6.6e-07,7.61765)(6.7e-07,8.16458)(6.8e-07,8.71563)(6.9e-07,9.26635)(7e-07,9.81231)(7.1e-07,10.3492)(7.2e-07,10.8731)(7.3e-07,11.3805)(7.4e-07,11.8681)(7.5e-07,12.3336)(7.6e-07,12.7749)(7.7e-07,13.1908)(7.8e-07,13.5803)(7.9e-07,13.9432)(8e-07,14.2796)(8.1e-07,14.59)(8.2e-07,14.8752)(8.3e-07,15.1362)(8.4e-07,15.3744)(8.5e-07,15.5909)(8.6e-07,15.7872)(8.7e-07,15.9648)(8.8e-07,16.1251)(8.9e-07,16.2695)(9e-07,16.3993)(9.1e-07,16.5158)(9.2e-07,16.6203)(9.3e-07,16.7138)(9.4e-07,16.7975)(9.5e-07,16.8722)(9.6e-07,16.9389)(9.7e-07,16.9985)(9.8e-07,17.0516)(9.9e-07,17.0989)(1e-06,17.141)(1.01e-06,17.1785)(1.02e-06,17.2119)(1.03e-06,17.2415)(1.04e-06,17.2679)(1.05e-06,17.2914)(1.06e-06,17.3123)(1.07e-06,17.3308)(1.08e-06,17.3473)(1.09e-06,17.3619)(1.1e-06,17.3749)(1.11e-06,17.3864)(1.12e-06,17.3967)(1.13e-06,17.4058)(1.14e-06,17.4138)(1.15e-06,17.421)(1.16e-06,17.4274)(1.17e-06,17.433)(1.18e-06,17.438)(1.19e-06,17.4425)(1.2e-06,17.4464)(1.21e-06,17.45)(1.22e-06,17.4531)(1.23e-06,17.4558)(1.24e-06,17.4583)(1.25e-06,17.4605)(1.26e-06,17.4624)(1.27e-06,17.4641)(1.28e-06,17.4656)(1.29e-06,17.467)(1.3e-06,17.4682)(1.31e-06,17.4692)(1.32e-06,17.4702)(1.33e-06,17.471)(1.34e-06,17.4717)(1.35e-06,17.4724)(1.36e-06,17.473)(1.37e-06,17.4735)(1.38e-06,17.474)(1.39e-06,17.4744)(1.4e-06,17.4747)(1.41e-06,17.4751)(1.42e-06,17.4753)(1.43e-06,17.4756)(1.44e-06,17.4758)(1.45e-06,17.476)(1.46e-06,17.4762)(1.47e-06,17.4763)(1.48e-06,17.4765)(1.49e-06,17.4766)(1.5e-06,17.4767)(1.51e-06,17.4768)(1.52e-06,17.4769)(1.53e-06,17.477)(1.54e-06,17.477)(1.55e-06,17.4771)(1.56e-06,17.4772)(1.57e-06,17.4772)(1.58e-06,17.4772)(1.59e-06,17.4773)(1.6e-06,17.4773)(1.61e-06,17.4773)(1.62e-06,17.4774)(1.63e-06,17.4774)(1.64e-06,17.4774)(1.65e-06,17.4774)(1.66e-06,17.4775)(1.67e-06,17.4775)(1.68e-06,17.4775)(1.69e-06,17.4775)(1.7e-06,17.4775)(1.71e-06,17.4775)(1.72e-06,17.4775)(1.73e-06,17.4775)(1.74e-06,17.4775)(1.75e-06,17.4775)(1.76e-06,17.4775)(1.77e-06,17.4775)(1.78e-06,17.4775)(1.79e-06,17.4776)(1.8e-06,17.4776)(1.81e-06,17.4776)(1.82e-06,17.4776)(1.83e-06,17.4776)(1.84e-06,17.4776)(1.85e-06,17.4776)(1.86e-06,17.4776)(1.87e-06,17.4776)(1.88e-06,17.4776)(1.89e-06,17.4776)(1.9e-06,17.4776)(1.91e-06,17.4776)(1.92e-06,17.4776)(1.93e-06,17.4776)(1.94e-06,17.4776)(1.95e-06,17.4776)(1.96e-06,17.4776)(1.97e-06,17.4776)(1.98e-06,17.4776)(1.99e-06,17.4776)(2e-06,17.4776)
};
\addlegendentry{Приближённое решение H2O}

\addplot[
	color=black,
	mark=square,
	mark size=0.5pt
]
coordinates {
(0,29.3991)(1e-08,29.3988)(2e-08,29.3983)(3e-08,29.3979)(4e-08,29.3974)(5e-08,29.3968)(6e-08,29.3961)(7e-08,29.3954)(8e-08,29.3945)(9e-08,29.3936)(1e-07,29.3925)(1.1e-07,29.3912)(1.2e-07,29.3898)(1.3e-07,29.3882)(1.4e-07,29.3864)(1.5e-07,29.3843)(1.6e-07,29.382)(1.7e-07,29.3793)(1.8e-07,29.3763)(1.9e-07,29.3729)(2e-07,29.369)(2.1e-07,29.3646)(2.2e-07,29.3596)(2.3e-07,29.354)(2.4e-07,29.3475)(2.5e-07,29.3403)(2.6e-07,29.332)(2.7e-07,29.3227)(2.8e-07,29.3121)(2.9e-07,29.3001)(3e-07,29.2866)(3.1e-07,29.2712)(3.2e-07,29.2538)(3.3e-07,29.2341)(3.4e-07,29.2118)(3.5e-07,29.1866)(3.6e-07,29.158)(3.7e-07,29.1258)(3.8e-07,29.0894)(3.9e-07,29.0483)(4e-07,29.0019)(4.1e-07,28.9497)(4.2e-07,28.8908)(4.3e-07,28.8245)(4.4e-07,28.7501)(4.5e-07,28.6665)(4.6e-07,28.5727)(4.7e-07,28.4677)(4.8e-07,28.3504)(4.9e-07,28.2195)(5e-07,28.0738)(5.1e-07,27.912)(5.2e-07,27.7328)(5.3e-07,27.5349)(5.4e-07,27.3169)(5.5e-07,27.0778)(5.6e-07,26.8165)(5.7e-07,26.5321)(5.8e-07,26.2241)(5.9e-07,25.8921)(6e-07,25.5361)(6.1e-07,25.1567)(6.2e-07,24.7548)(6.3e-07,24.3317)(6.4e-07,23.8896)(6.5e-07,23.4308)(6.6e-07,22.9581)(6.7e-07,22.475)(6.8e-07,21.985)(6.9e-07,21.492)(7e-07,20.9998)(7.1e-07,20.5124)(7.2e-07,20.0334)(7.3e-07,19.5663)(7.4e-07,19.1142)(7.5e-07,18.6796)(7.6e-07,18.2647)(7.7e-07,17.8711)(7.8e-07,17.5)(7.9e-07,17.1521)(8e-07,16.8275)(8.1e-07,16.5263)(8.2e-07,16.2478)(8.3e-07,15.9915)(8.4e-07,15.7564)(8.5e-07,15.5415)(8.6e-07,15.3456)(8.7e-07,15.1676)(8.8e-07,15.0062)(8.9e-07,14.8601)(9e-07,14.7283)(9.1e-07,14.6094)(9.2e-07,14.5025)(9.3e-07,14.4064)(9.4e-07,14.3201)(9.5e-07,14.2428)(9.6e-07,14.1736)(9.7e-07,14.1116)(9.8e-07,14.0563)(9.9e-07,14.0068)(1e-06,13.9627)(1.01e-06,13.9234)(1.02e-06,13.8883)(1.03e-06,13.857)(1.04e-06,13.8292)(1.05e-06,13.8044)(1.06e-06,13.7823)(1.07e-06,13.7627)(1.08e-06,13.7452)(1.09e-06,13.7297)(1.1e-06,13.7159)(1.11e-06,13.7036)(1.12e-06,13.6927)(1.13e-06,13.6831)(1.14e-06,13.6745)(1.15e-06,13.6668)(1.16e-06,13.66)(1.17e-06,13.654)(1.18e-06,13.6487)(1.19e-06,13.6439)(1.2e-06,13.6397)(1.21e-06,13.636)(1.22e-06,13.6326)(1.23e-06,13.6297)(1.24e-06,13.6271)(1.25e-06,13.6247)(1.26e-06,13.6227)(1.27e-06,13.6208)(1.28e-06,13.6192)(1.29e-06,13.6178)(1.3e-06,13.6165)(1.31e-06,13.6154)(1.32e-06,13.6144)(1.33e-06,13.6135)(1.34e-06,13.6127)(1.35e-06,13.612)(1.36e-06,13.6113)(1.37e-06,13.6108)(1.38e-06,13.6103)(1.39e-06,13.6099)(1.4e-06,13.6095)(1.41e-06,13.6091)(1.42e-06,13.6088)(1.43e-06,13.6086)(1.44e-06,13.6083)(1.45e-06,13.6081)(1.46e-06,13.6079)(1.47e-06,13.6077)(1.48e-06,13.6076)(1.49e-06,13.6075)(1.5e-06,13.6073)(1.51e-06,13.6072)(1.52e-06,13.6071)(1.53e-06,13.6071)(1.54e-06,13.607)(1.55e-06,13.6069)(1.56e-06,13.6069)(1.57e-06,13.6068)(1.58e-06,13.6068)(1.59e-06,13.6067)(1.6e-06,13.6067)(1.61e-06,13.6067)(1.62e-06,13.6066)(1.63e-06,13.6066)(1.64e-06,13.6066)(1.65e-06,13.6066)(1.66e-06,13.6066)(1.67e-06,13.6065)(1.68e-06,13.6065)(1.69e-06,13.6065)(1.7e-06,13.6065)(1.71e-06,13.6065)(1.72e-06,13.6065)(1.73e-06,13.6065)(1.74e-06,13.6065)(1.75e-06,13.6065)(1.76e-06,13.6065)(1.77e-06,13.6065)(1.78e-06,13.6065)(1.79e-06,13.6065)(1.8e-06,13.6064)(1.81e-06,13.6064)(1.82e-06,13.6064)(1.83e-06,13.6064)(1.84e-06,13.6064)(1.85e-06,13.6064)(1.86e-06,13.6064)(1.87e-06,13.6064)(1.88e-06,13.6064)(1.89e-06,13.6064)(1.9e-06,13.6064)(1.91e-06,13.6064)(1.92e-06,13.6064)(1.93e-06,13.6064)(1.94e-06,13.6064)(1.95e-06,13.6064)(1.96e-06,13.6064)(1.97e-06,13.6064)(1.98e-06,13.6064)(1.99e-06,13.6064)(2e-06,13.6064)
};
\addlegendentry{Приближённое решение O2}

\addplot[
	color=magenta,
	mark=square,
	mark size=0.5pt
]
coordinates {
(0,0)(1e-08,0.000104646)(2e-08,0.000370354)(3e-08,0.000750357)(4e-08,0.00122119)(5e-08,0.00177372)(6e-08,0.00240765)(7e-08,0.00312837)(8e-08,0.00394503)(9e-08,0.00486965)(1e-07,0.00591662)(1.1e-07,0.00710258)(1.2e-07,0.00844652)(1.3e-07,0.00996992)(1.4e-07,0.0116971)(1.5e-07,0.0136557)(1.6e-07,0.0158767)(1.7e-07,0.0183953)(1.8e-07,0.0212515)(1.9e-07,0.0244903)(2e-07,0.0281628)(2.1e-07,0.0323269)(2.2e-07,0.0370478)(2.3e-07,0.0423995)(2.4e-07,0.0484656)(2.5e-07,0.0553407)(2.6e-07,0.0631313)(2.7e-07,0.0719581)(2.8e-07,0.0819569)(2.9e-07,0.0932809)(3e-07,0.106103)(3.1e-07,0.120616)(3.2e-07,0.137039)(3.3e-07,0.155617)(3.4e-07,0.176624)(3.5e-07,0.200367)(3.6e-07,0.227188)(3.7e-07,0.257468)(3.8e-07,0.291633)(3.9e-07,0.33015)(4e-07,0.373539)(4.1e-07,0.42237)(4.2e-07,0.477266)(4.3e-07,0.538907)(4.4e-07,0.608028)(4.5e-07,0.685419)(4.6e-07,0.771921)(4.7e-07,0.868422)(4.8e-07,0.975846)(4.9e-07,1.09514)(5e-07,1.22726)(5.1e-07,1.37316)(5.2e-07,1.53372)(5.3e-07,1.70976)(5.4e-07,1.902)(5.5e-07,2.11097)(5.6e-07,2.33701)(5.7e-07,2.58016)(5.8e-07,2.84018)(5.9e-07,3.11644)(6e-07,3.40791)(6.1e-07,3.7131)(6.2e-07,4.0301)(6.3e-07,4.35655)(6.4e-07,4.68968)(6.5e-07,5.02641)(6.6e-07,5.36339)(6.7e-07,5.69718)(6.8e-07,6.02432)(6.9e-07,6.34149)(7e-07,6.64565)(7.1e-07,6.93415)(7.2e-07,7.2048)(7.3e-07,7.45599)(7.4e-07,7.68662)(7.5e-07,7.8962)(7.6e-07,8.08471)(7.7e-07,8.25263)(7.8e-07,8.40081)(7.9e-07,8.53038)(8e-07,8.64271)(8.1e-07,8.73928)(8.2e-07,8.82166)(8.3e-07,8.89138)(8.4e-07,8.94998)(8.5e-07,8.99887)(8.6e-07,9.03939)(8.7e-07,9.07274)(8.8e-07,9.10002)(8.9e-07,9.12217)(9e-07,9.14005)(9.1e-07,9.15437)(9.2e-07,9.16577)(9.3e-07,9.17477)(9.4e-07,9.18181)(9.5e-07,9.18728)(9.6e-07,9.19147)(9.7e-07,9.19464)(9.8e-07,9.19701)(9.9e-07,9.19875)(1e-06,9.19999)(1.01e-06,9.20084)(1.02e-06,9.20139)(1.03e-06,9.20172)(1.04e-06,9.20189)(1.05e-06,9.20192)(1.06e-06,9.20187)(1.07e-06,9.20176)(1.08e-06,9.2016)(1.09e-06,9.20141)(1.1e-06,9.20122)(1.11e-06,9.20101)(1.12e-06,9.20081)(1.13e-06,9.20061)(1.14e-06,9.20042)(1.15e-06,9.20024)(1.16e-06,9.20007)(1.17e-06,9.19991)(1.18e-06,9.19977)(1.19e-06,9.19963)(1.2e-06,9.19951)(1.21e-06,9.1994)(1.22e-06,9.1993)(1.23e-06,9.19921)(1.24e-06,9.19913)(1.25e-06,9.19905)(1.26e-06,9.19898)(1.27e-06,9.19893)(1.28e-06,9.19887)(1.29e-06,9.19882)(1.3e-06,9.19878)(1.31e-06,9.19874)(1.32e-06,9.19871)(1.33e-06,9.19868)(1.34e-06,9.19865)(1.35e-06,9.19863)(1.36e-06,9.1986)(1.37e-06,9.19858)(1.38e-06,9.19857)(1.39e-06,9.19855)(1.4e-06,9.19854)(1.41e-06,9.19853)(1.42e-06,9.19852)(1.43e-06,9.19851)(1.44e-06,9.1985)(1.45e-06,9.19849)(1.46e-06,9.19848)(1.47e-06,9.19848)(1.48e-06,9.19847)(1.49e-06,9.19847)(1.5e-06,9.19846)(1.51e-06,9.19846)(1.52e-06,9.19846)(1.53e-06,9.19845)(1.54e-06,9.19845)(1.55e-06,9.19845)(1.56e-06,9.19845)(1.57e-06,9.19845)(1.58e-06,9.19844)(1.59e-06,9.19844)(1.6e-06,9.19844)(1.61e-06,9.19844)(1.62e-06,9.19844)(1.63e-06,9.19844)(1.64e-06,9.19844)(1.65e-06,9.19844)(1.66e-06,9.19844)(1.67e-06,9.19844)(1.68e-06,9.19843)(1.69e-06,9.19843)(1.7e-06,9.19843)(1.71e-06,9.19843)(1.72e-06,9.19843)(1.73e-06,9.19843)(1.74e-06,9.19843)(1.75e-06,9.19843)(1.76e-06,9.19843)(1.77e-06,9.19843)(1.78e-06,9.19843)(1.79e-06,9.19843)(1.8e-06,9.19843)(1.81e-06,9.19843)(1.82e-06,9.19843)(1.83e-06,9.19843)(1.84e-06,9.19843)(1.85e-06,9.19843)(1.86e-06,9.19843)(1.87e-06,9.19843)(1.88e-06,9.19843)(1.89e-06,9.19843)(1.9e-06,9.19843)(1.91e-06,9.19843)(1.92e-06,9.19843)(1.93e-06,9.19843)(1.94e-06,9.19843)(1.95e-06,9.19843)(1.96e-06,9.19843)(1.97e-06,9.19843)(1.98e-06,9.19843)(1.99e-06,9.19843)(2e-06,9.19843)
};
\addlegendentry{Приближённое решение H}

\addplot[
	color=green,
	mark=square,
	mark size=0.5pt
]
coordinates {
(0,0)(1e-08,4.42326e-06)(2e-08,2.97897e-05)(3e-08,8.55471e-05)(4e-08,0.00017471)(5e-08,0.000297712)(6e-08,0.000454401)(7e-08,0.000644995)(8e-08,0.000870509)(9e-08,0.0011329)(1e-07,0.00143509)(1.1e-07,0.00178098)(1.2e-07,0.00217543)(1.3e-07,0.00262426)(1.4e-07,0.0031343)(1.5e-07,0.00371348)(1.6e-07,0.00437087)(1.7e-07,0.00511684)(1.8e-07,0.00596319)(1.9e-07,0.00692334)(2e-07,0.00801253)(2.1e-07,0.00924802)(2.2e-07,0.0106494)(2.3e-07,0.0122389)(2.4e-07,0.0140416)(2.5e-07,0.0160861)(2.6e-07,0.0184047)(2.7e-07,0.0210338)(2.8e-07,0.024015)(2.9e-07,0.027395)(3e-07,0.0312269)(3.1e-07,0.0355705)(3.2e-07,0.0404936)(3.3e-07,0.0460728)(3.4e-07,0.0523944)(3.5e-07,0.059556)(3.6e-07,0.0676675)(3.7e-07,0.0768528)(3.8e-07,0.0872514)(3.9e-07,0.0990199)(4e-07,0.112334)(4.1e-07,0.127392)(4.2e-07,0.144414)(4.3e-07,0.163647)(4.4e-07,0.185366)(4.5e-07,0.209877)(4.6e-07,0.237519)(4.7e-07,0.268667)(4.8e-07,0.303734)(4.9e-07,0.343172)(5e-07,0.387474)(5.1e-07,0.437175)(5.2e-07,0.492849)(5.3e-07,0.555112)(5.4e-07,0.624612)(5.5e-07,0.702026)(5.6e-07,0.788051)(5.7e-07,0.883394)(5.8e-07,0.988751)(5.9e-07,1.10479)(6e-07,1.23214)(6.1e-07,1.37132)(6.2e-07,1.52278)(6.3e-07,1.6868)(6.4e-07,1.86349)(6.5e-07,2.05275)(6.6e-07,2.25425)(6.7e-07,2.46739)(6.8e-07,2.69131)(6.9e-07,2.92486)(7e-07,3.16666)(7.1e-07,3.41509)(7.2e-07,3.66834)(7.3e-07,3.92448)(7.4e-07,4.1815)(7.5e-07,4.4374)(7.6e-07,4.69023)(7.7e-07,4.93816)(7.8e-07,5.17952)(7.9e-07,5.41285)(8e-07,5.63691)(8.1e-07,5.85073)(8.2e-07,6.05355)(8.3e-07,6.24486)(8.4e-07,6.42438)(8.5e-07,6.59201)(8.6e-07,6.74784)(8.7e-07,6.89208)(8.8e-07,7.02509)(8.9e-07,7.1473)(9e-07,7.25923)(9.1e-07,7.36143)(9.2e-07,7.45449)(9.3e-07,7.53903)(9.4e-07,7.61564)(9.5e-07,7.68493)(9.6e-07,7.74748)(9.7e-07,7.80384)(9.8e-07,7.85455)(9.9e-07,7.90012)(1e-06,7.94101)(1.01e-06,7.97765)(1.02e-06,8.01046)(1.03e-06,8.03981)(1.04e-06,8.06605)(1.05e-06,8.08948)(1.06e-06,8.11038)(1.07e-06,8.12903)(1.08e-06,8.14566)(1.09e-06,8.16047)(1.1e-06,8.17366)(1.11e-06,8.1854)(1.12e-06,8.19585)(1.13e-06,8.20515)(1.14e-06,8.21342)(1.15e-06,8.22077)(1.16e-06,8.22731)(1.17e-06,8.23311)(1.18e-06,8.23827)(1.19e-06,8.24286)(1.2e-06,8.24693)(1.21e-06,8.25055)(1.22e-06,8.25376)(1.23e-06,8.25662)(1.24e-06,8.25915)(1.25e-06,8.2614)(1.26e-06,8.26339)(1.27e-06,8.26517)(1.28e-06,8.26674)(1.29e-06,8.26814)(1.3e-06,8.26938)(1.31e-06,8.27048)(1.32e-06,8.27145)(1.33e-06,8.27232)(1.34e-06,8.27309)(1.35e-06,8.27377)(1.36e-06,8.27438)(1.37e-06,8.27491)(1.38e-06,8.27539)(1.39e-06,8.27581)(1.4e-06,8.27619)(1.41e-06,8.27652)(1.42e-06,8.27682)(1.43e-06,8.27708)(1.44e-06,8.27731)(1.45e-06,8.27752)(1.46e-06,8.2777)(1.47e-06,8.27787)(1.48e-06,8.27801)(1.49e-06,8.27814)(1.5e-06,8.27825)(1.51e-06,8.27835)(1.52e-06,8.27844)(1.53e-06,8.27852)(1.54e-06,8.27859)(1.55e-06,8.27865)(1.56e-06,8.27871)(1.57e-06,8.27876)(1.58e-06,8.2788)(1.59e-06,8.27884)(1.6e-06,8.27888)(1.61e-06,8.27891)(1.62e-06,8.27893)(1.63e-06,8.27896)(1.64e-06,8.27898)(1.65e-06,8.279)(1.66e-06,8.27901)(1.67e-06,8.27903)(1.68e-06,8.27904)(1.69e-06,8.27905)(1.7e-06,8.27906)(1.71e-06,8.27907)(1.72e-06,8.27908)(1.73e-06,8.27909)(1.74e-06,8.2791)(1.75e-06,8.2791)(1.76e-06,8.27911)(1.77e-06,8.27911)(1.78e-06,8.27912)(1.79e-06,8.27912)(1.8e-06,8.27912)(1.81e-06,8.27912)(1.82e-06,8.27913)(1.83e-06,8.27913)(1.84e-06,8.27913)(1.85e-06,8.27913)(1.86e-06,8.27913)(1.87e-06,8.27914)(1.88e-06,8.27914)(1.89e-06,8.27914)(1.9e-06,8.27914)(1.91e-06,8.27914)(1.92e-06,8.27914)(1.93e-06,8.27914)(1.94e-06,8.27914)(1.95e-06,8.27914)(1.96e-06,8.27914)(1.97e-06,8.27914)(1.98e-06,8.27914)(1.99e-06,8.27914)(2e-06,8.27914)
};
\addlegendentry{Приближённое решение O}

\end{axis}
\end{tikzpicture}
    \caption{Решение задачи №4}
    \label{fig:chem1}
\end{figure}

По графику видно, что реакция протекает примерно за 1 микросекунду после чего система приходит в равновесие.

По оси X указано время, по оси Y ~--- мольно-массовые концентрации веществ. Так как концентрации $H_2$ и $O_2$ заданы по $0.5$, то их
мольно-массовые концентрации оказались равны примерно $29.3991$ моль/кг.

Для решения данной задачи использовался неявный метод Гаусса 6го порядка с итерациями Зейделя, однако для неё можно использовать и
другие неявные методы. В таблице \ref{tab:Methods1} приведены сравнения некоторых неявных и явных методов для решения данной задачи по
времени и количеству шагов.

\begin{table}    
    \caption{Таблица сравнения методов}
    \begin{tabularx}{\textwidth}{|X|c|c|c|}
    \hline
    Метод & Время работы (мс) & Количество шагов & Точность\\
    \hline
    Неявный Гаусс 6-го порядка & 254 & 35 & 0.001\\
    \hline
    Неявный Гаусс 4-го порядка & 11 & 35 & 0.001\\
    \hline
    Неявный Гаусс 2-го порядка & 235 & 50 & 0.001\\
    \hline
    Явный Рунге-Кутта 6-го порядка & 92 & 100 & 0.001\\
    \hline
    Явный Рунге-Кутта 4-го порядка & 55 & 100 & 0.001\\
    \hline
    Вложенный Дормана-Принс 4(5)-го порядка & 191 & 200 & 0.001\\
    \hline
    Вложенный Фалберг 2(3)-го порядка & 97 & 200 & 0.001\\
    \hline
    \end{tabularx}
    \label{tab:Methods1}
\end{table}

\textit{Задача №5}

В следующем примере на рисунке \ref{fig:chem2} показано моделирование семи реакций с начальными концентрациями $\mu_{H_2} = 0.5$ и $\mu_{O_2} = 0.5$, нормальным
атмосферным давлением $P = 101325\text{Па}$ и температурой $T = 2300K$. Данный пример отличается от предыдущего наличием добавки $M$ в последних двух реакциях.

\begin{equation}
    \begin{cases}
        H_2 + O_2 \Longleftrightarrow 2OH\\
        H + O_2 \Longleftrightarrow OH + O\\
        H_2 + OH \Longleftrightarrow H_2O + H\\
        H_2 + O \Longleftrightarrow OH + H\\
        2OH \Longleftrightarrow H_2O + O\\
        H + OH + M \Longleftrightarrow H_2O + M\\
        2H + M \Longleftrightarrow H_2 + M
    \end{cases}
\label{eq:ChemTask2}
\end{equation}

\begin{figure}
    \begin{tikzpicture}
    \begin{axis}[
        enlargelimits=true,
        xlabel={$t, \text{с}$},
	    ylabel={$\gamma, \text{моль}/\text{кг}$},
        legend style={at={(1,1.75)},
                  anchor=north east},
        %legend pos=outer north east,
        ymajorgrids=true,
        grid style=dashed,
    ]
    
    \addplot[
        color=blue,
        mark=square,
        mark size=0.5pt
    ]
    coordinates {
    (0,29.3991)(1e-08,29.3979)(2e-08,29.3964)(3e-08,29.3945)(4e-08,29.3922)(5e-08,29.3895)(6e-08,29.3862)(7e-08,29.3823)(8e-08,29.3776)(9e-08,29.3722)(1e-07,29.3658)(1.1e-07,29.3584)(1.2e-07,29.3499)(1.3e-07,29.3399)(1.4e-07,29.3285)(1.5e-07,29.3153)(1.6e-07,29.3002)(1.7e-07,29.283)(1.8e-07,29.2632)(1.9e-07,29.2406)(2e-07,29.2148)(2.1e-07,29.1855)(2.2e-07,29.1521)(2.3e-07,29.1142)(2.4e-07,29.0711)(2.5e-07,29.0223)(2.6e-07,28.967)(2.7e-07,28.9044)(2.8e-07,28.8337)(2.9e-07,28.7539)(3e-07,28.664)(3.1e-07,28.5627)(3.2e-07,28.4488)(3.3e-07,28.321)(3.4e-07,28.1778)(3.5e-07,28.0177)(3.6e-07,27.8391)(3.7e-07,27.6401)(3.8e-07,27.4192)(3.9e-07,27.1745)(4e-07,26.9042)(4.1e-07,26.6068)(4.2e-07,26.2806)(4.3e-07,25.9242)(4.4e-07,25.5365)(4.5e-07,25.1167)(4.6e-07,24.6642)(4.7e-07,24.179)(4.8e-07,23.6616)(4.9e-07,23.113)(5e-07,22.5349)(5.1e-07,21.9294)(5.2e-07,21.2994)(5.3e-07,20.6481)(5.4e-07,19.9794)(5.5e-07,19.2974)(5.6e-07,18.6065)(5.7e-07,17.9113)(5.8e-07,17.2165)(5.9e-07,16.5263)(6e-07,15.8451)(6.1e-07,15.1768)(6.2e-07,14.5247)(6.3e-07,13.892)(6.4e-07,13.281)(6.5e-07,12.6939)(6.6e-07,12.132)(6.7e-07,11.5963)(6.8e-07,11.0876)(6.9e-07,10.6058)(7e-07,10.1508)(7.1e-07,9.72228)(7.2e-07,9.31944)(7.3e-07,8.94145)(7.4e-07,8.58729)(7.5e-07,8.25588)(7.6e-07,7.94604)(7.7e-07,7.65658)(7.8e-07,7.38629)(7.9e-07,7.13397)(8e-07,6.89845)(8.1e-07,6.67859)(8.2e-07,6.47331)(8.3e-07,6.28157)(8.4e-07,6.10238)(8.5e-07,5.93483)(8.6e-07,5.77805)(8.7e-07,5.63122)(8.8e-07,5.4936)(8.9e-07,5.36448)(9e-07,5.24321)(9.1e-07,5.1292)(9.2e-07,5.02189)(9.3e-07,4.92077)(9.4e-07,4.82537)(9.5e-07,4.73525)(9.6e-07,4.65002)(9.7e-07,4.56932)(9.8e-07,4.4928)(9.9e-07,4.42017)(1e-06,4.35112)(1.01e-06,4.28542)(1.02e-06,4.2228)(1.03e-06,4.16307)(1.04e-06,4.106)(1.05e-06,4.05143)(1.06e-06,3.99918)(1.07e-06,3.94909)(1.08e-06,3.90101)(1.09e-06,3.85482)(1.1e-06,3.8104)(1.11e-06,3.76763)(1.12e-06,3.7264)(1.13e-06,3.68663)(1.14e-06,3.64822)(1.15e-06,3.6111)(1.16e-06,3.57518)(1.17e-06,3.5404)(1.18e-06,3.5067)(1.19e-06,3.47401)(1.2e-06,3.44228)(1.21e-06,3.41145)(1.22e-06,3.38149)(1.23e-06,3.35234)(1.24e-06,3.32396)(1.25e-06,3.29633)(1.26e-06,3.26939)(1.27e-06,3.24312)(1.28e-06,3.21749)(1.29e-06,3.19246)(1.3e-06,3.16802)(1.31e-06,3.14413)(1.32e-06,3.12077)(1.33e-06,3.09792)(1.34e-06,3.07556)(1.35e-06,3.05367)(1.36e-06,3.03223)(1.37e-06,3.01122)(1.38e-06,2.99063)(1.39e-06,2.97044)(1.4e-06,2.95063)(1.41e-06,2.9312)(1.42e-06,2.91213)(1.43e-06,2.8934)(1.44e-06,2.87502)(1.45e-06,2.85695)(1.46e-06,2.8392)(1.47e-06,2.82175)(1.48e-06,2.8046)(1.49e-06,2.78774)(1.5e-06,2.77115)(1.51e-06,2.75483)(1.52e-06,2.73877)(1.53e-06,2.72297)(1.54e-06,2.70741)(1.55e-06,2.69209)(1.56e-06,2.67701)(1.57e-06,2.66215)(1.58e-06,2.64751)(1.59e-06,2.63309)(1.6e-06,2.61888)(1.61e-06,2.60487)(1.62e-06,2.59107)(1.63e-06,2.57745)(1.64e-06,2.56403)(1.65e-06,2.55079)(1.66e-06,2.53774)(1.67e-06,2.52486)(1.68e-06,2.51215)(1.69e-06,2.49962)(1.7e-06,2.48724)(1.71e-06,2.47503)(1.72e-06,2.46298)(1.73e-06,2.45108)(1.74e-06,2.43934)(1.75e-06,2.42774)(1.76e-06,2.41629)(1.77e-06,2.40497)(1.78e-06,2.3938)(1.79e-06,2.38277)(1.8e-06,2.37186)(1.81e-06,2.36109)(1.82e-06,2.35045)(1.83e-06,2.33993)(1.84e-06,2.32954)(1.85e-06,2.31927)(1.86e-06,2.30911)(1.87e-06,2.29908)(1.88e-06,2.28915)(1.89e-06,2.27934)(1.9e-06,2.26964)(1.91e-06,2.26005)(1.92e-06,2.25057)(1.93e-06,2.24119)(1.94e-06,2.23191)(1.95e-06,2.22273)(1.96e-06,2.21365)(1.97e-06,2.20467)(1.98e-06,2.19578)(1.99e-06,2.18699)(2e-06,2.17829)
    };
    \addlegendentry{H2}
    
    \addplot[
        color=purple,
        mark=square,
        mark size=0.5pt
    ]
    coordinates {
    (0,0)(1e-08,0.00075173)(2e-08,0.00152024)(3e-08,0.00235042)(4e-08,0.00327374)(5e-08,0.00431466)(6e-08,0.00549471)(7e-08,0.00683494)(8e-08,0.00835748)(9e-08,0.0100866)(1e-07,0.0120495)(1.1e-07,0.0142767)(1.2e-07,0.016803)(1.3e-07,0.0196677)(1.4e-07,0.0229152)(1.5e-07,0.0265958)(1.6e-07,0.0307663)(1.7e-07,0.0354908)(1.8e-07,0.0408415)(1.9e-07,0.0468996)(2e-07,0.0537567)(2.1e-07,0.0615153)(2.2e-07,0.0702905)(2.3e-07,0.0802111)(2.4e-07,0.091421)(2.5e-07,0.10408)(2.6e-07,0.118368)(2.7e-07,0.134481)(2.8e-07,0.152638)(2.9e-07,0.17308)(3e-07,0.19607)(3.1e-07,0.221895)(3.2e-07,0.250868)(3.3e-07,0.283325)(3.4e-07,0.319625)(3.5e-07,0.360148)(3.6e-07,0.405293)(3.7e-07,0.455473)(3.8e-07,0.511109)(3.9e-07,0.572621)(4e-07,0.640419)(4.1e-07,0.714891)(4.2e-07,0.796389)(4.3e-07,0.885213)(4.4e-07,0.98159)(4.5e-07,1.08566)(4.6e-07,1.19745)(4.7e-07,1.31687)(4.8e-07,1.44369)(4.9e-07,1.5775)(5e-07,1.71775)(5.1e-07,1.86374)(5.2e-07,2.01459)(5.3e-07,2.16929)(5.4e-07,2.32672)(5.5e-07,2.48566)(5.6e-07,2.64485)(5.7e-07,2.803)(5.8e-07,2.95886)(5.9e-07,3.11123)(6e-07,3.25903)(6.1e-07,3.40126)(6.2e-07,3.53712)(6.3e-07,3.66594)(6.4e-07,3.78722)(6.5e-07,3.9006)(6.6e-07,4.00592)(6.7e-07,4.10313)(6.8e-07,4.19229)(6.9e-07,4.2736)(7e-07,4.34732)(7.1e-07,4.41379)(7.2e-07,4.47337)(7.3e-07,4.52648)(7.4e-07,4.57355)(7.5e-07,4.61501)(7.6e-07,4.65129)(7.7e-07,4.68282)(7.8e-07,4.70998)(7.9e-07,4.73317)(8e-07,4.75274)(8.1e-07,4.76903)(8.2e-07,4.78234)(8.3e-07,4.79296)(8.4e-07,4.80116)(8.5e-07,4.80717)(8.6e-07,4.81121)(8.7e-07,4.81348)(8.8e-07,4.81416)(8.9e-07,4.81341)(9e-07,4.81137)(9.1e-07,4.80819)(9.2e-07,4.80398)(9.3e-07,4.79885)(9.4e-07,4.79289)(9.5e-07,4.7862)(9.6e-07,4.77886)(9.7e-07,4.77094)(9.8e-07,4.7625)(9.9e-07,4.7536)(1e-06,4.7443)(1.01e-06,4.73465)(1.02e-06,4.72468)(1.03e-06,4.71444)(1.04e-06,4.70396)(1.05e-06,4.69328)(1.06e-06,4.68242)(1.07e-06,4.67141)(1.08e-06,4.66028)(1.09e-06,4.64905)(1.1e-06,4.63773)(1.11e-06,4.62635)(1.12e-06,4.61492)(1.13e-06,4.60346)(1.14e-06,4.59197)(1.15e-06,4.58048)(1.16e-06,4.56898)(1.17e-06,4.5575)(1.18e-06,4.54603)(1.19e-06,4.5346)(1.2e-06,4.5232)(1.21e-06,4.51184)(1.22e-06,4.50052)(1.23e-06,4.48926)(1.24e-06,4.47806)(1.25e-06,4.46691)(1.26e-06,4.45583)(1.27e-06,4.44481)(1.28e-06,4.43387)(1.29e-06,4.423)(1.3e-06,4.4122)(1.31e-06,4.40147)(1.32e-06,4.39083)(1.33e-06,4.38026)(1.34e-06,4.36977)(1.35e-06,4.35936)(1.36e-06,4.34904)(1.37e-06,4.33879)(1.38e-06,4.32863)(1.39e-06,4.31855)(1.4e-06,4.30855)(1.41e-06,4.29864)(1.42e-06,4.28881)(1.43e-06,4.27906)(1.44e-06,4.26939)(1.45e-06,4.2598)(1.46e-06,4.2503)(1.47e-06,4.24087)(1.48e-06,4.23153)(1.49e-06,4.22227)(1.5e-06,4.21308)(1.51e-06,4.20398)(1.52e-06,4.19495)(1.53e-06,4.186)(1.54e-06,4.17712)(1.55e-06,4.16833)(1.56e-06,4.1596)(1.57e-06,4.15096)(1.58e-06,4.14238)(1.59e-06,4.13388)(1.6e-06,4.12545)(1.61e-06,4.1171)(1.62e-06,4.10881)(1.63e-06,4.10059)(1.64e-06,4.09244)(1.65e-06,4.08436)(1.66e-06,4.07635)(1.67e-06,4.0684)(1.68e-06,4.06053)(1.69e-06,4.05271)(1.7e-06,4.04496)(1.71e-06,4.03727)(1.72e-06,4.02965)(1.73e-06,4.02209)(1.74e-06,4.01458)(1.75e-06,4.00714)(1.76e-06,3.99976)(1.77e-06,3.99244)(1.78e-06,3.98518)(1.79e-06,3.97797)(1.8e-06,3.97082)(1.81e-06,3.96373)(1.82e-06,3.95669)(1.83e-06,3.94971)(1.84e-06,3.94278)(1.85e-06,3.9359)(1.86e-06,3.92908)(1.87e-06,3.92231)(1.88e-06,3.91559)(1.89e-06,3.90892)(1.9e-06,3.9023)(1.91e-06,3.89573)(1.92e-06,3.88922)(1.93e-06,3.88274)(1.94e-06,3.87632)(1.95e-06,3.86994)(1.96e-06,3.86361)(1.97e-06,3.85733)(1.98e-06,3.85109)(1.99e-06,3.8449)(2e-06,3.83875)
    };
    \addlegendentry{OH}
    
    \addplot[
        color=red,
        mark=square,
        mark size=0.5pt
    ]
    coordinates {
    (0,0)(1e-08,0.000118829)(2e-08,0.000475885)(3e-08,0.00108398)(4e-08,0.00196772)(5e-08,0.0031604)(6e-08,0.00470243)(7e-08,0.00664085)(8e-08,0.00902944)(9e-08,0.0119292)(1e-07,0.015409)(1.1e-07,0.019547)(1.2e-07,0.0244311)(1.3e-07,0.0301608)(1.4e-07,0.0368487)(1.5e-07,0.0446216)(1.6e-07,0.0536228)(1.7e-07,0.0640143)(1.8e-07,0.0759783)(1.9e-07,0.0897205)(2e-07,0.105472)(2.1e-07,0.123494)(2.2e-07,0.144077)(2.3e-07,0.167551)(2.4e-07,0.194282)(2.5e-07,0.224681)(2.6e-07,0.259205)(2.7e-07,0.298364)(2.8e-07,0.342725)(2.9e-07,0.392913)(3e-07,0.449617)(3.1e-07,0.513598)(3.2e-07,0.585683)(3.3e-07,0.666774)(3.4e-07,0.757847)(3.5e-07,0.859947)(3.6e-07,0.974191)(3.7e-07,1.10176)(3.8e-07,1.24387)(3.9e-07,1.4018)(4e-07,1.57683)(4.1e-07,1.77023)(4.2e-07,1.98327)(4.3e-07,2.21712)(4.4e-07,2.47284)(4.5e-07,2.75136)(4.6e-07,3.0534)(4.7e-07,3.37945)(4.8e-07,3.72969)(4.9e-07,4.104)(5e-07,4.50188)(5.1e-07,4.92244)(5.2e-07,5.36441)(5.3e-07,5.82613)(5.4e-07,6.30557)(5.5e-07,6.80039)(5.6e-07,7.30794)(5.7e-07,7.8254)(5.8e-07,8.34979)(5.9e-07,8.87809)(6e-07,9.40728)(6.1e-07,9.93445)(6.2e-07,10.4569)(6.3e-07,10.972)(6.4e-07,11.4774)(6.5e-07,11.9713)(6.6e-07,12.452)(6.7e-07,12.9179)(6.8e-07,13.368)(6.9e-07,13.8016)(7e-07,14.2181)(7.1e-07,14.6171)(7.2e-07,14.9987)(7.3e-07,15.3628)(7.4e-07,15.7098)(7.5e-07,16.04)(7.6e-07,16.3539)(7.7e-07,16.6521)(7.8e-07,16.9352)(7.9e-07,17.2038)(8e-07,17.4587)(8.1e-07,17.7004)(8.2e-07,17.9298)(8.3e-07,18.1474)(8.4e-07,18.354)(8.5e-07,18.5502)(8.6e-07,18.7367)(8.7e-07,18.914)(8.8e-07,19.0827)(8.9e-07,19.2433)(9e-07,19.3964)(9.1e-07,19.5425)(9.2e-07,19.6819)(9.3e-07,19.8152)(9.4e-07,19.9426)(9.5e-07,20.0646)(9.6e-07,20.1816)(9.7e-07,20.2938)(9.8e-07,20.4015)(9.9e-07,20.5051)(1e-06,20.6047)(1.01e-06,20.7007)(1.02e-06,20.7932)(1.03e-06,20.8824)(1.04e-06,20.9686)(1.05e-06,21.0518)(1.06e-06,21.1324)(1.07e-06,21.2104)(1.08e-06,21.2859)(1.09e-06,21.3592)(1.1e-06,21.4303)(1.11e-06,21.4993)(1.12e-06,21.5664)(1.13e-06,21.6316)(1.14e-06,21.6951)(1.15e-06,21.7568)(1.16e-06,21.817)(1.17e-06,21.8757)(1.18e-06,21.9329)(1.19e-06,21.9887)(1.2e-06,22.0432)(1.21e-06,22.0964)(1.22e-06,22.1484)(1.23e-06,22.1993)(1.24e-06,22.249)(1.25e-06,22.2977)(1.26e-06,22.3453)(1.27e-06,22.3919)(1.28e-06,22.4376)(1.29e-06,22.4824)(1.3e-06,22.5263)(1.31e-06,22.5693)(1.32e-06,22.6115)(1.33e-06,22.6529)(1.34e-06,22.6936)(1.35e-06,22.7335)(1.36e-06,22.7726)(1.37e-06,22.8111)(1.38e-06,22.8489)(1.39e-06,22.8861)(1.4e-06,22.9226)(1.41e-06,22.9585)(1.42e-06,22.9938)(1.43e-06,23.0286)(1.44e-06,23.0627)(1.45e-06,23.0963)(1.46e-06,23.1294)(1.47e-06,23.162)(1.48e-06,23.1941)(1.49e-06,23.2256)(1.5e-06,23.2567)(1.51e-06,23.2874)(1.52e-06,23.3175)(1.53e-06,23.3473)(1.54e-06,23.3766)(1.55e-06,23.4054)(1.56e-06,23.4339)(1.57e-06,23.462)(1.58e-06,23.4896)(1.59e-06,23.5169)(1.6e-06,23.5438)(1.61e-06,23.5704)(1.62e-06,23.5966)(1.63e-06,23.6224)(1.64e-06,23.6479)(1.65e-06,23.6731)(1.66e-06,23.6979)(1.67e-06,23.7224)(1.68e-06,23.7466)(1.69e-06,23.7705)(1.7e-06,23.7941)(1.71e-06,23.8174)(1.72e-06,23.8404)(1.73e-06,23.8632)(1.74e-06,23.8856)(1.75e-06,23.9078)(1.76e-06,23.9297)(1.77e-06,23.9514)(1.78e-06,23.9728)(1.79e-06,23.9939)(1.8e-06,24.0148)(1.81e-06,24.0354)(1.82e-06,24.0559)(1.83e-06,24.076)(1.84e-06,24.096)(1.85e-06,24.1157)(1.86e-06,24.1352)(1.87e-06,24.1545)(1.88e-06,24.1736)(1.89e-06,24.1925)(1.9e-06,24.2111)(1.91e-06,24.2296)(1.92e-06,24.2479)(1.93e-06,24.2659)(1.94e-06,24.2838)(1.95e-06,24.3015)(1.96e-06,24.319)(1.97e-06,24.3363)(1.98e-06,24.3535)(1.99e-06,24.3704)(2e-06,24.3872)
    };
    \addlegendentry{H2O}
    
    \addplot[
        color=black,
        mark=square,
        mark size=0.5pt
    ]
    coordinates {
    (0,29.3991)(1e-08,29.3987)(2e-08,29.398)(3e-08,29.3971)(4e-08,29.396)(5e-08,29.3946)(6e-08,29.3929)(7e-08,29.3908)(8e-08,29.3885)(9e-08,29.3857)(1e-07,29.3824)(1.1e-07,29.3786)(1.2e-07,29.3741)(1.3e-07,29.369)(1.4e-07,29.3631)(1.5e-07,29.3563)(1.6e-07,29.3486)(1.7e-07,29.3396)(1.8e-07,29.3295)(1.9e-07,29.3178)(2e-07,29.3045)(2.1e-07,29.2894)(2.2e-07,29.2722)(2.3e-07,29.2527)(2.4e-07,29.2305)(2.5e-07,29.2053)(2.6e-07,29.1767)(2.7e-07,29.1444)(2.8e-07,29.1079)(2.9e-07,29.0666)(3e-07,29.02)(3.1e-07,28.9675)(3.2e-07,28.9085)(3.3e-07,28.8421)(3.4e-07,28.7676)(3.5e-07,28.6842)(3.6e-07,28.5909)(3.7e-07,28.4868)(3.8e-07,28.371)(3.9e-07,28.2423)(4e-07,28.0998)(4.1e-07,27.9425)(4.2e-07,27.7693)(4.3e-07,27.5794)(4.4e-07,27.3719)(4.5e-07,27.146)(4.6e-07,26.9014)(4.7e-07,26.6377)(4.8e-07,26.3547)(4.9e-07,26.0527)(5e-07,25.7322)(5.1e-07,25.394)(5.2e-07,25.0393)(5.3e-07,24.6697)(5.4e-07,24.2868)(5.5e-07,23.8928)(5.6e-07,23.49)(5.7e-07,23.0808)(5.8e-07,22.6677)(5.9e-07,22.2535)(6e-07,21.8407)(6.1e-07,21.4317)(6.2e-07,21.0288)(6.3e-07,20.6343)(6.4e-07,20.2499)(6.5e-07,19.8772)(6.6e-07,19.5177)(6.7e-07,19.1724)(6.8e-07,18.8422)(6.9e-07,18.5274)(7e-07,18.2286)(7.1e-07,17.9458)(7.2e-07,17.679)(7.3e-07,17.4278)(7.4e-07,17.192)(7.5e-07,16.9712)(7.6e-07,16.7647)(7.7e-07,16.572)(7.8e-07,16.3924)(7.9e-07,16.2253)(8e-07,16.07)(8.1e-07,15.9258)(8.2e-07,15.792)(8.3e-07,15.6679)(8.4e-07,15.553)(8.5e-07,15.4466)(8.6e-07,15.3481)(8.7e-07,15.2569)(8.8e-07,15.1726)(8.9e-07,15.0945)(9e-07,15.0223)(9.1e-07,14.9555)(9.2e-07,14.8937)(9.3e-07,14.8365)(9.4e-07,14.7836)(9.5e-07,14.7345)(9.6e-07,14.6891)(9.7e-07,14.647)(9.8e-07,14.6079)(9.9e-07,14.5717)(1e-06,14.5381)(1.01e-06,14.5068)(1.02e-06,14.4778)(1.03e-06,14.4508)(1.04e-06,14.4257)(1.05e-06,14.4023)(1.06e-06,14.3805)(1.07e-06,14.3602)(1.08e-06,14.3412)(1.09e-06,14.3235)(1.1e-06,14.3069)(1.11e-06,14.2914)(1.12e-06,14.2769)(1.13e-06,14.2633)(1.14e-06,14.2505)(1.15e-06,14.2385)(1.16e-06,14.2273)(1.17e-06,14.2167)(1.18e-06,14.2067)(1.19e-06,14.1973)(1.2e-06,14.1885)(1.21e-06,14.1801)(1.22e-06,14.1722)(1.23e-06,14.1647)(1.24e-06,14.1577)(1.25e-06,14.151)(1.26e-06,14.1446)(1.27e-06,14.1386)(1.28e-06,14.1329)(1.29e-06,14.1274)(1.3e-06,14.1223)(1.31e-06,14.1174)(1.32e-06,14.1127)(1.33e-06,14.1082)(1.34e-06,14.1039)(1.35e-06,14.0999)(1.36e-06,14.096)(1.37e-06,14.0923)(1.38e-06,14.0887)(1.39e-06,14.0853)(1.4e-06,14.082)(1.41e-06,14.0789)(1.42e-06,14.0759)(1.43e-06,14.073)(1.44e-06,14.0703)(1.45e-06,14.0676)(1.46e-06,14.0651)(1.47e-06,14.0626)(1.48e-06,14.0603)(1.49e-06,14.058)(1.5e-06,14.0558)(1.51e-06,14.0537)(1.52e-06,14.0517)(1.53e-06,14.0497)(1.54e-06,14.0478)(1.55e-06,14.046)(1.56e-06,14.0442)(1.57e-06,14.0425)(1.58e-06,14.0409)(1.59e-06,14.0393)(1.6e-06,14.0378)(1.61e-06,14.0363)(1.62e-06,14.0348)(1.63e-06,14.0335)(1.64e-06,14.0321)(1.65e-06,14.0308)(1.66e-06,14.0296)(1.67e-06,14.0283)(1.68e-06,14.0272)(1.69e-06,14.026)(1.7e-06,14.0249)(1.71e-06,14.0239)(1.72e-06,14.0228)(1.73e-06,14.0218)(1.74e-06,14.0208)(1.75e-06,14.0199)(1.76e-06,14.019)(1.77e-06,14.0181)(1.78e-06,14.0173)(1.79e-06,14.0164)(1.8e-06,14.0156)(1.81e-06,14.0148)(1.82e-06,14.0141)(1.83e-06,14.0134)(1.84e-06,14.0127)(1.85e-06,14.012)(1.86e-06,14.0113)(1.87e-06,14.0107)(1.88e-06,14.01)(1.89e-06,14.0094)(1.9e-06,14.0089)(1.91e-06,14.0083)(1.92e-06,14.0077)(1.93e-06,14.0072)(1.94e-06,14.0067)(1.95e-06,14.0062)(1.96e-06,14.0057)(1.97e-06,14.0053)(1.98e-06,14.0048)(1.99e-06,14.0044)(2e-06,14.004)
    };
    \addlegendentry{O2}
    
    \addplot[
        color=magenta,
        mark=square,
        mark size=0.5pt
    ]
    coordinates {
    (0,0)(1e-08,0.00147222)(2e-08,0.00303428)(3e-08,0.00472728)(4e-08,0.00659161)(5e-08,0.00866733)(6e-08,0.0109951)(7e-08,0.0136173)(8e-08,0.0165793)(9e-08,0.0199304)(1e-07,0.0237249)(1.1e-07,0.0280235)(1.2e-07,0.0328941)(1.3e-07,0.0384128)(1.4e-07,0.0446654)(1.5e-07,0.0517484)(1.6e-07,0.0597705)(1.7e-07,0.068854)(1.8e-07,0.0791362)(1.9e-07,0.0907716)(2e-07,0.103933)(2.1e-07,0.118814)(2.2e-07,0.135632)(2.3e-07,0.154628)(2.4e-07,0.176069)(2.5e-07,0.200255)(2.6e-07,0.227514)(2.7e-07,0.25821)(2.8e-07,0.292739)(2.9e-07,0.331536)(3e-07,0.375073)(3.1e-07,0.423856)(3.2e-07,0.478429)(3.3e-07,0.539367)(3.4e-07,0.607272)(3.5e-07,0.682766)(3.6e-07,0.766483)(3.7e-07,0.859053)(3.8e-07,0.961087)(3.9e-07,1.07316)(4e-07,1.19577)(4.1e-07,1.32934)(4.2e-07,1.47417)(4.3e-07,1.63039)(4.4e-07,1.79795)(4.5e-07,1.97656)(4.6e-07,2.16571)(4.7e-07,2.36458)(4.8e-07,2.57207)(4.9e-07,2.78678)(5e-07,3.00703)(5.1e-07,3.23087)(5.2e-07,3.45615)(5.3e-07,3.68056)(5.4e-07,3.90169)(5.5e-07,4.11713)(5.6e-07,4.32457)(5.7e-07,4.52182)(5.8e-07,4.70694)(5.9e-07,4.87826)(6e-07,5.03447)(6.1e-07,5.17459)(6.2e-07,5.29802)(6.3e-07,5.40451)(6.4e-07,5.49413)(6.5e-07,5.56725)(6.6e-07,5.62451)(6.7e-07,5.6667)(6.8e-07,5.69482)(6.9e-07,5.70992)(7e-07,5.71317)(7.1e-07,5.70571)(7.2e-07,5.68871)(7.3e-07,5.6633)(7.4e-07,5.63055)(7.5e-07,5.59148)(7.6e-07,5.54702)(7.7e-07,5.49803)(7.8e-07,5.4453)(7.9e-07,5.38953)(8e-07,5.33133)(8.1e-07,5.27126)(8.2e-07,5.20979)(8.3e-07,5.14735)(8.4e-07,5.08431)(8.5e-07,5.02096)(8.6e-07,4.95757)(8.7e-07,4.89437)(8.8e-07,4.83155)(8.9e-07,4.76925)(9e-07,4.70762)(9.1e-07,4.64674)(9.2e-07,4.58671)(9.3e-07,4.5276)(9.4e-07,4.46945)(9.5e-07,4.4123)(9.6e-07,4.35618)(9.7e-07,4.30111)(9.8e-07,4.2471)(9.9e-07,4.19416)(1e-06,4.14227)(1.01e-06,4.09144)(1.02e-06,4.04166)(1.03e-06,3.99291)(1.04e-06,3.94517)(1.05e-06,3.89844)(1.06e-06,3.8527)(1.07e-06,3.80792)(1.08e-06,3.76409)(1.09e-06,3.72118)(1.1e-06,3.67917)(1.11e-06,3.63804)(1.12e-06,3.59778)(1.13e-06,3.55836)(1.14e-06,3.51975)(1.15e-06,3.48194)(1.16e-06,3.44491)(1.17e-06,3.40863)(1.18e-06,3.3731)(1.19e-06,3.33828)(1.2e-06,3.30416)(1.21e-06,3.27071)(1.22e-06,3.23793)(1.23e-06,3.2058)(1.24e-06,3.17429)(1.25e-06,3.14339)(1.26e-06,3.11309)(1.27e-06,3.08337)(1.28e-06,3.0542)(1.29e-06,3.02559)(1.3e-06,2.99751)(1.31e-06,2.96996)(1.32e-06,2.9429)(1.33e-06,2.91635)(1.34e-06,2.89027)(1.35e-06,2.86466)(1.36e-06,2.83951)(1.37e-06,2.8148)(1.38e-06,2.79053)(1.39e-06,2.76668)(1.4e-06,2.74324)(1.41e-06,2.72021)(1.42e-06,2.69756)(1.43e-06,2.6753)(1.44e-06,2.65341)(1.45e-06,2.63189)(1.46e-06,2.61072)(1.47e-06,2.5899)(1.48e-06,2.56942)(1.49e-06,2.54927)(1.5e-06,2.52944)(1.51e-06,2.50993)(1.52e-06,2.49073)(1.53e-06,2.47182)(1.54e-06,2.45321)(1.55e-06,2.43489)(1.56e-06,2.41685)(1.57e-06,2.39908)(1.58e-06,2.38159)(1.59e-06,2.36435)(1.6e-06,2.34737)(1.61e-06,2.33065)(1.62e-06,2.31417)(1.63e-06,2.29793)(1.64e-06,2.28192)(1.65e-06,2.26615)(1.66e-06,2.2506)(1.67e-06,2.23527)(1.68e-06,2.22016)(1.69e-06,2.20526)(1.7e-06,2.19056)(1.71e-06,2.17607)(1.72e-06,2.16178)(1.73e-06,2.14769)(1.74e-06,2.13378)(1.75e-06,2.12006)(1.76e-06,2.10653)(1.77e-06,2.09317)(1.78e-06,2.07999)(1.79e-06,2.06699)(1.8e-06,2.05415)(1.81e-06,2.04148)(1.82e-06,2.02897)(1.83e-06,2.01662)(1.84e-06,2.00443)(1.85e-06,1.99239)(1.86e-06,1.98051)(1.87e-06,1.96877)(1.88e-06,1.95718)(1.89e-06,1.94573)(1.9e-06,1.93442)(1.91e-06,1.92324)(1.92e-06,1.91221)(1.93e-06,1.9013)(1.94e-06,1.89053)(1.95e-06,1.87988)(1.96e-06,1.86936)(1.97e-06,1.85896)(1.98e-06,1.84869)(1.99e-06,1.83853)(2e-06,1.82849)
    };
    \addlegendentry{H}
    
    \addplot[
        color=green,
        mark=square,
        mark size=0.5pt
    ]
    coordinates {
    (0,0)(1e-08,8.76304e-05)(2e-08,0.000323537)(3e-08,0.000679369)(4e-08,0.00113927)(5e-08,0.00169636)(6e-08,0.00235025)(7e-08,0.00310541)(8e-08,0.00397008)(9e-08,0.00495564)(1e-07,0.00607627)(1.1e-07,0.00734882)(1.2e-07,0.00879287)(1.3e-07,0.0104309)(1.4e-07,0.0122885)(1.5e-07,0.0143947)(1.6e-07,0.0167825)(1.7e-07,0.0194891)(1.8e-07,0.0225565)(1.9e-07,0.0260322)(2e-07,0.0299698)(2.1e-07,0.0344297)(2.2e-07,0.0394798)(2.3e-07,0.0451965)(2.4e-07,0.0516658)(2.5e-07,0.0589839)(2.6e-07,0.0672586)(2.7e-07,0.0766105)(2.8e-07,0.0871741)(2.9e-07,0.0990991)(3e-07,0.112552)(3.1e-07,0.127715)(3.2e-07,0.144793)(3.3e-07,0.164007)(3.4e-07,0.185602)(3.5e-07,0.20984)(3.6e-07,0.23701)(3.7e-07,0.267417)(3.8e-07,0.301389)(3.9e-07,0.339271)(4e-07,0.381421)(4.1e-07,0.428211)(4.2e-07,0.480015)(4.3e-07,0.537206)(4.4e-07,0.600145)(4.5e-07,0.669171)(4.6e-07,0.744589)(4.7e-07,0.826658)(4.8e-07,0.915573)(4.9e-07,1.01145)(5e-07,1.11433)(5.1e-07,1.22411)(5.2e-07,1.34062)(5.3e-07,1.46353)(5.4e-07,1.5924)(5.5e-07,1.72664)(5.6e-07,1.86557)(5.7e-07,2.00837)(5.8e-07,2.15413)(5.9e-07,2.30188)(6e-07,2.45058)(6.1e-07,2.59917)(6.2e-07,2.7466)(6.3e-07,2.89183)(6.4e-07,3.03388)(6.5e-07,3.17186)(6.6e-07,3.30493)(6.7e-07,3.43239)(6.8e-07,3.55364)(6.9e-07,3.66817)(7e-07,3.77562)(7.1e-07,3.87573)(7.2e-07,3.96832)(7.3e-07,4.05335)(7.4e-07,4.13084)(7.5e-07,4.20088)(7.6e-07,4.26364)(7.7e-07,4.31935)(7.8e-07,4.36826)(7.9e-07,4.41068)(8e-07,4.44692)(8.1e-07,4.47733)(8.2e-07,4.50226)(8.3e-07,4.52207)(8.4e-07,4.53709)(8.5e-07,4.5477)(8.6e-07,4.55422)(8.7e-07,4.55698)(8.8e-07,4.5563)(8.9e-07,4.55248)(9e-07,4.5458)(9.1e-07,4.53654)(9.2e-07,4.52494)(9.3e-07,4.51125)(9.4e-07,4.49567)(9.5e-07,4.47842)(9.6e-07,4.45968)(9.7e-07,4.43962)(9.8e-07,4.41842)(9.9e-07,4.3962)(1e-06,4.37312)(1.01e-06,4.34928)(1.02e-06,4.3248)(1.03e-06,4.29979)(1.04e-06,4.27433)(1.05e-06,4.24852)(1.06e-06,4.22241)(1.07e-06,4.19609)(1.08e-06,4.16962)(1.09e-06,4.14304)(1.1e-06,4.11641)(1.11e-06,4.08978)(1.12e-06,4.06318)(1.13e-06,4.03664)(1.14e-06,4.0102)(1.15e-06,3.98389)(1.16e-06,3.95773)(1.17e-06,3.93174)(1.18e-06,3.90594)(1.19e-06,3.88034)(1.2e-06,3.85497)(1.21e-06,3.82983)(1.22e-06,3.80493)(1.23e-06,3.78029)(1.24e-06,3.7559)(1.25e-06,3.73178)(1.26e-06,3.70792)(1.27e-06,3.68434)(1.28e-06,3.66104)(1.29e-06,3.63801)(1.3e-06,3.61527)(1.31e-06,3.5928)(1.32e-06,3.57061)(1.33e-06,3.5487)(1.34e-06,3.52707)(1.35e-06,3.50572)(1.36e-06,3.48464)(1.37e-06,3.46383)(1.38e-06,3.4433)(1.39e-06,3.42303)(1.4e-06,3.40303)(1.41e-06,3.38329)(1.42e-06,3.36381)(1.43e-06,3.34458)(1.44e-06,3.32561)(1.45e-06,3.30688)(1.46e-06,3.2884)(1.47e-06,3.27017)(1.48e-06,3.25217)(1.49e-06,3.23441)(1.5e-06,3.21687)(1.51e-06,3.19957)(1.52e-06,3.18249)(1.53e-06,3.16563)(1.54e-06,3.14898)(1.55e-06,3.13255)(1.56e-06,3.11633)(1.57e-06,3.10031)(1.58e-06,3.0845)(1.59e-06,3.06889)(1.6e-06,3.05347)(1.61e-06,3.03824)(1.62e-06,3.0232)(1.63e-06,3.00835)(1.64e-06,2.99369)(1.65e-06,2.9792)(1.66e-06,2.96488)(1.67e-06,2.95075)(1.68e-06,2.93678)(1.69e-06,2.92297)(1.7e-06,2.90934)(1.71e-06,2.89586)(1.72e-06,2.88255)(1.73e-06,2.86938)(1.74e-06,2.85638)(1.75e-06,2.84352)(1.76e-06,2.83081)(1.77e-06,2.81825)(1.78e-06,2.80583)(1.79e-06,2.79355)(1.8e-06,2.78141)(1.81e-06,2.76941)(1.82e-06,2.75754)(1.83e-06,2.7458)(1.84e-06,2.73419)(1.85e-06,2.7227)(1.86e-06,2.71135)(1.87e-06,2.70011)(1.88e-06,2.689)(1.89e-06,2.678)(1.9e-06,2.66713)(1.91e-06,2.65636)(1.92e-06,2.64571)(1.93e-06,2.63517)(1.94e-06,2.62475)(1.95e-06,2.61442)(1.96e-06,2.60421)(1.97e-06,2.5941)(1.98e-06,2.58409)(1.99e-06,2.57419)(2e-06,2.56438)
    };
    \addlegendentry{O}
    
    \end{axis}
    \end{tikzpicture}
    \caption{Решение задачи №5}
    \label{fig:chem2}
\end{figure}

Количество ДУ в данном случае не меняется и зависит только от количества участвующих в реакциях веществ.

\begin{table}    
    \caption{Таблица сравнения методов}
    \begin{tabularx}{\textwidth}{|X|c|c|c|}
    \hline
    Метод & Время работы (мс) & Количество шагов & Точность\\
    \hline
    Неявный Гаусс 6-го порядка & 331 & 50 & 0.001\\
    \hline
    Неявный Гаусс 4-го порядка & 145 & 35 & 0.001\\
    \hline
    Неявный Гаусс 2-го порядка & 201 & 100 & 0.001\\
    \hline
    Явный Рунге-Кутта 6-го порядка & 64 & 100 & 0.001\\
    \hline
    Явный Рунге-Кутта 4-го порядка & 41 & 100 & 0.001\\
    \hline
    Вложенный Дормана-Принс 4(5)-го порядка & 245 & 200 & 0.001\\
    \hline
    Вложенный Фалберг 2(3)-го порядка & 102 & 200 & 0.001\\
    \hline
    \end{tabularx}
    \label{tab:Methods2}
\end{table}

Стоит отметить, что добавка $M$ существенно влияет на химическое взаимодействие и распределение концентраций химических веществ,
поэтому приходится
отдельно обрабатывать случай наличия добавки в реакции и время работы алгоритма заметно падает.
В таблице \ref{tab:Methods2} приведено сравнение различных методов по времени и количеству шагов.

Оба этих примера показывают, что задачи химической кинетики в некоторых случаях можно решать и при помощи использования
явных методов, но при этом может кратно увеличиться время работы и количество шагов.

\subsection{Моделирование взрыва в камере постоянного объёма}

Оба этих примера расчитывались при постоянной температуре и меняющейся плотности.
Рассмотрим замкнутую систему неизменного объёма, заполненную смесью газов. Предположим, что
в данной системе протекают 7 реакций вида (\ref{fig:chem2}).
В задачах моделирования взрыва температура меняется
вместе с концентрациями по закону (\ref{eq:TempFind}). На рисунке \ref{fig:bomb} показано моделирование взрыва "бомбы" 
с начальными концентрациями
$\mu_{H_2} = 0.5$ и $\mu_{O_2} = 0.5$, нормальным атмосферным давлением $P = 101325\text{Па}$ и температурой $T = 2300\text{К}$. 

%график
\begin{figure}
    \begin{subfigure}[t]{0.45\linewidth}
        \centering
        \begin{tikzpicture}[scale = 0.8]
    \begin{axis}[
        enlargelimits=true,
        xlabel={$t, \text{с}$},
	    ylabel={$\gamma, \text{моль}/\text{кг}$},
        legend style={at={(1,1.75)},
            anchor=north east},
        ymajorgrids=true,
        grid style=dashed,
    ]
    
    \addplot[
        color=blue,
        mark=square,
        mark size=0.5pt
    ]
    coordinates {
    (0,29.3991)(1e-08,29.3986)(2e-08,29.398)(3e-08,29.3971)(4e-08,29.3962)(5e-08,29.395)(6e-08,29.3937)(7e-08,29.3923)(8e-08,29.3906)(9e-08,29.3887)(1e-07,29.3866)(1.1e-07,29.3841)(1.2e-07,29.3814)(1.3e-07,29.3782)(1.4e-07,29.3747)(1.5e-07,29.3707)(1.6e-07,29.3661)(1.7e-07,29.3609)(1.8e-07,29.355)(1.9e-07,29.3484)(2e-07,29.3408)(2.1e-07,29.3323)(2.2e-07,29.3226)(2.3e-07,29.3116)(2.4e-07,29.2991)(2.5e-07,29.2849)(2.6e-07,29.2689)(2.7e-07,29.2508)(2.8e-07,29.2302)(2.9e-07,29.2069)(3e-07,29.1805)(3.1e-07,29.1506)(3.2e-07,29.1168)(3.3e-07,29.0785)(3.4e-07,29.0352)(3.5e-07,28.9863)(3.6e-07,28.931)(3.7e-07,28.8685)(3.8e-07,28.798)(3.9e-07,28.7184)(4e-07,28.6287)(4.1e-07,28.5277)(4.2e-07,28.4139)(4.3e-07,28.286)(4.4e-07,28.1424)(4.5e-07,27.9812)(4.6e-07,27.8006)(4.7e-07,27.5986)(4.8e-07,27.3729)(4.9e-07,27.1213)(5e-07,26.8412)(5.1e-07,26.53)(5.2e-07,26.185)(5.3e-07,25.8034)(5.4e-07,25.3823)(5.5e-07,24.9187)(5.6e-07,24.4096)(5.7e-07,23.8521)(5.8e-07,23.243)(5.9e-07,22.5795)(6e-07,21.8588)(6.1e-07,21.0782)(6.2e-07,20.2353)(6.3e-07,19.3281)(6.4e-07,18.3551)(6.5e-07,17.3159)(6.6e-07,16.2112)(6.7e-07,15.0437)(6.8e-07,13.8185)(6.9e-07,12.5439)(7e-07,11.2326)(7.1e-07,9.90183)(7.2e-07,8.57377)(7.3e-07,7.27524)(7.4e-07,6.03612)(7.5e-07,4.88678)(7.6e-07,3.85453)(7.7e-07,2.95987)(7.8e-07,2.21345)(7.9e-07,1.6147)(8e-07,1.15276)(8.1e-07,0.809271)(8.2e-07,0.562161)(8.3e-07,0.389213)(8.4e-07,0.270634)(8.5e-07,0.19035)(8.6e-07,0.136209)(8.7e-07,0.0995373)(8.8e-07,0.0743962)(8.9e-07,0.0568473)(9e-07,0.0443278)(9.1e-07,0.0351855)(9.2e-07,0.0283545)(9.3e-07,0.0231415)(9.4e-07,0.0190883)(9.5e-07,0.0158856)(9.6e-07,0.0133202)(9.7e-07,0.0112414)(9.8e-07,0.00954014)(9.9e-07,0.00813613)(1e-06,0.00696888)(1.01e-06,0.00599221)(1.02e-06,0.00517033)(1.03e-06,0.0044752)(1.04e-06,0.00388456)(1.05e-06,0.00338063)(1.06e-06,0.00294906)(1.07e-06,0.00257818)(1.08e-06,0.00225845)(1.09e-06,0.00198202)(1.1e-06,0.00174239)(1.11e-06,0.00153414)(1.12e-06,0.00135275)(1.13e-06,0.00119442)(1.14e-06,0.00105595)(1.15e-06,0.000934623)(1.16e-06,0.00082814)(1.17e-06,0.000734536)(1.18e-06,0.000652131)(1.19e-06,0.000579484)(1.2e-06,0.000515358)(1.21e-06,0.000458684)(1.22e-06,0.00040854)(1.23e-06,0.000364125)(1.24e-06,0.000324748)(1.25e-06,0.000289802)(1.26e-06,0.000258763)(1.27e-06,0.000231171)(1.28e-06,0.000206624)(1.29e-06,0.00018477)(1.3e-06,0.000165301)(1.31e-06,0.000147944)(1.32e-06,0.000132462)(1.33e-06,0.000118643)(1.34e-06,0.000106303)(1.35e-06,9.52779e-05)(1.36e-06,8.54224e-05)(1.37e-06,7.66086e-05)(1.38e-06,6.87231e-05)(1.39e-06,6.16652e-05)(1.4e-06,5.53457e-05)(1.41e-06,4.96853e-05)(1.42e-06,4.46135e-05)(1.43e-06,4.00676e-05)(1.44e-06,3.5992e-05)(1.45e-06,3.23368e-05)(1.46e-06,2.90579e-05)(1.47e-06,2.61157e-05)(1.48e-06,2.3475e-05)(1.49e-06,2.11045e-05)(1.5e-06,1.89759e-05)(1.51e-06,1.70642e-05)(1.52e-06,1.53471e-05)(1.53e-06,1.38043e-05)(1.54e-06,1.24179e-05)(1.55e-06,1.1172e-05)(1.56e-06,1.00521e-05)(1.57e-06,9.04522e-06)(1.58e-06,8.13994e-06)(1.59e-06,7.32588e-06)(1.6e-06,6.59374e-06)(1.61e-06,5.9352e-06)(1.62e-06,5.34281e-06)(1.63e-06,4.80986e-06)(1.64e-06,4.33035e-06)(1.65e-06,3.89887e-06)(1.66e-06,3.51057e-06)(1.67e-06,3.16111e-06)(1.68e-06,2.84658e-06)(1.69e-06,2.56347e-06)(1.7e-06,2.30861e-06)(1.71e-06,2.07918e-06)(1.72e-06,1.87262e-06)(1.73e-06,1.68664e-06)(1.74e-06,1.51919e-06)(1.75e-06,1.36841e-06)(1.76e-06,1.23263e-06)(1.77e-06,1.11036e-06)(1.78e-06,1.00024e-06)(1.79e-06,9.01065e-07)(1.8e-06,8.11745e-07)(1.81e-06,7.31295e-07)(1.82e-06,6.58832e-07)(1.83e-06,5.93562e-07)(1.84e-06,5.34768e-07)(1.85e-06,4.81806e-07)(1.86e-06,4.34097e-07)(1.87e-06,3.91118e-07)(1.88e-06,3.524e-07)(1.89e-06,3.17519e-07)(1.9e-06,2.86094e-07)(1.91e-06,2.57782e-07)(1.92e-06,2.32275e-07)(1.93e-06,2.09294e-07)(1.94e-06,1.88588e-07)(1.95e-06,1.69932e-07)(1.96e-06,1.53124e-07)(1.97e-06,1.37979e-07)(1.98e-06,1.24332e-07)(1.99e-06,1.12037e-07)(2e-06,1.00957e-07)(2.01e-06,9.09744e-08)
    };
    \addlegendentry{Приближённое решение 1}
    
    \addplot[
        color=purple,
        mark=square,
        mark size=0.5pt
    ]
    coordinates {
    (0,0)(1e-08,0.000659709)(2e-08,0.00116714)(3e-08,0.00158621)(4e-08,0.0019613)(5e-08,0.00232339)(6e-08,0.00269465)(7e-08,0.00309154)(8e-08,0.00352711)(9e-08,0.00401252)(1e-07,0.0045581)(1.1e-07,0.00517413)(1.2e-07,0.00587135)(1.3e-07,0.00666139)(1.4e-07,0.00755714)(1.5e-07,0.00857297)(1.6e-07,0.00972508)(1.7e-07,0.0110318)(1.8e-07,0.0125137)(1.9e-07,0.0141944)(2e-07,0.0161002)(2.1e-07,0.0182612)(2.2e-07,0.0207113)(2.3e-07,0.023489)(2.4e-07,0.0266377)(2.5e-07,0.0302065)(2.6e-07,0.0342508)(2.7e-07,0.0388334)(2.8e-07,0.0440249)(2.9e-07,0.049905)(3e-07,0.0565636)(3.1e-07,0.0641018)(3.2e-07,0.0726333)(3.3e-07,0.0822858)(3.4e-07,0.0932027)(3.5e-07,0.105544)(3.6e-07,0.11949)(3.7e-07,0.135241)(3.8e-07,0.15302)(3.9e-07,0.173074)(4e-07,0.19568)(4.1e-07,0.221141)(4.2e-07,0.249792)(4.3e-07,0.282001)(4.4e-07,0.318171)(4.5e-07,0.358742)(4.6e-07,0.404188)(4.7e-07,0.455025)(4.8e-07,0.511805)(4.9e-07,0.57512)(5e-07,0.645599)(5.1e-07,0.723908)(5.2e-07,0.81075)(5.3e-07,0.906862)(5.4e-07,1.01301)(5.5e-07,1.13001)(5.6e-07,1.25868)(5.7e-07,1.39989)(5.8e-07,1.55456)(5.9e-07,1.7236)(6e-07,1.90798)(6.1e-07,2.10867)(6.2e-07,2.32664)(6.3e-07,2.56282)(6.4e-07,2.81802)(6.5e-07,3.09281)(6.6e-07,3.38738)(6.7e-07,3.7013)(6.8e-07,4.03318)(6.9e-07,4.38026)(7e-07,4.73796)(7.1e-07,5.09939)(7.2e-07,5.45494)(7.3e-07,5.79223)(7.4e-07,6.09649)(7.5e-07,6.35169)(7.6e-07,6.54245)(7.7e-07,6.65647)(7.8e-07,6.68698)(7.9e-07,6.63431)(8e-07,6.50608)(8.1e-07,6.31567)(8.2e-07,6.07963)(8.3e-07,5.8149)(8.4e-07,5.5365)(8.5e-07,5.25634)(8.6e-07,4.98286)(8.7e-07,4.72145)(8.8e-07,4.47508)(8.9e-07,4.24496)(9e-07,4.03121)(9.1e-07,3.83325)(9.2e-07,3.65014)(9.3e-07,3.48078)(9.4e-07,3.32403)(9.5e-07,3.17879)(9.6e-07,3.04401)(9.7e-07,2.91874)(9.8e-07,2.80209)(9.9e-07,2.69329)(1e-06,2.59161)(1.01e-06,2.49643)(1.02e-06,2.40717)(1.03e-06,2.32333)(1.04e-06,2.24445)(1.05e-06,2.17012)(1.06e-06,2.09996)(1.07e-06,2.03366)(1.08e-06,1.9709)(1.09e-06,1.91142)(1.1e-06,1.85498)(1.11e-06,1.80135)(1.12e-06,1.75034)(1.13e-06,1.70176)(1.14e-06,1.65544)(1.15e-06,1.61124)(1.16e-06,1.56901)(1.17e-06,1.52863)(1.18e-06,1.48999)(1.19e-06,1.45297)(1.2e-06,1.41748)(1.21e-06,1.38342)(1.22e-06,1.35072)(1.23e-06,1.3193)(1.24e-06,1.28909)(1.25e-06,1.26001)(1.26e-06,1.23201)(1.27e-06,1.20504)(1.28e-06,1.17903)(1.29e-06,1.15394)(1.3e-06,1.12973)(1.31e-06,1.10634)(1.32e-06,1.08375)(1.33e-06,1.0619)(1.34e-06,1.04078)(1.35e-06,1.02034)(1.36e-06,1.00055)(1.37e-06,0.981389)(1.38e-06,0.962823)(1.39e-06,0.944827)(1.4e-06,0.927378)(1.41e-06,0.910453)(1.42e-06,0.89403)(1.43e-06,0.878088)(1.44e-06,0.862609)(1.45e-06,0.847573)(1.46e-06,0.832963)(1.47e-06,0.818764)(1.48e-06,0.804958)(1.49e-06,0.791532)(1.5e-06,0.778472)(1.51e-06,0.765763)(1.52e-06,0.753393)(1.53e-06,0.74135)(1.54e-06,0.729623)(1.55e-06,0.718199)(1.56e-06,0.707069)(1.57e-06,0.696223)(1.58e-06,0.685651)(1.59e-06,0.675343)(1.6e-06,0.665292)(1.61e-06,0.655487)(1.62e-06,0.645922)(1.63e-06,0.636589)(1.64e-06,0.62748)(1.65e-06,0.618588)(1.66e-06,0.609906)(1.67e-06,0.601428)(1.68e-06,0.593147)(1.69e-06,0.585057)(1.7e-06,0.577153)(1.71e-06,0.569429)(1.72e-06,0.561879)(1.73e-06,0.554499)(1.74e-06,0.547283)(1.75e-06,0.540227)(1.76e-06,0.533325)(1.77e-06,0.526574)(1.78e-06,0.519969)(1.79e-06,0.513506)(1.8e-06,0.507181)(1.81e-06,0.50099)(1.82e-06,0.494929)(1.83e-06,0.488995)(1.84e-06,0.483185)(1.85e-06,0.477494)(1.86e-06,0.47192)(1.87e-06,0.466459)(1.88e-06,0.461109)(1.89e-06,0.455866)(1.9e-06,0.450728)(1.91e-06,0.445692)(1.92e-06,0.440755)(1.93e-06,0.435914)(1.94e-06,0.431168)(1.95e-06,0.426513)(1.96e-06,0.421948)(1.97e-06,0.41747)(1.98e-06,0.413077)(1.99e-06,0.408766)(2e-06,0.404537)(2.01e-06,0.400385)
    };
    \addlegendentry{Приближённое решение 2}
    
    \addplot[
        color=black,
        mark=square,
        mark size=0.5pt
    ]
    coordinates {
    (0,0)(1e-08,0.000109057)(2e-08,0.000400118)(3e-08,0.00083585)(4e-08,0.00139581)(5e-08,0.00207128)(6e-08,0.00286184)(7e-08,0.00377308)(8e-08,0.00481519)(9e-08,0.00600214)(1e-07,0.00735126)(1.1e-07,0.00888309)(1.2e-07,0.0106215)(1.3e-07,0.0125938)(1.4e-07,0.0148312)(1.5e-07,0.0173692)(1.6e-07,0.0202479)(1.7e-07,0.0235131)(1.8e-07,0.0272164)(1.9e-07,0.0314165)(2e-07,0.0361797)(2.1e-07,0.0415812)(2.2e-07,0.0477062)(2.3e-07,0.0546509)(2.4e-07,0.0625244)(2.5e-07,0.07145)(2.6e-07,0.0815671)(2.7e-07,0.0930333)(2.8e-07,0.106027)(2.9e-07,0.120748)(3e-07,0.137424)(3.1e-07,0.156309)(3.2e-07,0.177693)(3.3e-07,0.201899)(3.4e-07,0.22929)(3.5e-07,0.260275)(3.6e-07,0.295313)(3.7e-07,0.334916)(3.8e-07,0.379658)(3.9e-07,0.430177)(4e-07,0.487186)(4.1e-07,0.551474)(4.2e-07,0.623916)(4.3e-07,0.70548)(4.4e-07,0.797229)(4.5e-07,0.900329)(4.6e-07,1.01606)(4.7e-07,1.1458)(4.8e-07,1.29107)(4.9e-07,1.45348)(5e-07,1.63478)(5.1e-07,1.83682)(5.2e-07,2.06159)(5.3e-07,2.31117)(5.4e-07,2.58776)(5.5e-07,2.89363)(5.6e-07,3.23119)(5.7e-07,3.60287)(5.8e-07,4.01121)(5.9e-07,4.45877)(6e-07,4.94813)(6.1e-07,5.48187)(6.2e-07,6.06246)(6.3e-07,6.69225)(6.4e-07,7.3733)(6.5e-07,8.1072)(6.6e-07,8.89486)(6.7e-07,9.73618)(6.8e-07,10.6297)(6.9e-07,11.572)(7e-07,12.5575)(7.1e-07,13.5778)(7.2e-07,14.6217)(7.3e-07,15.6751)(7.4e-07,16.7219)(7.5e-07,17.745)(7.6e-07,18.7278)(7.7e-07,19.6562)(7.8e-07,20.5193)(7.9e-07,21.3107)(8e-07,22.0282)(8.1e-07,22.6732)(8.2e-07,23.2495)(8.3e-07,23.7625)(8.4e-07,24.2186)(8.5e-07,24.624)(8.6e-07,24.9848)(8.7e-07,25.3064)(8.8e-07,25.5938)(8.9e-07,25.8515)(9e-07,26.0831)(9.1e-07,26.292)(9.2e-07,26.4811)(9.3e-07,26.6527)(9.4e-07,26.8089)(9.5e-07,26.9516)(9.6e-07,27.0823)(9.7e-07,27.2022)(9.8e-07,27.3127)(9.9e-07,27.4146)(1e-06,27.5089)(1.01e-06,27.5963)(1.02e-06,27.6774)(1.03e-06,27.753)(1.04e-06,27.8235)(1.05e-06,27.8893)(1.06e-06,27.9509)(1.07e-06,28.0086)(1.08e-06,28.0628)(1.09e-06,28.1137)(1.1e-06,28.1616)(1.11e-06,28.2068)(1.12e-06,28.2494)(1.13e-06,28.2897)(1.14e-06,28.3278)(1.15e-06,28.3639)(1.16e-06,28.3981)(1.17e-06,28.4306)(1.18e-06,28.4614)(1.19e-06,28.4907)(1.2e-06,28.5185)(1.21e-06,28.5451)(1.22e-06,28.5704)(1.23e-06,28.5945)(1.24e-06,28.6175)(1.25e-06,28.6395)(1.26e-06,28.6605)(1.27e-06,28.6806)(1.28e-06,28.6999)(1.29e-06,28.7183)(1.3e-06,28.7359)(1.31e-06,28.7529)(1.32e-06,28.7691)(1.33e-06,28.7847)(1.34e-06,28.7997)(1.35e-06,28.8141)(1.36e-06,28.8279)(1.37e-06,28.8412)(1.38e-06,28.854)(1.39e-06,28.8663)(1.4e-06,28.8782)(1.41e-06,28.8897)(1.42e-06,28.9007)(1.43e-06,28.9113)(1.44e-06,28.9216)(1.45e-06,28.9315)(1.46e-06,28.9411)(1.47e-06,28.9503)(1.48e-06,28.9593)(1.49e-06,28.9679)(1.5e-06,28.9763)(1.51e-06,28.9843)(1.52e-06,28.9922)(1.53e-06,28.9997)(1.54e-06,29.0071)(1.55e-06,29.0142)(1.56e-06,29.0211)(1.57e-06,29.0278)(1.58e-06,29.0342)(1.59e-06,29.0405)(1.6e-06,29.0466)(1.61e-06,29.0525)(1.62e-06,29.0583)(1.63e-06,29.0639)(1.64e-06,29.0693)(1.65e-06,29.0745)(1.66e-06,29.0797)(1.67e-06,29.0846)(1.68e-06,29.0895)(1.69e-06,29.0942)(1.7e-06,29.0988)(1.71e-06,29.1032)(1.72e-06,29.1076)(1.73e-06,29.1118)(1.74e-06,29.1159)(1.75e-06,29.1199)(1.76e-06,29.1239)(1.77e-06,29.1277)(1.78e-06,29.1314)(1.79e-06,29.135)(1.8e-06,29.1386)(1.81e-06,29.142)(1.82e-06,29.1454)(1.83e-06,29.1487)(1.84e-06,29.1519)(1.85e-06,29.155)(1.86e-06,29.1581)(1.87e-06,29.1611)(1.88e-06,29.164)(1.89e-06,29.1668)(1.9e-06,29.1696)(1.91e-06,29.1723)(1.92e-06,29.175)(1.93e-06,29.1776)(1.94e-06,29.1802)(1.95e-06,29.1827)(1.96e-06,29.1851)(1.97e-06,29.1875)(1.98e-06,29.1899)(1.99e-06,29.1922)(2e-06,29.1944)(2.01e-06,29.1966)
    };
    \addlegendentry{Приближённое решение 3}
    
    \addplot[
        color=magenta,
        mark=square,
        mark size=0.5pt
    ]
    coordinates {
    (0,29.3991)(1e-08,29.3988)(2e-08,29.3983)(3e-08,29.3979)(4e-08,29.3974)(5e-08,29.3968)(6e-08,29.3961)(7e-08,29.3954)(8e-08,29.3945)(9e-08,29.3936)(1e-07,29.3925)(1.1e-07,29.3912)(1.2e-07,29.3898)(1.3e-07,29.3882)(1.4e-07,29.3864)(1.5e-07,29.3843)(1.6e-07,29.382)(1.7e-07,29.3793)(1.8e-07,29.3763)(1.9e-07,29.3729)(2e-07,29.369)(2.1e-07,29.3646)(2.2e-07,29.3596)(2.3e-07,29.354)(2.4e-07,29.3475)(2.5e-07,29.3403)(2.6e-07,29.332)(2.7e-07,29.3227)(2.8e-07,29.3121)(2.9e-07,29.3001)(3e-07,29.2866)(3.1e-07,29.2712)(3.2e-07,29.2538)(3.3e-07,29.2341)(3.4e-07,29.2118)(3.5e-07,29.1866)(3.6e-07,29.1581)(3.7e-07,29.1259)(3.8e-07,29.0895)(3.9e-07,29.0484)(4e-07,29.002)(4.1e-07,28.9497)(4.2e-07,28.8908)(4.3e-07,28.8245)(4.4e-07,28.75)(4.5e-07,28.6662)(4.6e-07,28.5722)(4.7e-07,28.4668)(4.8e-07,28.3488)(4.9e-07,28.217)(5e-07,28.0698)(5.1e-07,27.9059)(5.2e-07,27.7235)(5.3e-07,27.521)(5.4e-07,27.2965)(5.5e-07,27.0483)(5.6e-07,26.7743)(5.7e-07,26.4723)(5.8e-07,26.1404)(5.9e-07,25.7762)(6e-07,25.3773)(6.1e-07,24.9413)(6.2e-07,24.4658)(6.3e-07,23.9484)(6.4e-07,23.3865)(6.5e-07,22.778)(6.6e-07,22.1209)(6.7e-07,21.414)(6.8e-07,20.657)(6.9e-07,19.8508)(7e-07,18.9982)(7.1e-07,18.1043)(7.2e-07,17.1769)(7.3e-07,16.2268)(7.4e-07,15.2674)(7.5e-07,14.3141)(7.6e-07,13.3834)(7.7e-07,12.4913)(7.8e-07,11.6513)(7.9e-07,10.8738)(8e-07,10.165)(8.1e-07,9.52695)(8.2e-07,8.9581)(8.3e-07,8.45434)(8.4e-07,8.00998)(8.5e-07,7.61862)(8.6e-07,7.27379)(8.7e-07,6.96944)(8.8e-07,6.70007)(8.9e-07,6.46088)(9e-07,6.24773)(9.1e-07,6.05709)(9.2e-07,5.88595)(9.3e-07,5.73179)(9.4e-07,5.59246)(9.5e-07,5.46613)(9.6e-07,5.35125)(9.7e-07,5.2465)(9.8e-07,5.15073)(9.9e-07,5.06297)(1e-06,4.98237)(1.01e-06,4.90819)(1.02e-06,4.83978)(1.03e-06,4.77659)(1.04e-06,4.71812)(1.05e-06,4.66392)(1.06e-06,4.61362)(1.07e-06,4.56688)(1.08e-06,4.52338)(1.09e-06,4.48285)(1.1e-06,4.44505)(1.11e-06,4.40975)(1.12e-06,4.37676)(1.13e-06,4.34591)(1.14e-06,4.31701)(1.15e-06,4.28994)(1.16e-06,4.26456)(1.17e-06,4.24074)(1.18e-06,4.21837)(1.19e-06,4.19736)(1.2e-06,4.1776)(1.21e-06,4.15901)(1.22e-06,4.14152)(1.23e-06,4.12505)(1.24e-06,4.10953)(1.25e-06,4.09491)(1.26e-06,4.08112)(1.27e-06,4.06812)(1.28e-06,4.05584)(1.29e-06,4.04426)(1.3e-06,4.03333)(1.31e-06,4.023)(1.32e-06,4.01324)(1.33e-06,4.00402)(1.34e-06,3.9953)(1.35e-06,3.98706)(1.36e-06,3.97926)(1.37e-06,3.97189)(1.38e-06,3.96492)(1.39e-06,3.95831)(1.4e-06,3.95207)(1.41e-06,3.94615)(1.42e-06,3.94055)(1.43e-06,3.93525)(1.44e-06,3.93023)(1.45e-06,3.92547)(1.46e-06,3.92097)(1.47e-06,3.9167)(1.48e-06,3.91265)(1.49e-06,3.90882)(1.5e-06,3.90519)(1.51e-06,3.90174)(1.52e-06,3.89848)(1.53e-06,3.89539)(1.54e-06,3.89245)(1.55e-06,3.88967)(1.56e-06,3.88703)(1.57e-06,3.88453)(1.58e-06,3.88216)(1.59e-06,3.87991)(1.6e-06,3.87777)(1.61e-06,3.87575)(1.62e-06,3.87383)(1.63e-06,3.87201)(1.64e-06,3.87028)(1.65e-06,3.86864)(1.66e-06,3.86708)(1.67e-06,3.86561)(1.68e-06,3.86421)(1.69e-06,3.86288)(1.7e-06,3.86162)(1.71e-06,3.86042)(1.72e-06,3.85929)(1.73e-06,3.85821)(1.74e-06,3.85719)(1.75e-06,3.85622)(1.76e-06,3.8553)(1.77e-06,3.85442)(1.78e-06,3.85359)(1.79e-06,3.85281)(1.8e-06,3.85206)(1.81e-06,3.85135)(1.82e-06,3.85068)(1.83e-06,3.85004)(1.84e-06,3.84943)(1.85e-06,3.84885)(1.86e-06,3.84831)(1.87e-06,3.84779)(1.88e-06,3.8473)(1.89e-06,3.84683)(1.9e-06,3.84639)(1.91e-06,3.84596)(1.92e-06,3.84556)(1.93e-06,3.84519)(1.94e-06,3.84483)(1.95e-06,3.84448)(1.96e-06,3.84416)(1.97e-06,3.84385)(1.98e-06,3.84356)(1.99e-06,3.84328)(2e-06,3.84302)(2.01e-06,3.84277)
    };
    \addlegendentry{Приближённое решение 4}
    
    \addplot[
        color=green,
        mark=square,
        mark size=0.5pt
    ]
    coordinates {
    (0,0)(1e-08,0.000104621)(2e-08,0.000370316)(3e-08,0.000750298)(4e-08,0.0012211)(5e-08,0.00177358)(6e-08,0.00240745)(7e-08,0.00312807)(8e-08,0.00394461)(9e-08,0.00486907)(1e-07,0.00591582)(1.1e-07,0.00710151)(1.2e-07,0.00844507)(1.3e-07,0.00996798)(1.4e-07,0.0116945)(1.5e-07,0.0136522)(1.6e-07,0.015872)(1.7e-07,0.0183892)(1.8e-07,0.0212433)(1.9e-07,0.0244795)(2e-07,0.0281485)(2.1e-07,0.032308)(2.2e-07,0.0370228)(2.3e-07,0.0423666)(2.4e-07,0.0484224)(2.5e-07,0.055284)(2.6e-07,0.063057)(2.7e-07,0.0718609)(2.8e-07,0.0818299)(2.9e-07,0.0931153)(3e-07,0.105887)(3.1e-07,0.120336)(3.2e-07,0.136676)(3.3e-07,0.155147)(3.4e-07,0.176016)(3.5e-07,0.199581)(3.6e-07,0.226176)(3.7e-07,0.256168)(3.8e-07,0.289964)(3.9e-07,0.328013)(4e-07,0.370809)(4.1e-07,0.41889)(4.2e-07,0.472843)(4.3e-07,0.5333)(4.4e-07,0.600942)(4.5e-07,0.676495)(4.6e-07,0.760724)(4.7e-07,0.854429)(4.8e-07,0.958438)(4.9e-07,1.07359)(5e-07,1.20074)(5.1e-07,1.3407)(5.2e-07,1.49428)(5.3e-07,1.6622)(5.4e-07,1.84512)(5.5e-07,2.04358)(5.6e-07,2.25797)(5.7e-07,2.48854)(5.8e-07,2.7353)(5.9e-07,2.99806)(6e-07,3.27635)(6.1e-07,3.56938)(6.2e-07,3.87603)(6.3e-07,4.19473)(6.4e-07,4.52337)(6.5e-07,4.85922)(6.6e-07,5.19869)(6.7e-07,5.53719)(6.8e-07,5.86881)(6.9e-07,6.18613)(7e-07,6.48002)(7.1e-07,6.73959)(7.2e-07,6.95244)(7.3e-07,7.10542)(7.4e-07,7.1858)(7.5e-07,7.1831)(7.6e-07,7.0911)(7.7e-07,6.90967)(7.8e-07,6.64579)(7.9e-07,6.31312)(8e-07,5.93022)(8.1e-07,5.51773)(8.2e-07,5.09542)(8.3e-07,4.67986)(8.4e-07,4.28322)(8.5e-07,3.91316)(8.6e-07,3.57345)(8.7e-07,3.26499)(8.8e-07,2.98677)(8.9e-07,2.7367)(9e-07,2.51221)(9.1e-07,2.31058)(9.2e-07,2.12925)(9.3e-07,1.96582)(9.4e-07,1.8182)(9.5e-07,1.68449)(9.6e-07,1.5631)(9.7e-07,1.45259)(9.8e-07,1.35176)(9.9e-07,1.25954)(1e-06,1.175)(1.01e-06,1.09736)(1.02e-06,1.0259)(1.03e-06,0.960011)(1.04e-06,0.899156)(1.05e-06,0.842859)(1.06e-06,0.790699)(1.07e-06,0.742303)(1.08e-06,0.69734)(1.09e-06,0.655514)(1.1e-06,0.61656)(1.11e-06,0.580242)(1.12e-06,0.546345)(1.13e-06,0.514677)(1.14e-06,0.485065)(1.15e-06,0.457351)(1.16e-06,0.431394)(1.17e-06,0.407062)(1.18e-06,0.384238)(1.19e-06,0.362814)(1.2e-06,0.342691)(1.21e-06,0.323779)(1.22e-06,0.305996)(1.23e-06,0.289264)(1.24e-06,0.273514)(1.25e-06,0.25868)(1.26e-06,0.244705)(1.27e-06,0.231532)(1.28e-06,0.21911)(1.29e-06,0.207392)(1.3e-06,0.196334)(1.31e-06,0.185896)(1.32e-06,0.17604)(1.33e-06,0.16673)(1.34e-06,0.157934)(1.35e-06,0.149622)(1.36e-06,0.141763)(1.37e-06,0.134333)(1.38e-06,0.127306)(1.39e-06,0.120659)(1.4e-06,0.11437)(1.41e-06,0.108419)(1.42e-06,0.102786)(1.43e-06,0.0974531)(1.44e-06,0.0924044)(1.45e-06,0.0876234)(1.46e-06,0.0830955)(1.47e-06,0.0788066)(1.48e-06,0.0747436)(1.49e-06,0.0708941)(1.5e-06,0.0672465)(1.51e-06,0.0637899)(1.52e-06,0.0605138)(1.53e-06,0.0574085)(1.54e-06,0.054465)(1.55e-06,0.0516744)(1.56e-06,0.0490287)(1.57e-06,0.0465201)(1.58e-06,0.0441414)(1.59e-06,0.0418856)(1.6e-06,0.0397463)(1.61e-06,0.0377173)(1.62e-06,0.0357929)(1.63e-06,0.0339675)(1.64e-06,0.032236)(1.65e-06,0.0305934)(1.66e-06,0.0290351)(1.67e-06,0.0275567)(1.68e-06,0.0261541)(1.69e-06,0.0248234)(1.7e-06,0.0235607)(1.71e-06,0.0223626)(1.72e-06,0.0212257)(1.73e-06,0.0201469)(1.74e-06,0.0191231)(1.75e-06,0.0181516)(1.76e-06,0.0172297)(1.77e-06,0.0163547)(1.78e-06,0.0155244)(1.79e-06,0.0147363)(1.8e-06,0.0139883)(1.81e-06,0.0132785)(1.82e-06,0.0126047)(1.83e-06,0.0119652)(1.84e-06,0.0113582)(1.85e-06,0.0107821)(1.86e-06,0.0102353)(1.87e-06,0.00971621)(1.88e-06,0.00922352)(1.89e-06,0.00875586)(1.9e-06,0.00831193)(1.91e-06,0.00789055)(1.92e-06,0.00749056)(1.93e-06,0.00711086)(1.94e-06,0.00675043)(1.95e-06,0.00640829)(1.96e-06,0.0060835)(1.97e-06,0.00577519)(1.98e-06,0.00548251)(1.99e-06,0.00520467)(2e-06,0.00494093)(2.01e-06,0.00469055)
    };
    \addlegendentry{Приближённое решение 5}
    
    \addplot[
        color=orange,
        mark=square,
        mark size=0.5pt
    ]
    coordinates {
    (0,0)(1e-08,4.43585e-06)(2e-08,2.98014e-05)(3e-08,8.55487e-05)(4e-08,0.000174693)(5e-08,0.000297669)(6e-08,0.000454322)(7e-08,0.000644872)(8e-08,0.000870328)(9e-08,0.00113264)(1e-07,0.00143474)(1.1e-07,0.00178051)(1.2e-07,0.00217481)(1.3e-07,0.00262344)(1.4e-07,0.00313323)(1.5e-07,0.00371208)(1.6e-07,0.00436903)(1.7e-07,0.00511444)(1.8e-07,0.00596006)(1.9e-07,0.00691926)(2e-07,0.0080072)(2.1e-07,0.00924107)(2.2e-07,0.0106403)(2.3e-07,0.0122271)(2.4e-07,0.0140263)(2.5e-07,0.0160661)(2.6e-07,0.0183787)(2.7e-07,0.0210001)(2.8e-07,0.0239711)(2.9e-07,0.0273381)(3e-07,0.0311531)(3.1e-07,0.035475)(3.2e-07,0.0403701)(3.3e-07,0.0459133)(3.4e-07,0.0521886)(3.5e-07,0.0592909)(3.6e-07,0.0673264)(3.7e-07,0.0764146)(3.8e-07,0.0866892)(3.9e-07,0.0983)(4e-07,0.111414)(4.1e-07,0.126218)(4.2e-07,0.14292)(4.3e-07,0.161749)(4.4e-07,0.182962)(4.5e-07,0.206841)(4.6e-07,0.233697)(4.7e-07,0.263873)(4.8e-07,0.297745)(4.9e-07,0.335725)(5e-07,0.378263)(5.1e-07,0.425847)(5.2e-07,0.479013)(5.3e-07,0.538341)(5.4e-07,0.604464)(5.5e-07,0.678071)(5.6e-07,0.759916)(5.7e-07,0.850827)(5.8e-07,0.951719)(5.9e-07,1.06361)(6e-07,1.18764)(6.1e-07,1.32511)(6.2e-07,1.47748)(6.3e-07,1.64645)(6.4e-07,1.83398)(6.5e-07,2.04237)(6.6e-07,2.27426)(6.7e-07,2.53276)(6.8e-07,2.82147)(6.9e-07,3.1445)(7e-07,3.5065)(7.1e-07,3.91251)(7.2e-07,4.36778)(7.3e-07,4.87731)(7.4e-07,5.44514)(7.5e-07,6.07339)(7.6e-07,6.7611)(7.7e-07,7.5031)(7.8e-07,8.28944)(7.9e-07,9.10563)(8e-07,9.93392)(8.1e-07,10.7555)(8.2e-07,11.553)(8.3e-07,12.3122)(8.4e-07,13.0232)(8.5e-07,13.6807)(8.6e-07,14.2831)(8.7e-07,14.8316)(8.8e-07,15.3292)(8.9e-07,15.7801)(9e-07,16.1885)(9.1e-07,16.5588)(9.2e-07,16.8951)(9.3e-07,17.2012)(9.4e-07,17.4804)(9.5e-07,17.7356)(9.6e-07,17.9695)(9.7e-07,18.1843)(9.8e-07,18.382)(9.9e-07,18.5645)(1e-06,18.7331)(1.01e-06,18.8892)(1.02e-06,19.0341)(1.03e-06,19.1688)(1.04e-06,19.2941)(1.05e-06,19.411)(1.06e-06,19.5202)(1.07e-06,19.6223)(1.08e-06,19.7179)(1.09e-06,19.8075)(1.1e-06,19.8916)(1.11e-06,19.9706)(1.12e-06,20.045)(1.13e-06,20.115)(1.14e-06,20.181)(1.15e-06,20.2432)(1.16e-06,20.302)(1.17e-06,20.3576)(1.18e-06,20.4102)(1.19e-06,20.4599)(1.2e-06,20.5071)(1.21e-06,20.5518)(1.22e-06,20.5941)(1.23e-06,20.6344)(1.24e-06,20.6726)(1.25e-06,20.7089)(1.26e-06,20.7435)(1.27e-06,20.7764)(1.28e-06,20.8077)(1.29e-06,20.8375)(1.3e-06,20.866)(1.31e-06,20.8931)(1.32e-06,20.9189)(1.33e-06,20.9436)(1.34e-06,20.9672)(1.35e-06,20.9898)(1.36e-06,21.0113)(1.37e-06,21.0319)(1.38e-06,21.0516)(1.39e-06,21.0705)(1.4e-06,21.0886)(1.41e-06,21.1059)(1.42e-06,21.1225)(1.43e-06,21.1384)(1.44e-06,21.1536)(1.45e-06,21.1683)(1.46e-06,21.1823)(1.47e-06,21.1958)(1.48e-06,21.2087)(1.49e-06,21.2212)(1.5e-06,21.2332)(1.51e-06,21.2447)(1.52e-06,21.2558)(1.53e-06,21.2664)(1.54e-06,21.2767)(1.55e-06,21.2865)(1.56e-06,21.2961)(1.57e-06,21.3052)(1.58e-06,21.3141)(1.59e-06,21.3226)(1.6e-06,21.3308)(1.61e-06,21.3388)(1.62e-06,21.3464)(1.63e-06,21.3538)(1.64e-06,21.361)(1.65e-06,21.3679)(1.66e-06,21.3745)(1.67e-06,21.381)(1.68e-06,21.3872)(1.69e-06,21.3933)(1.7e-06,21.3991)(1.71e-06,21.4048)(1.72e-06,21.4102)(1.73e-06,21.4155)(1.74e-06,21.4207)(1.75e-06,21.4257)(1.76e-06,21.4305)(1.77e-06,21.4352)(1.78e-06,21.4397)(1.79e-06,21.4441)(1.8e-06,21.4484)(1.81e-06,21.4526)(1.82e-06,21.4566)(1.83e-06,21.4606)(1.84e-06,21.4644)(1.85e-06,21.4681)(1.86e-06,21.4717)(1.87e-06,21.4752)(1.88e-06,21.4786)(1.89e-06,21.4819)(1.9e-06,21.4852)(1.91e-06,21.4883)(1.92e-06,21.4914)(1.93e-06,21.4944)(1.94e-06,21.4973)(1.95e-06,21.5001)(1.96e-06,21.5029)(1.97e-06,21.5056)(1.98e-06,21.5082)(1.99e-06,21.5108)(2e-06,21.5133)(2.01e-06,21.5158)
    };
    \addlegendentry{Приближённое решение 6}
    
    \end{axis}
    \end{tikzpicture}
        \caption{}
    \end{subfigure}
    \hfill
    \begin{subfigure}[t]{0.45\linewidth}
        \centering
        \begin{tikzpicture}[scale = 0.8]
    \begin{axis}[
        enlargelimits=true,
        xlabel={$t, \text{с}$},
        ylabel={$T, \text{К}$},
        legend style={at={(1,1.25)},
            anchor=north east},
        ymajorgrids=true,
        grid style=dashed,
    ]
    
    \addplot[
        color=blue,
        mark=square,
        mark size=0.5pt
    ]
    coordinates {
    (0,2300)(1e-08,2299.99)(2e-08,2299.98)(3e-08,2299.98)(4e-08,2299.97)(5e-08,2299.97)(6e-08,2299.97)(7e-08,2299.97)(8e-08,2299.98)(9e-08,2299.98)(1e-07,2299.98)(1.1e-07,2299.98)(1.2e-07,2299.99)(1.3e-07,2299.99)(1.4e-07,2299.99)(1.5e-07,2300)(1.6e-07,2300)(1.7e-07,2300.01)(1.8e-07,2300.01)(1.9e-07,2300.02)(2e-07,2300.03)(2.1e-07,2300.04)(2.2e-07,2300.05)(2.3e-07,2300.06)(2.4e-07,2300.08)(2.5e-07,2300.1)(2.6e-07,2300.12)(2.7e-07,2300.14)(2.8e-07,2300.17)(2.9e-07,2300.2)(3e-07,2300.24)(3.1e-07,2300.29)(3.2e-07,2300.34)(3.3e-07,2300.41)(3.4e-07,2300.48)(3.5e-07,2300.57)(3.6e-07,2300.69)(3.7e-07,2300.82)(3.8e-07,2300.98)(3.9e-07,2301.18)(4e-07,2301.41)(4.1e-07,2301.7)(4.2e-07,2302.06)(4.3e-07,2302.49)(4.4e-07,2303.02)(4.5e-07,2303.67)(4.6e-07,2304.48)(4.7e-07,2305.46)(4.8e-07,2306.67)(4.9e-07,2308.16)(5e-07,2309.98)(5.1e-07,2312.22)(5.2e-07,2314.95)(5.3e-07,2318.29)(5.4e-07,2322.34)(5.5e-07,2327.26)(5.6e-07,2333.2)(5.7e-07,2340.33)(5.8e-07,2348.87)(5.9e-07,2359.02)(6e-07,2371.02)(6.1e-07,2385.11)(6.2e-07,2401.56)(6.3e-07,2420.6)(6.4e-07,2442.48)(6.5e-07,2467.4)(6.6e-07,2495.53)(6.7e-07,2526.96)(6.8e-07,2561.71)(6.9e-07,2599.67)(7e-07,2640.6)(7.1e-07,2684.1)(7.2e-07,2729.59)(7.3e-07,2776.33)(7.4e-07,2823.42)(7.5e-07,2869.89)(7.6e-07,2914.72)(7.7e-07,2957.03)(7.8e-07,2996.08)(7.9e-07,3031.37)(8e-07,3062.67)(8.1e-07,3090)(8.2e-07,3113.57)(8.3e-07,3133.71)(8.4e-07,3150.82)(8.5e-07,3165.32)(8.6e-07,3177.6)(8.7e-07,3188.02)(8.8e-07,3196.88)(8.9e-07,3204.45)(9e-07,3210.93)(9.1e-07,3216.52)(9.2e-07,3221.35)(9.3e-07,3225.56)(9.4e-07,3229.23)(9.5e-07,3232.45)(9.6e-07,3235.3)(9.7e-07,3237.81)(9.8e-07,3240.05)(9.9e-07,3242.05)(1e-06,3243.85)(1.01e-06,3245.47)(1.02e-06,3246.93)(1.03e-06,3248.25)(1.04e-06,3249.46)(1.05e-06,3250.57)(1.06e-06,3251.58)(1.07e-06,3252.52)(1.08e-06,3253.38)(1.09e-06,3254.18)(1.1e-06,3254.93)(1.11e-06,3255.62)(1.12e-06,3256.27)(1.13e-06,3256.88)(1.14e-06,3257.45)(1.15e-06,3258)(1.16e-06,3258.51)(1.17e-06,3258.99)(1.18e-06,3259.45)(1.19e-06,3259.89)(1.2e-06,3260.31)(1.21e-06,3260.72)(1.22e-06,3261.1)(1.23e-06,3261.47)(1.24e-06,3261.82)(1.25e-06,3262.17)(1.26e-06,3262.5)(1.27e-06,3262.81)(1.28e-06,3263.12)(1.29e-06,3263.42)(1.3e-06,3263.71)(1.31e-06,3263.99)(1.32e-06,3264.26)(1.33e-06,3264.52)(1.34e-06,3264.78)(1.35e-06,3265.03)(1.36e-06,3265.27)(1.37e-06,3265.51)(1.38e-06,3265.74)(1.39e-06,3265.97)(1.4e-06,3266.19)(1.41e-06,3266.4)(1.42e-06,3266.61)(1.43e-06,3266.82)(1.44e-06,3267.02)(1.45e-06,3267.21)(1.46e-06,3267.4)(1.47e-06,3267.59)(1.48e-06,3267.78)(1.49e-06,3267.96)(1.5e-06,3268.13)(1.51e-06,3268.3)(1.52e-06,3268.47)(1.53e-06,3268.64)(1.54e-06,3268.8)(1.55e-06,3268.96)(1.56e-06,3269.12)(1.57e-06,3269.27)(1.58e-06,3269.42)(1.59e-06,3269.57)(1.6e-06,3269.71)(1.61e-06,3269.85)(1.62e-06,3269.99)(1.63e-06,3270.13)(1.64e-06,3270.26)(1.65e-06,3270.39)(1.66e-06,3270.52)(1.67e-06,3270.64)(1.68e-06,3270.77)(1.69e-06,3270.89)(1.7e-06,3271.01)(1.71e-06,3271.13)(1.72e-06,3271.24)(1.73e-06,3271.35)(1.74e-06,3271.46)(1.75e-06,3271.57)(1.76e-06,3271.68)(1.77e-06,3271.78)(1.78e-06,3271.89)(1.79e-06,3271.99)(1.8e-06,3272.09)(1.81e-06,3272.18)(1.82e-06,3272.28)(1.83e-06,3272.37)(1.84e-06,3272.47)(1.85e-06,3272.56)(1.86e-06,3272.65)(1.87e-06,3272.73)(1.88e-06,3272.82)(1.89e-06,3272.9)(1.9e-06,3272.99)(1.91e-06,3273.07)(1.92e-06,3273.15)(1.93e-06,3273.23)(1.94e-06,3273.31)(1.95e-06,3273.38)(1.96e-06,3273.46)(1.97e-06,3273.53)(1.98e-06,3273.6)(1.99e-06,3273.67)(2e-06,3273.74)(2.01e-06,3273.81)
    };
    \addlegendentry{Температура}
    
    \end{axis}
    \end{tikzpicture}
        \caption{}
    \end{subfigure}
    \hfill
    \caption{Моделирование взрыва: а) Мольно-массовые концентрации; б) Температура}
    \label{fig:bomb}
\end{figure}

Конечное значение температуры стало равно примерно $3285K$. Сами концентрации меняются быстрее, чем в предыдущем примере и система
приходит в равновесие примерно за 2 микросекунды.