\abstract % Структурный элемент: РЕФЕРАТ

\keywords{ЧИСЛЕННЫЕ МЕТОДЫ, МЕТОД КРАНКА-НИКОЛСОНА, ЖЁСТКИЕ СИСТЕМЫ, ХИМИЧЕСКАЯ КИНЕТИКА, ТУРБУЛЕНТНОСТЬ, ВЯЗКОСТЬ, РЕЙНОЛЬДС}

Объектом исследования является моделирование турбулентных течений.

Цель работы ~--– разработка программы для моделирования уравнений газовой динамики в турбулентных и ламинарных потоках.

Значительный прогресс в понимании природы и свойств турбулентности произошёл в последние десятилетия благодаря успехам теории динамических систем, позволившим понять как хаотическое поведение возникает в детерминированных системах.

Также развитие численных методов решения систем дифференциальных уравнений и увеличение расчётных мощностей современных компьютеров позволяют применять различные современные модели турбулентности и усовершенствовать их на основании результатов численных экспериментов.

В процессе работы были созданы отдельные модули для решения задач химической кинетики, моделирования ламинарных и турбулентных потоков и генерации отчётов решения/моделирования. Так же были созданы модули для графического отображения результатов и взаимодействия с пользователем.

В качестве основного решателя используется схема Кранка-Николсона, решающая систему уравнений итеративно. На каждой итерации отдельно решается система дифференциальных уравнений для моделирования горения химических компонент с использованием явной схемы Рунге-Кутты 4-го порядка.

В дальнейшем результаты работы можно использовать для научных расчётов и моделирования физических процессов течения струй.
