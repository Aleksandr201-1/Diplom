\introduction % Структурный элемент: ВВЕДЕНИЕ

При всём разнообразии теплоэнергетических устройств процесс сжигания топлива происходит практически по единой схеме, которую можно иллюстрировать на примере газового факела. Для рассматриваемого течения характерно наличие двух разделённых до некоторого момента многокомпонентных потоков газа с разными физическими (температура, скорость) и химическими (молекулярный вес, концентрации компонентов) свойствами, причем потоки могут состоять не только из топлива и окислителя в чистом виде, но и представлять готовую горючую смесь или продукты сгорания. Скорости потоков могут быть дозвуковыми (энергетические установки) и сверхзвуковыми (реактивные двигатели), а сам процесс истечения может быть ламинарным или турбулентным в зависимости от конкретных параметров горелки.

Задачу моделирования течения спутных струй с газовой смесью можно условно разделить на 2 подзадачи: моделирование течения потока и моделирование горения химических компонент. Первая подзадача связана с решением уравнений Навье–Стокса или их упрощённых аналогов, учитывающих турбулентность, сжимаемость и другие физические эффекты. Вторая подзадача требует описания кинетики химических реакций, включая процессы диссоциации, рекомбинации и горения, которые часто протекают в условиях значительного отклонения от термодинамического равновесия.

Современные вычислительные методы и алгоритмы позволяют проводить комплексное моделирование таких течений, однако остаются открытыми вопросы, связанные с повышением точности, устойчивости и эффективности расчётов. В данной работе рассматриваются подходы к численному моделированию неравновесных химических процессов в струйных течениях, а также анализируются их преимущества и ограничения.

В рамках данной работы была разработана программа для численного моделирования течений газа, учитывающая как ламинарные, так и турбулентные режимы. Программа основана на решении усреднённых уравнений Навье–Стокса с использованием моделей турбулентности (например, $k-\epsilon$, $SST$ и так далее), а также включает модуль расчёта химических реакций в приближении конечной скорости. Для верификации алгоритма проведено сравнение с экспериментальными данными для различных конфигураций струй, включая случаи с закритическими давлениями и высокоскоростными течениями. Результаты демонстрируют хорошее соответствие расчётных и экспериментальных распределений температуры, концентраций компонент и скорости потока. Помимо этого был реализован отдельный модуль для работы с СДУ при помощи множества методов семейства Рунге-Кутты. В работе реализовано 18 явных методов до 6 порядка точности, 9 вложенных, включая схему Дормана-Принса 4-5 порядка, 19 неявных до 6 порядка. Для решения СНУ при итерировании в неявных схемах используются схемы первого порядка (простой итерации, Зейделя) и второго порядка (метод Ньютона), причём для обращения матрицы применялся метод LU-разложения. Для дифференцирования функции при построении матрицы Якоби для метода Ньютоны использовались формулы с 4 порядком точности. При необходимости можно использовать формулы с меньшим порядком.

Разработанный программный комплекс позволяет проводить расчёты с приемлемой точностью при умеренных вычислительных затратах, что делает его применимым как для научных исследований, так и для инженерных расчётов. Дальнейшее развитие работы может быть связано с внедрением более детальных моделей турбулентности, учётом радиационного теплообмена и оптимизацией вычислительных алгоритмов для расчёта сложных химических кинетик.
