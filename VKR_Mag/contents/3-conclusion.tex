\conclusion

В данной работе проведено исследование численного моделирования течения струй газа с неравновесными химическими процессами. Разработана математическая модель и программный комплекс, позволяющий рассчитывать как ламинарные, так и турбулентные режимы течения с учетом химических реакций. Основные результаты работы заключаются в следующем:

\begin{enumerate}[arabic]
\item Разработана и реализована параболизированная модель струйных течений, позволяющая существенно сократить вычислительные затраты при сохранении приемлемой точности расчетов. Показано, что в области развитого течения ($x/d > 10$) погрешность модели не превышает $5-8\%$ по сравнению с полной системой уравнений Навье-Стокса и экспериментальными данными.
\item Создано программное обеспечение, дополненное модулями для расчета химической кинетики. Программа обеспечивает:
\begin{enumerate}
    \item Моделирование многослойных струйных течений
    \item Учет сложных механизмов химических реакций
    \item Визуализацию полей скорости, температуры и концентраций
\end{enumerate}
\item Проведена верификация модели на ряде тестовых случаев, включая:
\begin{enumerate}
    \item Сравнение с экспериментальными данными по структуре турбулентных струй
    \item Анализ чувствительности к параметрам турбулентности и химической кинетики
    \item Оценку погрешностей в различных областях течения
\end{enumerate}
\item Выявлены границы применимости параболизированного подхода:
\begin{enumerate}
    \item Модель дает хорошие результаты для дальнего поля струи ($x/d > 10$)
    \item Требует осторожного применения в зонах с обратными течениями
    \item Эффективна для инженерных расчетов при ограниченных ресурсах
\end{enumerate}
\item Разработан пользовательский интерфейс для отображения результатов в удобном виде
\end{enumerate}

Отдельно стоит уделить внимание модулю для работы с СДУ при помощи множества методов семейства Рунге-Кутты. В работе реализовано 62 схемы со 2 по 6 порядок точности. Из них 18 явных со 2 по 6 порядок точности, 9 вложенных, включая схему Дормана-Принса 4(5) порядка, 22 неявных, в том числе схемы Радо, Гаусса и Лобатто для полных и неполных матриц. Помимо этого, протестирован один L-стабильный диагональный метод. Для неявных схем используются схемы решения САУ первого порядка (простой итерации, Зейделя) и второго порядка (метод Ньютона), причём для обращения матрицы применялся метод LU-разложения. Для дифференцирования функции при построении матрицы Якоби для метода Ньютоны использовались формулы с 4 порядком точности. При необходимости можно использовать формулы с меньшим порядком.

Перспективы дальнейших исследований:
\begin{itemize}
    \item Внедрение более точных моделей турбулентного горения
    \item Учет радиационного теплообмена
    \item Оптимизация вычислительных алгоритмов для работы с большими химическими механизмами
    \item Разработка гибридных методов, сочетающих параболизированный подход с полным CFD-моделированием в критических областях
\end{itemize}

Практическая значимость работы заключается в создании инструмента для инженерных расчетов реактивных струй, который может быть использован при проектировании двигательных установок, систем сгорания и других устройств, где важны точные прогнозы параметров течения с химическими реакциями.

Таким образом, проведенное исследование демонстрирует возможность эффективного моделирования сложных неравновесных процессов в струйных течениях при разумных вычислительных затратах, что открывает перспективы для дальнейшего совершенствования методов численного анализа в этой области.