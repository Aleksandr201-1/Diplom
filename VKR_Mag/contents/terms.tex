\newglossaryentry{id1}{ % Нужны разные id, можно ставить просто последовательно
    name={Жёсткая система},
    description={ОДУ, численное решение которого явными методами является неудовлетворительным из-за резкого увеличения
    числа вычислений или из-за резкого возрастания погрешности при недостаточно малом шаге} 
}
\newglossaryentry{id2}{
    name={Методы Рунге-Кутты},
    description={большой класс численных методов решения задачи Коши для обыкновенных дифференциальных уравнений и их систем} 
}
\newglossaryentry{id3}{
    name={Условие химического равновесия},
    description={равенство полных химических потенциалов исходных веществ и продуктов} 
}
\newglossaryentry{id4}{
    name={Химическая кинетика},
    description={раздел физической химии, изучающий закономерности протекания химических реакций во времени, зависимости этих
    закономерностей от внешних условий, а также механизмы химических превращений} 
}
\newglossaryentry{id5}{
    name={Критерий Рейнольдса},
    description={отношение инерционных сил к вязким, определяющее этапы перехода от ламинарных течений к турбулентным} 
}
\newglossaryentry{id6}{
    name={Турбулентность},
    description={сложное, неупорядоченное во времени и пространстве поведение диссипативной среды (или поля), детали которого не могут быть воспроизведены на больших интервалах времени при сколь угодно точном задании начальных и граничных условий} 
}
\newglossaryentry{id7}{
    name={Ламинарный поток},
    description={упорядоченное течение жидкости или газа, при котором жидкость или газ перемещается как бы слоями, параллельными направлению течения} 
}
\newglossaryentry{id8}{
    name={Турбулентная вязкость},
    description={мера сопротивления течению, вызванного турбулентными флуктуациями} 
}
\newglossaryentry{id9}{
    name={Число Прандтля},
    description={характеристика влияния свойств жидкости на интенсивность теплообмена, являющаяся критерием подобия температурного и скоростного полей, а также описывает свойства теплоносителя} 
}
\newglossaryentry{id10}{
    name={Число Шмидта},
    description={отношение коэффициентов кинематической вязкости и диффузии, описывающее относительную роль молекулярных процессов переноса количества движения и переноса массы примеси диффузией} 
}
\newglossaryentry{id11}{
    name={Число Льюиса},
    description={соотношение между интенсивностями переноса массы компонента диффузией и переноса теплоты теплопроводностью, пользующееся для характеристики потоков жидкости, в которых одновременно происходит тепломассообмен} 
}
\newglossaryentry{id12}{
    name={Неравновесный химический процесс},
    description={реакция, скорость которой определяется кинетикой, а не термодинамическим равновесием} 
}
\newglossaryentry{id13}{
    name={Скорость химической реакции},
    description={изменение концентрации реагентов или продуктов во времени} 
}
\newglossaryentry{id14}{
    name={Сетка (расчётная область)},
    description={дискретное представление пространства, на котором решаются уравнения} 
}
\newglossaryentry{id15}{
    name={Ламинарное горение},
    description={режим горения без турбулентных пульсаций} 
}
\newglossaryentry{id16}{
    name={Турбулентное горение},
    description={горение, осложнённое турбулентным перемешиванием} 
}
\newglossaryentry{id17}{
    name={Параболизированные уравнения Навье-Стокса},
    description={упрощённая система уравнений, в которой пренебрегают обратными течениями, сохраняя только доминирующее направление потока} 
}
\newglossaryentry{id18}{
    name={Струйное течение},
    description={течение газа, истекающего из сопла или отверстия в окружающую среду} 
}
\newglossaryentry{id19}{
    name={Пограничный слой},
    description={тонкая область у поверхности, где вязкие эффекты существенны} 
}
\newglossaryentry{id20}{
    name={Верификация},
    description={проверка корректности реализации численного метода} 
}
\newglossaryentry{id21}{
    name={Валидация},
    description={сравнение результатов моделирования с экспериментальными данными} 
}
