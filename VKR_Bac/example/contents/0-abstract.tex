% O текст про жёсткость (ссылки)
% O устойчивость (а-, л-устойчивость, критерии устойчивости)
% O литература (30-40)
% V полунеявные методы в отдельный пункт
% O списки и переносы на новые страницы
% V уменьшить шрифт букв на картинках
% V поменять пунк про пользовательский интерфейс и парсер (вывод для парсера)
% V названия для задач
% O убрать температуру и давление из систем (перенести их на строчку повыше)
% V бомба в отдельный заголовок
% V апробация

\abstract % Структурный элемент: РЕФЕРАТ
%количество источников

\keywords{ЧИСЛЕННЫЕ МЕТОДЫ, МЕТОД РУНГЕ-КУТТУ, ЖЁСТКИЕ СИСТЕМЫ, СДУ, ОДУ, ХИМИЧЕСКАЯ КИНЕТИКА}

Объектом разработки является программа, позволяющая решать системы дифференциальных уравнений.

Цель работы ~--- разработка и отладка программы, выбор оптимальных методов решения обыкновенных дифференциальных уравнений.

В процессе работы были использованы явные и неявные методы Рунге-Кутты, метод QR и LU разложения матрицы, методы итераций, Зейделя и
Ньютона для решения систем нелинейных алгебраических уравнений.

В результате работы была создана программа, позволяющая моделировать уравнения химической кинетики и решать СДУ при помощи большой
коллекции методов.

Обыкновенные дифференциальные уравнения  и системы дифференциальных уравнений  широко используются для математического
моделирования процессов и явлений в различных областях науки и техники. Переходные процессы в радиотехнике, динамика биологических
популяций, модели экономического развития, движение космических объектов и так далее исследуются с помощью ОДУ и СДУ.

Данное ПО можно использовать для исследований химической кинетики. Помимо этого её можно использовать в учебных целях для решения
простых ОДУ или систем, а также в инженерном моделировании процессов газовой динамики.

Преимуществами моей работы по сравнению с аналогами являются большое число явных, неявных или вложенных методов на выбор, возможность
добавления своего метода
решения, быстрота вычисления результата, возможность построения правых частей ОДУ при помощи разработанного интерфейса, вывод
результатов расчётов в текстовом и графическом виде.

В дальнейшем программу можно улучшить путём разработки модулей для работы с динамикой биологических популяций, моделями экономического
развития, движением космических объектов и так далее.