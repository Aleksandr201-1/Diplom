\introduction % Структурный элемент: ВВЕДЕНИЕ

Для решения систем уравнений, описывающих химические процессы, нужно использовать методы, которые дали бы высокую точность при низких
затратах по времени, потому что ЭВМ должна работать в режиме реального времени. В данной работе будет рассмотрено семейство методов
Рунге-Кутты.

В это семейство входит огромное число методов, как явных, так и неявных. Преимущество явных методов заключается в производительности,
так как им не нужно решать на каждом шаге системы алгебраических уравнений. Отсюда следует, что использование явных методов даёт
больший выигрыш по времени, чем использование более производительного оборудования и распараллеливания. Главным преимуществом неявных
методов является наличие устойчивости (А-устойчивости, L-устойчивости и так далее), в результате чего полученное ими решение является
гарантированно устойчивым, в отличие от явных. Однако некоторые СДУ, описывающие химические процессы можно решить и явными методами с
требуемой точностью, потому что свойства А и L-устойчивости являются лишь достаточным, но не необходимым условием эффективности
решения, поэтому нельзя использовать только те или иные методы. Выбрать оптимальный метод можно путём целевого тестирования.

\textit{Результат} работы представляет собой программу для работы с СДУ при помощи множества методов семейства Рунге-Кутты. В работе
реализовано 18 явных методов до 6 порядка точности, 9 вложенных, включая схему Дормана-Принса 4-5 порядка, 19 неявных до 6 порядка. Для 
решения СНУ при итерировании в неявных схемах используются схемы первого порядка (простой итерации, Зейделя) и второго порядка
(метод Ньютона), причём для обращения матрицы применялся метод LU-разложения. Для дифференцирования функции при построении
матрицы Якоби для метода Ньютоны использовались формулы с 4 порядком точности. При необходимости можно использовать формулы
с меньшим порядком.

Для того чтобы определить, какие методы можно использовать, был реализован алгоритм вычисления числа жёсткости СДУ,
использующий QR-разложение матрицы системы для поиска всех собственных чисел. Систему можно считать жёсткой, если для
неё коэффициент жёсткости намного больше единицы. Чёткой границы между жёсткой и не жёсткой системой нет, поэтому пользователь
может выбрать это значение сам, например 100, или же воспользоваться классификацией \cite{Article3}, которая была использована в работе.
Тогда для расчёта применялись только неявные схемы Рунге-Кутты.

Для удобного отображения результатов вычислений был создан генератор pdf-отчётов, в которых представлена информация о решаемой
задаче, метод её решения и таблица Бутчера для этого метода, график, отображающий решение численными методами и аналитическое
(при наличии) и время, затраченное на работу программы. Если задача решалась при помощи жёстких схем, то дополнительно выводится
информация о количестве итераций. Так же, при необходимости, может быть построен приближающий полином. Кроме этого, реализована
возможность простого добавления новых методов решения на случай, если пользователю нужно решение каким-либо специфическим методом,
которого нет в программе. При помощи пользовательского интерфейса можно использовать разработанную программу как для решения ОДУ, так
и в целях обучения.

Программа тестировалась как на задачах химической кинетики, так и на модельных уравнениях и дала положительные результаты.