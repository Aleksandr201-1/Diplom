\section{КЛАССИЧЕСКИЕ МЕТОДЫ ДЛЯ РЕШЕНИЯ ОДУ}

\subsection{Постановка задачи}

Программа, над которой ведётся работа, предназначена для решения ДУ или СДУ
с начальными значениями, то есть задачи Коши ~--- классической математической постановки. 
Сама по себе задача достаточно сложная и изучалась уже более ста лет.

Конкретная прикладная задача сводится к решению ДУ произвольного порядка. Общий вид такой задачи представлен
в примере (\ref{eq:Koshi}).

\begin{equation}
    \begin{cases}
        y^{(n)} = f(x, y, y', y'', ..., y^{(n - 1)})\\
        y(x_0) = y_0\\
        y'(x_0) = y_1\\
        y''(x_0) = y_2\\
        ...\\
        y^{(n - 1)}(x_0) = y_{n - 1}
    \end{cases}
    \label{eq:Koshi}
\end{equation}

Данное уравнение произвольного порядка $n$ может быть преобразовано в систему из $n$ ДУ первого порядка путём
замены переменных. Пример (\ref{eq:KoshiSystem}) демонстрирует преобразование задачи Коши второго порядка в систему из 2-х уравнений
первого порядка, путём замены $y'$ на $z$:

\begin{equation}
    \begin{cases}
        z' = f(x, y, y', y'')\\
        y' = z\\
        y(x_0) = y_0\\
        z(x_0) = y_1
    \end{cases}
    \label{eq:KoshiSystem}
\end{equation}

Из курса ДУ известно, что задача с начальными условиями при непрерывных правых частях, удовлетворяющих условию
Липшица по всем переменным, имеет единственное решение \cite{journal2, book4}.

Методы решения ДУ можно классифицировать на точные, приближенные и численные. Точные методы, которые изучаются
в курсе ДУ и могут быть применены к очень ограниченному кругу уравнений, позволяют выразить решение
ДУ либо через элементарные функции, либо с помощью квадратур от элементарных функций. К приближенным методам
относятся приемы, в которых решение ДУ получается, как предел некоторой последовательности, элементы которой
построены с помощью элементарных функций. Численные методы представляют собой алгоритмы вычисления приближенных значений искомой
функции в узлах. %ссылку бы сюда

Такие задачи могут отличаться между собой сложностью решения. Так одни задачи можно решать при помощи группы явных методов и получать
достаточно точное решение. Другие же задачи, в которых присутствует резкий скачок градиента функции, решать приходится с использованием
неявных схем, так как у них повышенные требования к устойчивости схем решения \cite{book1}. Такие задачи называются жёсткими.

\subsection{Жёсткость}

Будем считать линейную систему обыкновенных ДУ жёсткой, если для уравнения вида
\begin{equation}
    u' = Au,
    \label{eq:SystemDU}
\end{equation}

где $A$ ~--- постоянная матрица $n \times n$,
выполняются следующие требования:

\begin{enumerate}
    \item все собственные числа $\lambda_i$ матрицы $A$ имеют отрицательную действительную часть, т. е. $Re\lambda_i < 0$, 
    $i = 1, 2, ..., n$;
    \item число $S$, вычисляемое по формуле (\ref{eq:tough}) велико.

        \begin{equation}
            S = \dfrac{\max\limits_{1 \leq k \leq n}|Re\lambda_k|}{\min\limits_{1 \leq k \leq n}|Re\lambda_k|}
            \label{eq:tough}
        \end{equation}

\end{enumerate}

Число $S$ называется жёсткостью задачи \cite{Article4, journal1}. Для жёстких задач это число должно быть намного больше единицы, однако чёткой
границы между жёсткой и нежёсткой задачей нет. Для нежёстких систем значение шага по времени ошраничивается в первую очередь желаемой
точностью решения.
Для жёстких же значение шага ограничивается устойчивостью. Так
модельные уравнения с числом жёсткости более $100$ уже дают небольшие скачки погрешности решения. Рассмотрим таблицу \ref{tab:ToughCoeff}.

\begin{table}  
    \caption{Классификация коэффициентов жёсткости}
    \begin{tabularx}{\textwidth}{|l|l|X|}
    \hline
    Классификация & Число жёсткости & Рекомендуемые методы\\
    \hline
    Умеренно жёсткая & {\small$S = O(10)$} & Явные или вложенные 2-4 порядок\\
    \hline
    Средне жёсткая & {\small$S = O(10^2)$} & Явные или вложенные 4-6 порядок\\
    \hline
    Сильно жёсткая & {\small$O(10^2) \leq S \leq O(10^5)$} & Вложенные 6 порядка, неявные 2-4 порядка\\
    \hline
    Экстремально жёсткая & {\small$O(10^6) \leq S \leq O(10^8)$} & Неявные 4-6 порядка\\
    \hline
    Патологически жёсткая & {\small$S \geq O(10^9)$} & Неявные 6 порядка\\
    \hline
    \end{tabularx}
    \label{tab:ToughCoeff}
\end{table}

Разные источники
предлагают свою классификацию задач в зависимости от числа жёсткости. Так в \cite{Article3} в соответствии с величиной $S$ задачу
можно классифицировать как умеренно жёсткую,
средне жёсткую, сильно жёсткую и так далее, что показано в таблице \ref{tab:ToughCoeff}.

В задачах химической кинетики число жёсткости может быть более $10^6$.

\subsection{Явные схемы для решения ОДУ}

Для решения жёстких и нежёстких задач можно использовать различные семейства методов. В данной работе будет рассмотрено семейство
одношаговых методов Рунге-Кутты \cite{book9}.

В семейство методов Рунге-Кутты входит огромное число схем, как явных, так и неявных. Все эти методы представлены в виде таблиц
Бутчера \cite{book1, cite_1_3}. Общий вид явных схем представлен на рисунке \ref{fig:Runge}.

Явные методы обладают нижней диагональной формой таблицы Бутчера и позволяют решать задачи обычными маршевыми методами. В связи с тем,
что явные методы являются условно устойчивыми, работа с ними сильно зависит от размера шага итегрирования и для достижения заданной
точности требуют достаточно мелкий шаг интегрирования и повторного вычисления для повышения порядка точности процедурой Рунге-Ромберга.

\begin{figure}
    \begin{minipage}[t]{8.5cm}
        {\small
        \begin{equation*}
            \begin{cases}
                y_{k + 1} = y_k + \Delta y_k\\
                \Delta y_k = \sum\limits_{i = 1}^sb_iK_i^k\\
                K_i^k = hf(x_k + c_ih, y_k + h\sum\limits_{j = 1}^{i - 1}a_{ij}K_j^k)
            \end{cases}
            %\label{eq-koshi-system}
        \end{equation*}
        }
    \end{minipage}
    \begin{minipage}[t]{7.5cm}
        \begin{table}    
            %\caption{Таблица Бутчера для метода явного метода Рунге-Кутты 4-го порядка}
            \begin{tabular}{|c|c|c|c|c|c|c|}
            \hline
            $c_1$ & $0$ & $0$ & $0$ & & $0$ & $0$\\
            \hline
            $c_2$ & $a_{21}$ & $0$ & $0$ & & $0$ & $0$\\
            \hline
            $c_3$ & $a_{31}$ & $a_{32}$ & $0$ & & $0$ & $0$\\
            \hline
            & & & & & &\\
            \hline
            $c_s$ & $a_{s1}$ & $a_{s2}$ & $a_{s3}$ & & $a_{ss-1}$ & $0$\\
            \hline
            & \cellcolor{lightgray} $b_1$ & \cellcolor{lightgray} $b_2$ & \cellcolor{lightgray} $b_3$ & \cellcolor{lightgray} & \cellcolor{lightgray} $b_{s-1}$ &  \cellcolor{lightgray} $b_s$\\
            \hline
            \end{tabular}
            %\label{tab:RungeKutta4}
        \end{table}
    \end{minipage}
    \caption{Общий вид явных схем}
    \label{fig:Runge}
\end{figure}

Все явные методы являются условно устойчивыми.
В литературе \cite{book1} можно подобрать целую коллекцию одношаговых маршевых многоэтапных методов, которые обеспечивают адекватную точность и
подходят для решения нежёстких ОДУ. Среди этих методов можно выделить классический метод Рунге-Кутты 4-го порядка, схема котрого
представлена на
рисунке \ref{fig:RungeKutta4}.

\begin{figure}
\begin{minipage}[t]{8.5cm}
    {\small
    \begin{equation*}
        \begin{cases}
            y_{k + 1} = y_k + \Delta y_k\\
            \Delta y_k = \frac{1}{6} (K_1^k + 2K_2^k + 2K_3^k + K_4^k)\\
            K_1^k = hf(x_k, y_k)\\
            K_2^k = hf(x_k + \frac{1}{2}h, y_k + \frac{1}{2}K_1^k)\\
            K_3^k = hf(x_k + \frac{1}{2}h, y_k + \frac{1}{2}K_2^k)\\
            K_4^k = hf(x_k + h, y_k + K_3^k)
        \end{cases}
        %\label{eq-koshi-system}
    \end{equation*}
    }
\end{minipage}
\begin{minipage}[t]{7.5cm}
    \begin{table}    
        %\caption{Таблица Бутчера для метода явного метода Рунге-Кутты 4-го порядка}
        \begin{tabular}{|c|c|c|c|c|}
        \hline
        $0$ & $0$ & $0$ & $0$ & $0$\\
        \hline
        $\frac{1}{2}$ & $\frac{1}{2}$ & $0$ & $0$ & $0$\\
        \hline
        $\frac{1}{2}$ & $0$ & $\frac{1}{2}$ & $0$ & $0$\\
        \hline
        $1$ & $0$ & $0$ & $1$ & $0$\\
        \hline
        $0$ & \cellcolor{lightgray} $\frac{1}{6}$ & \cellcolor{lightgray} $\frac{1}{3}$ & \cellcolor{lightgray} $\frac{1}{3}$ & \cellcolor{lightgray} $\frac{1}{6}$\\
        \hline
        \end{tabular}
        %\label{tab:RungeKutta4}
    \end{table}
\end{minipage}
\caption{Схема Рунге-Кутты 4-го порядка}
\label{fig:RungeKutta4}
\end{figure}

Для более жёстких задач метод Рунге-Кутты 4-го порядка может быть недостаточно точным. Иногда приходится сильно уменьшать шаг
интегрирования для того, чтобы решение было устойчивым. Другой путь ~--- использование методов более высоких порядков, например метода
Рунге-Кутты 6-го порядка. Схема данного метода представлена на рисунке \ref{fig:RungeKutta6}.

\begin{figure}
    \begin{minipage}[t]{8.5cm}
        {\small
        \begin{equation*}
            \begin{cases}
                y_{k + 1} = y_k + \Delta y_k\\
                \Delta y_k = \frac{7}{90} (K_1^k + K_6^k) + \frac{16}{45} (K_2^k + K_5^k) -\\
                - \frac{1}{3}K_3^k + \frac{7}{15}K_4^k\\
                K_1^k = hf(x_k, y_k)\\
                K_2^k = hf(x_k + \frac{1}{4}h, y_k + \frac{1}{4}K_1^k)\\
                K_3^k = hf(x_k + \frac{1}{2}h, y_k + \frac{1}{2}K_1^k)\\
                K_4^k = hf(x_k + \frac{1}{2}h, y_k + \frac{1}{7}K_1^k + \frac{2}{7}K_2^k +\\
                + \frac{1}{14}K_3^k)\\
                K_5^k = hf(x_k + \frac{3}{4}h, y_k + \frac{3}{8}K_1^k - \frac{1}{2}K_3^k +\\
                + \frac{7}{8}K_4^k)\\
                K_6^k = hf(x_k + h, y_k - \frac{4}{7}K_1^k + \frac{12}{7}K_2^k -\\
                - \frac{2}{7}K_3^k - K_4^k + \dfrac{8}{7}K_5^k)
            \end{cases}
            %\label{eq-koshi-system}
        \end{equation*}
        }
    \end{minipage}
    \begin{minipage}[t]{7.5cm}
        \begin{table}    
            %\caption{Таблица Бутчера для метода явного метода Рунге-Кутты 6-го порядка}
            \begin{tabular}{|c|c|c|c|c|c|c|}
            \hline
            $0$ & $0$ & $0$ & $0$ & $0$ & $0$ & $0$\\
            \hline
            $\frac{1}{4}$ & $\frac{1}{4}$ & $0$ & $0$ & $0$ & $0$ & $0$\\
            \hline
            $\frac{1}{2}$ & $\frac{1}{2}$ & $0$ & $0$ & $0$ & $0$ & $0$\\
            \hline
            $\frac{1}{2}$ & $\frac{1}{7}$ & $\frac{2}{7}$ & $\frac{1}{14}$ & $0$ & $0$ & $0$\\
            \hline
            $\frac{3}{4}$ & $\frac{3}{8}$ & $0$ & $-\frac{1}{2}$ & $\frac{7}{8}$ & $0$ & $0$\\
            \hline
            $1$ & $-\frac{4}{7}$ & $\frac{12}{7}$ & $-\frac{2}{7}$ & $-1$ & $7$ & $0$\\
            \hline
            $0$ & \cellcolor{lightgray} $\frac{7}{90}$ & \cellcolor{lightgray} $\frac{16}{45}$ & \cellcolor{lightgray} $-\frac{1}{3}$ & \cellcolor{lightgray} $\frac{7}{15}$ & \cellcolor{lightgray} $\frac{16}{45}$ & \cellcolor{lightgray} $\frac{7}{90}$\\
            \hline
            \end{tabular}
            %\label{tab:RungeKutta6}
        \end{table}
    \end{minipage}
    \caption{Схема Рунге-Кутты 6-го порядка}
    \label{fig:RungeKutta6}
\end{figure}

Все явные методы относятся к классу условно устойчивых методов. Точность решения условно устойчивых методов сильно зависит от размера шага,
поэтому у каждого из них есть свой критерий устойчивости.

\subsection{Явные вложенные схемы для решения ОДУ}

В связи с перечисленными недостатками обычных явных схем, есть смысл применить явные вложенные схемы, которые базируются так же на
нижней треугольной матрице Бутчера, обладают маршевым
методом решения и позволяют на базе одних и тех же поправочных коэффициентов моделировать решение с разным порядком точности и тем
самым либо увеличивать шаг интегрирования, либо уменьшать \cite{Article7, cite_1_3}. Общий вид этих схем отличается от явных лишь наличием дополнительной строки
в таблице Бутчера, по которая и позволяет моделировать решение с другим порядком точности для сравнения погрешности. На рисунке
\ref{fig:Falberg} представлен общий вид явных вложенных схем.

\begin{figure}
    \begin{minipage}[t]{8.5cm}
        {\small
        \begin{equation*}
            \begin{cases}
                y_{k + 1} = y_k + \sum\limits_{i = 1}^sb_iK_i^k\\
                y_{k + 1}^* = y_k + \sum\limits_{i = 1}^sb_i^*K_i^k\\
                K_i^k = hf(x_k + c_ih, y_k + h\sum\limits_{j = 1}^{i - 1}a_{ij}K_j^k)
            \end{cases}
            %\label{eq-koshi-system}
        \end{equation*}
        }
    \end{minipage}
    \begin{minipage}[t]{7.5cm}
        \begin{table}    
            %\caption{Таблица Бутчера для метода явного метода Рунге-Кутты 4-го порядка}
            \begin{tabular}{|c|c|c|c|c|c|c|}
            \hline
            $c_1$ & $0$ & $0$ & $0$ & & $0$ & $0$\\
            \hline
            $c_2$ & $a_{21}$ & $0$ & $0$ & & $0$ & $0$\\
            \hline
            $c_3$ & $a_{31}$ & $a_{32}$ & $0$ & & $0$ & $0$\\
            \hline
            & & & & & &\\
            \hline
            $c_s$ & $a_{s1}$ & $a_{s2}$ & $a_{s3}$ & & $a_{ss-1}$ & $0$\\
            \hline
            & \cellcolor{lightgray} $b_1$ & \cellcolor{lightgray} $b_2$ & \cellcolor{lightgray} $b_3$ & \cellcolor{lightgray} & \cellcolor{lightgray} $b_{s-1}$ &  \cellcolor{lightgray} $b_s$\\
            \hline
            & \cellcolor{lightgray} $b_1^*$ & \cellcolor{lightgray} $b_2^*$ & \cellcolor{lightgray} $b_3^*$ & \cellcolor{lightgray} & \cellcolor{lightgray} $b_{s-1}^*$ &  \cellcolor{lightgray} $b_s^*$\\
            \hline
            \end{tabular}
            %\label{tab:RungeKutta4}
        \end{table}
    \end{minipage}
    \caption{Общий вид явных вложенных схем}
    \label{fig:Falberg}
\end{figure}

Порядок точности таких схем записывется в виде $n(m)$, где $n$ ~--- порядок схемы при использовании верхней строки коэффициентов
$b$, $m$ ~--- нижней. Если разница в решениях при использовании этих порядков высока, то шаг следует уменьшить, если же наоборот ~---
разница очень низкая, то шаг можно увеличить для ускорения работы алгоритма.
Пример схемы для метода Фалберга 2(3)-го порядка отображён на рисунке \ref{fig:Falberg2}.

\begin{figure}
    \begin{minipage}[t]{8.5cm}
        {\small
        \begin{equation*}
            \begin{cases}
                y_{k + 1} = y_k + \frac{1}{2}K_1^k + \frac{1}{2}K_2^k\\
                y^{*}_{k + 1} = y_k + \frac{1}{2}K_1^k + \frac{1}{2}K_2^k\\
                K_1^k = hf(x_k, y_k)\\
                K_2^k = hf(x_k + h, y_k + K_1^k)\\
                K_3^k = hf(x_k + \frac{1}{2}h, y_k + \frac{1}{4}K_1^k + \frac{1}{4}K_2^k)\\
                R = |y_{k + 1} - y^{*}_{k + 1}|
            \end{cases}
            %\label{eq-koshi-system}
        \end{equation*}
        }
    \end{minipage}
    \begin{minipage}[t]{7.5cm}
        \begin{table}
            %\caption{Схема метода Фалберга 2(3)-го порядка}
            \begin{tabular}{|c|c|c|c|}
            \hline
            $0$ & $0$ & $0$ & $0$\\
            \hline
            $1$ & $1$ & $0$ & $0$\\
            \hline
            $\frac{1}{2}$ & $\frac{1}{4}$ & $\frac{1}{4}$ & $0$\\
            \hline
            $0$ & \cellcolor{lightgray} $\frac{1}{2}$ & \cellcolor{lightgray} $\frac{1}{2}$ & \cellcolor{lightgray} $0$\\
            \hline
            $0$ & \cellcolor{lightgray} $\frac{1}{6}$ & \cellcolor{lightgray} $\frac{1}{6}$ & \cellcolor{lightgray} $\frac{4}{6}$\\
            \hline
            \end{tabular}
            %\label{tab:Falberg2}
        \end{table}
    \end{minipage}
    \caption{Схема Фалберга 2(3)-го порядка}
    \label{fig:Falberg2}
\end{figure}

Среди вложенных методов выделяется Дорман-Принс 4(5)-го порядка, который является наиболее популярным в этой группе методов \cite{Article2}.
Таблица \ref{tab:DormanPrince45}
для метода Дормана-Принса обладает большим размером и состоит в основном из громоздких дробей. Стоит отметить,
что последний этап этого метода
рассчитывается в той же точке, что и первый этап следующего шага, что позволяет сэкономить немного времи на вычислениях.

\begin{table}    
    \caption{Таблица Бутчера для метода Дормана-Принса 4(5)-го порядка}
    \begin{tabular}{|c|c|c|c|c|c|c|c|}
    \hline
    $0$ & $0$ & $0$ & $0$ & $0$ & $0$ & $0$ & $0$\\
    \hline
    $\frac{1}{5}$ & $\frac{1}{5}$ & $0$ & $0$ & $0$ & $0$ & $0$ & $0$\\
    \hline
    $\frac{3}{10}$ & $\frac{3}{40}$ & $\frac{9}{40}$ & $0$ & $0$ & $0$ & $0$ & $0$\\
    \hline
    $\frac{4}{5}$ & $\frac{44}{45}$ & $-\frac{56}{15}$ & $\frac{32}{9}$ & $0$ & $0$ & $0$ & $0$\\
    \hline
    $\frac{8}{9}$ & $\frac{19372}{6561}$ & $-\frac{25360}{2187}$ & $\frac{64448}{6561}$ & $-\frac{212}{729}$ & $0$ & $0$ & $0$\\
    \hline
    $1$ & $\frac{9017}{3168}$ & $-\frac{355}{33}$ & $\frac{46732}{5247}$ & $\frac{49}{176}$ & $-\frac{5103}{18656}$ & $0$ & $0$\\
    \hline
    $1$ & $\frac{35}{384}$ & $0$ & $\frac{500}{1113}$ & $\frac{125}{192}$ & $-\frac{2187}{6784}$ & $\frac{11}{84}$ & $0$\\
    \hline
    $0$ & \cellcolor{lightgray} $\frac{35}{384}$ & \cellcolor{lightgray} $0$ & \cellcolor{lightgray} $\frac{500}{1113}$ & \cellcolor{lightgray} $\frac{125}{192}$ & \cellcolor{lightgray} $-\frac{2187}{6784}$ & \cellcolor{lightgray} $\frac{11}{84}$ & \cellcolor{lightgray} $0$\\
    \hline
    $0$ & \cellcolor{lightgray} $\frac{5179}{57600}$ & \cellcolor{lightgray} $0$ & \cellcolor{lightgray} $\frac{7571}{16695}$ & \cellcolor{lightgray} $\frac{393}{640}$ & \cellcolor{lightgray} $-\frac{92097}{339200}$ & \cellcolor{lightgray} $\frac{187}{2100}$ & \cellcolor{lightgray} $\frac{1}{40}$\\
    \hline
    \end{tabular}
    \label{tab:DormanPrince45}
\end{table}

Разработано множество вложенных многоэтапных методов, но для решения сложных задач представляют интерес методы, обладающие минимальным
числом этапов и максимальным порядком точности.

\subsection{Неявные схемы для решения ОДУ}

К наиболее сложным методам можно отнести группу неявных схем. Плотно заполненная таблица Бутчера не позволяет использовать маршевые
методы и принуждает решать систему алгебраических уравнений на каждом шаге интегрирования, что вызывет определённые сложности у
разработчиков \cite{book2, book6, cite_1_3, book5}. Общий вид неявных схем продемонстрирован на рисунке \ref{fig:Gauss}.

\begin{figure}
    \begin{minipage}[t]{8.5cm}
        {\small
        \begin{equation*}
            \begin{cases}
                y_{k + 1} = y_k + \sum\limits_{i = 1}^sb_iK_i^k\\
                K_i^k = hf(x_k + c_ih, y_k + h\sum\limits_{j = 1}^{s}a_{ij}K_j^k)
            \end{cases}
            %\label{eq-koshi-system}
        \end{equation*}
        }
    \end{minipage}
    \begin{minipage}[t]{7.5cm}
        \begin{table}    
            %\caption{Таблица Бутчера для метода явного метода Рунге-Кутты 4-го порядка}
            \begin{tabular}{|c|c|c|c|c|c|c|}
            \hline
            $c_1$ & $a_{11}$ & $a_{12}$ & $a_{13}$ & & $a_{1s-1}$ & $a_{1s}$\\
            \hline
            $c_2$ & $a_{21}$ & $a_{22}$ & $a_{23}$ & & $a_{2s-1}$ & $a_{2s}$\\
            \hline
            $c_3$ & $a_{31}$ & $a_{32}$ & $a_{33}$ & & $a_{3s-1}$ & $a_{3s}$\\
            \hline
            & & & & & &\\
            \hline
            $c_s$ & $a_{s1}$ & $a_{s2}$ & $a_{s3}$ & & $a_{ss-1}$ & $a_{ss}$\\
            \hline
            & \cellcolor{lightgray} $b_1$ & \cellcolor{lightgray} $b_2$ & \cellcolor{lightgray} $b_3$ & \cellcolor{lightgray} & \cellcolor{lightgray} $b_{s-1}$ &  \cellcolor{lightgray} $b_s$\\
            \hline
            \end{tabular}
            %\label{tab:RungeKutta4}
        \end{table}
    \end{minipage}
    \caption{Общий вид неявных схем}
    \label{fig:Gauss}
\end{figure}

Особенностью этих методов является то, что в качестве базовых опорных точек являются иррациональные корни многочлена Лежанра.
На рисунке \ref{fig:Gauss4} показана схема неявного метода Гаусса 4-го порядка.

\begin{figure}
    \begin{minipage}[t]{8.5cm}
        {\small
        \begin{equation*}
            \begin{cases}
                y_{k + 1} = y_k + \frac{1}{2}hK_1 + \frac{1}{2}hK_2\\
                K_1 = f(x_k + h(\frac{1}{2} - \frac{\sqrt{3}}{6}), y_k + \frac{1}{4}hK_1 +\\
                + (\frac{1}{4} - \frac{\sqrt{3}}{6})hK_2)\\
                K_2 = f(x_k + h(\frac{1}{2} + \frac{\sqrt{3}}{6}), y_k + (\frac{1}{4} + \frac{\sqrt{3}}{6})hK_1 +\\
                + \frac{1}{4}hK_2)
            \end{cases}
            %\label{eq-koshi-system}
        \end{equation*}
        }
    \end{minipage}
    \begin{minipage}[t]{7.5cm}
        \begin{table}    
            %\caption{Таблица Бутчера для метода неявного метода Гаусса 4-го порядка}
            \begin{tabular}{|c|c|c|}
            \hline
            $\frac{1}{2} - \frac{\sqrt{3}}{6}$ & $\frac{1}{4}$ & $\frac{1}{4} - \frac{\sqrt{3}}{6}$\\
            \hline
            $\frac{1}{2} + \frac{\sqrt{3}}{6}$ & $\frac{1}{4} + \frac{\sqrt{3}}{6}$ & $\frac{1}{4}$\\
            \hline
            $0$ & \cellcolor{lightgray} $\frac{1}{2}$ & \cellcolor{lightgray} $\frac{1}{2}$\\
            \hline
            \end{tabular}
            %\label{tab:Gauss4}
        \end{table}
    \end{minipage}
    \caption{Схема неявного Гаусса 4-го порядка}
    \label{fig:Gauss4}
\end{figure}

По схеме видно, что для нахождения коэффициентов $K_i$ обычные маршевые методы не подходят. Для их получения нужно использовать
итерационные методы решения СНУ, такие как метод простой итерации, метод Зейделя или метод Ньютона. Таблицы Бутчера неявных методов обладают
меньшим размером по сравнению с явными, сохраняя тот же порядок точности. Хорошим примером копактной записи послужит
таблица \ref{tab:Gauss6}, неявного метода Гаусса 6-го порядка.

\begin{table}    
    \caption{Таблица Бутчера для метода неявного метода Гаусса 6-го порядка}
    \begin{tabular}{|c|c|c|c|}
    \hline
    $\frac{1}{2} - \frac{\sqrt{15}}{10}$ & $\frac{5}{36}$ & $\frac{2}{9} - \frac{\sqrt{15}}{15}$ & $\frac{5}{36} - \frac{\sqrt{15}}{30}$\\
    \hline
    $\frac{1}{2}$ & $\frac{5}{36} + \frac{\sqrt{15}}{24}$ & $\frac{2}{9}$ & $\frac{5}{36} - \frac{\sqrt{15}}{24}$\\
    \hline
    $\frac{1}{2} + \frac{\sqrt{15}}{10}$ & $\frac{5}{36} + \frac{\sqrt{15}}{30}$ & $\frac{2}{9} + \frac{\sqrt{15}}{15}$ & $\frac{5}{36}$\\
    \hline
    $0$ & \cellcolor{lightgray} $\frac{5}{18}$ & \cellcolor{lightgray} $\frac{4}{9}$ & \cellcolor{lightgray} $\frac{5}{18}$\\
    \hline
    \end{tabular}
    \label{tab:Gauss6}
\end{table}

Помимо перечисленных групп методов, существуют так же диагональные неявные методы, неявные вложенные, неявные методы без одной строки
или столбца, но все они являются подгруппами неявных методов и используют тот же вид общей схемы. Отдельно можно уделить внимание
диагональным методам или, как их ещё называют, полунеявным.

\subsection{Диагональные схемы для решения ОДУ}

Полунеявные или диагональные схемы входят в группу неявных методо, имеют нижнюю треугольную форму таблицы Бутчера и ненулевыми
элементами диагонали \cite{Article1, Article6, cite_1_3}. Данная особенность позволяет решать системы уравнений для поиска $K_i$ при помощи распараллеливания, что гораздо
быстрее на многоядерных процессорах, а так же с использованием не более одной итерации. К недостаткам таких схем относится плохая
устойчивость по сравнению с обычными неявными.

В качестве примера таких схем можно привести Диагональный метод 4-го порядка, представленный на рисунке \ref{fig:Diagonal4}.

\begin{figure}
    \begin{minipage}[t]{8.5cm}
        {\small
        \begin{equation*}
            \begin{cases}
                y_{k + 1} = y_k - hK_1 + \frac{3}{2}hK_2 - hK_3 + \frac{3}{2}hK_4\\
                K_1 = f(x_k + \frac{1}{2}h, y_k + \frac{1}{2}hK_1)\\
                K_2 = f(x_k + \frac{2}{3}h, y_k + \frac{2}{3}hK_2)\\
                K_3 = f(x_k + \frac{1}{2}h, y_k - \frac{5}{2}hK_1 + \frac{5}{2}hK_2 - \frac{1}{2}hK_3)\\
                K_4 = f(x_k + \frac{1}{3}h, y_k - \frac{5}{3}hK_1 + \frac{4}{3}hK_2 - \frac{2}{3}hK_4)\\
            \end{cases}
            %\label{eq-koshi-system}
        \end{equation*}
        }
    \end{minipage}
    \begin{minipage}[t]{7.5cm}
        \begin{table}    
            %\caption{Таблица Бутчера для метода неявного метода Гаусса 4-го порядка}
            \begin{tabular}{|c|c|c|c|c|}
            \hline
            $\frac{1}{2}$ & $\frac{1}{2}$ & $0$ & $0$ & $0$\\
            \hline
            $\frac{2}{3}$ & $0$ & $\frac{2}{3}$ & $0$ & $0$\\
            \hline
            $\frac{1}{2}$ & $-\frac{5}{2}$ & $\frac{5}{2}$ & $\frac{1}{2}$ & $0$\\
            \hline
            $\frac{1}{3}$ & $-\frac{5}{3}$ & $\frac{4}{3}$ & $0$ & $\frac{2}{3}$\\
            \hline
            $0$ & \cellcolor{lightgray} $-1$ & \cellcolor{lightgray} $\frac{3}{2}$ & \cellcolor{lightgray} $-1$ & \cellcolor{lightgray} $\frac{3}{2}$\\
            \hline
            \end{tabular}
            %\label{tab:Gauss4}
        \end{table}
    \end{minipage}
    \caption{Схема Диагонального метода 4-го порядка}
    \label{fig:Diagonal4}
\end{figure}

Подведём итоги. Преимущество явных методов заключается в производительности, так как им не нужно решать на каждом шаге
системы алгебраических
уравнений. Отсюда следует, что использование явных методов даёт больший выигрыш по времени, чем использование более производительного
оборудования и распараллеливания. Главным же преимуществом неявных методов является наличие устойчивости (А-устойчивости, L-устойчивости и
так далее), в результате чего полученное ими решение является гарантированно устойчивым, в отличие от явных. Для решения жёстких задач
рекомендуется использовать неявные методы как раз потому что они обладают L-устойчивостью.

%\subsection{Устойчивость}

% Поведение численных методов на жестких задачах можно проанализировать, применив эти методы к испытательному уравнению
% $y ' = ky$ при начальном условии $y(0) = 1$. Решение этого уравнения $y(t) = e^{kt}$.
% Если численный метод демонстрирует одинаковое поведение как для фиксированного размера шага, так и для переменного, то метод называется
% A-устойчивым \cite{Wikipedia11}.

% L-устойчивость - это частный случай A-устойчивости, свойства методов Рунге – Кутты для решения обыкновенных дифференциальных уравнений.
% Метод является L-устойчивым, если он является A-стабильным L-устойчивым и имеет более сильное свойство, заключающееся в том,
% что решение приближается к нулю за один шаг, когда размер шага стремится к бесконечности \cite{Wikipedia12}.

Однако некоторые СДУ, описывающие химические процессы можно решить и явными методами с требуемой точностью, потому что свойства А и
L-устойчивости являются
лишь достаточным, но не необходимым условием эффективности решения, поэтому нельзя использовать только те или иные методы. Выбрать
оптимальный метод можно путём целевого тестирования.

\subsection{Вспомогательные методы для работы неявных схем}
%\subsection{Решение систем нелилейных уравнений}

\textit{Решение систем нелилейных уравнений}

Теперь рассмотрим алгоритмы решения СНУ для итерационных процессов в неявных методах. В данной работе
реализовано 3 таких алгоритма:
\begin{itemize}
    \item метод простой итерации,
    \item метод Зейделя,
    \item метод Ньютона.
\end{itemize}

Метод Зейделя и простой итерации являются достаточно быстрыми алгоритмами, так как им не нужно дифференцировать функции или
вычислять Якобиан. Отличие метода Зейделя от метода простой итерации заключается в том, что при вычислении очередного приближения
вектора неизвестных используются уже уточненные значения на этом же шаге итерации. Это обеспечивает более быструю сходимость метода
Зейделя. Общий вид метода простой итерации и Зейделя для неявных методов представлен в формулах (\ref{eq:SI}) и (\ref{eq:Zeidel})
соответственно.

% \begin{figure}
%     \begin{minipage}[t]{8.5cm}
%         {\small
%         \begin{equation*}
%             \begin{cases}
%                 y_{k + 1} = y_k - hK_1 + \frac{3}{2}hK_2 - hK_3 + \frac{3}{2}hK_4\\
%                 K_1 = f(x_k + \frac{1}{2}h, y_k + \frac{1}{2}hK_1)\\
%                 K_2 = f(x_k + \frac{2}{3}h, y_k + \frac{2}{3}hK_2)\\
%                 K_3 = f(x_k + \frac{1}{2}h, y_k - \frac{5}{2}hK_1 + \frac{5}{2}hK_2 - \frac{1}{2}hK_3)\\
%                 K_4 = f(x_k + \frac{1}{3}h, y_k - \frac{5}{3}hK_1 + \frac{4}{3}hK_2 - \frac{2}{3}hK_4)\\
%             \end{cases}
%             %\label{eq-koshi-system}
%         \end{equation*}
%         }
%     \end{minipage}
%     \begin{minipage}[t]{7.5cm}
%         {\small
%         \begin{equation*}
%             \begin{cases}
%                 y_{k + 1} = y_k - hK_1 + \frac{3}{2}hK_2 - hK_3 + \frac{3}{2}hK_4\\
%                 K_1 = f(x_k + \frac{1}{2}h, y_k + \frac{1}{2}hK_1)\\
%                 K_2 = f(x_k + \frac{2}{3}h, y_k + \frac{2}{3}hK_2)\\
%                 K_3 = f(x_k + \frac{1}{2}h, y_k - \frac{5}{2}hK_1 + \frac{5}{2}hK_2 - \frac{1}{2}hK_3)\\
%                 K_4 = f(x_k + \frac{1}{3}h, y_k - \frac{5}{3}hK_1 + \frac{4}{3}hK_2 - \frac{2}{3}hK_4)\\
%             \end{cases}
%             %\label{eq-koshi-system}
%         \end{equation*}
%         }
%     \end{minipage}
%     \caption{Схема Диагонального метода 4-го порядка}
%     \label{fig:Diagonal4}
% \end{figure}

% \begin{figure}
%     \begin{subfigure}[t]{0.45\linewidth}
%         \centering
%         {\small
%             $K_i^{j + 1} = f(x_k + c_ih, y_k + a_{i1}hK_1^{j} + a_{i2}hK_2^{j} + ... a_{ii}hK_i^{j} + ... + a_{is}hK_s^{j})$
%         }
%         \caption{}
%     \end{subfigure}
%     \hfill
%     \begin{subfigure}[t]{0.45\linewidth}
%         \centering
%         {\small
%             $K_i^{j + 1} = f(x_k + c_ih, y_k + a_{i1}hK_1^{j + 1} + a_{i2}hK_2^{j + 1} + ... a_{ii}hK_i^{j} + ... + a_{is}hK_s^{j})$
%         }
%         \caption{}
%     \end{subfigure}
%     \hfill
%     \caption{Схемы методов решения САУ: а) простой итерации; б) Зейделя}
%     \label{fig:bomb}
% \end{figure}

\begin{equation}
    K_i^{j + 1} = f(x_k + c_ih, y_k + a_{i1}hK_1^{j} + a_{i2}hK_2^{j} + ... a_{ii}hK_i^{j} + ... + a_{is}hK_s^{j})
    \label{eq:SI}
\end{equation}

\begin{equation}
    K_i^{j + 1} = f(x_k + c_ih, y_k + a_{i1}hK_1^{j + 1} + a_{i2}hK_2^{j + 1} + ... a_{ii}hK_i^{j} + ... + a_{is}hK_s^{j})
    \label{eq:Zeidel}
\end{equation}

В качестве начальных приближений можно взять значения $K$ с предыдущего шага. Для первого же шага можно посчитать значения $K$ явными
методами.

Более интересным для рассмотрения является метод Ньютона (\ref{eq:Newton}). Медленная скорость работы, обусловленная необходимостью строить матрицу Якоби
на каждом шаге итерирования, позволяет получать решение устойчиво независимо от начального приближения $K$.

\begin{equation}
    \begin{gathered}
        \begin{pmatrix}
            K_1^{j + 1}\\
            K_2^{j + 1}\\
            ...\\
            K_s^{j + 1}
        \end{pmatrix}
        =
        \begin{pmatrix}
            K_1^{j}\\
            K_2^{j}\\
            ...\\
            K_s^{j}
        \end{pmatrix}
        -
        \begin{pmatrix}
            \frac{\partial F_1}{\partial K_1} & \frac{\partial F_1}{\partial K_2} & ... & \frac{\partial F_1}{\partial K_s}\\
            \frac{\partial F_2}{\partial K_1} & \frac{\partial F_2}{\partial K_2} & ... & \frac{\partial F_2}{\partial K_s}\\
            ...\\
            \frac{\partial F_s}{\partial K_1} & \frac{\partial F_s}{\partial K_2} & ... & \frac{\partial F_s}{\partial K_s}\\
        \end{pmatrix}
        \times
        \begin{pmatrix}
            F_1\\
            F_2\\
            ...\\
            F_s
        \end{pmatrix}\\
        F_i = K_i^j - f(x_k + c_ih, y_k + a_{i1}hK_1^{j} + a_{i2}hK_2^{j} + ... a_{ii}hK_i^{j} + ... + a_{is}hK_s^{j})
    \end{gathered}
    \label{eq:Newton}
\end{equation}

Несмотря на лучшую сходимость решения, в большинстве случаев для решения уравнений химической кинетики было достаточно использовать
методы простой итерации и Зейделя, так как они быстрее. В случаях, когда их было недостаточно, применялся метод Ньютона. 
Для построения матрицы Якоби сначала строятся функции $F_i$, после чего используются алгоритмы численного дифференцирования с
различным порядком точности.

%\subsection{Дифференцирование}
\textit{Дифференцирование}

Дифференцирование ~--- операция взятия полной или частной производной функции \cite{book11}. Формулы численного дифференцирования
в основном
используются при нахождении производных от функции вида $y = f(x)$, заданной таблично. Если функция задана не таблично, то
можно просто взять её значения с некоторым, достаточно малым шагом. При решении практических задач, как правило,
используются аппроксимации
первых и вторых производных. В работе было реализовано 4 метода
дифференцирования первого порядка: 2-х точечный, 3-х точечный и два 4-х~точечных.

\begin{equation}
    \begin{gathered}
        y_i' = \frac{y_{i+1} - y_{i-1}}{2h}\\
        y_i' = \frac{-3y_{i} + 4y_{i+1} - y_{i+2}}{2h}\\
        y_i' = \frac{-y_{i+2} + 8y_{i+1} - 8y_{i-1} + y_{i-2}}{12h}\\
        y_i' = \frac{-11y_{i} + 18y_{i+1} - 9y_{i+2} + 2y_{i+3}}{6h}
    \end{gathered}
    \label{eq:Diff}
\end{equation}

Для метода Ньютона использовалась симметричная формула дифференцирования по 4-м точкам.