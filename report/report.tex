\documentclass[a4paper,14pt]{extarticle}
\usepackage[left=1.5cm,right=1.5cm,top=2cm,bottom=2cm]{geometry}
\usepackage[T2A]{fontenc}
\usepackage[utf8x]{inputenc}
\usepackage[russian]{babel}
\usepackage{amsmath}
\usepackage{graphicx}
\usepackage[table]{xcolor}
\usepackage{titlesec}
\usepackage{longtable}
\usepackage{tocloft}
\usepackage{pgfplots}
\pgfplotsset{width=10cm, compat=1.16}

\titleformat{\section}
{\normalfont\bfseries}{}{0pt}{\fontsize{16}{12}\selectfont}
\titlespacing*{\section}{\parindent}{5ex}{1.0ex}
\titleformat{\subsection}
{\normalfont\bfseries}{}{0pt}{\fontsize{14}{12}\selectfont}
\titlespacing*{\subsection}{\parindent}{0.5ex}{1ex}

\setcounter{secnumdepth}{0}
\setlength{\cftbeforesecskip}{1mm}
\setlength{\cftbeforesubsecskip}{1mm}
\renewcommand\cftsecdotsep{\cftdot}
\renewcommand\cftsubsecdotsep{\cftdot}

\title{Отчёт по решению системы ОДУ}
\author{LaTeX-версия}
\date{Mon Mar  6 11:38:49 2023}

\begin{document}

\maketitle

\tableofcontents
\pagebreak

\section{Задача}

$$
\begin{cases}
	y' + 100(y - \sin(x)) = 0\\
	y(0) = 1\\
	x \in [0, 1]
\end{cases}
$$

Порядок задачи: 1

Начальный размер шага: 0.1

\section{Метод решения}

Метод: Gauss\\
Порядок точности: 4\\
Способ: 1

\section{Таблица Бутчера}

\begin{table}[h]
\centering
\begin{tabular}{|c||c|c|}
\hline
0.211325 & 0.25 & -0.0386751\\
\hline
0.788675 & 0.538675 & 0.25\\
\hline
0 & \cellcolor{lightgray} 0.5 & \cellcolor{lightgray} 0.5\\
\hline
\end{tabular}
\end{table}

\section{Жёсткость}

Коэффициент жёсткости задачи: 100\\
Задача жёсткая

\section{Решение задачи}

\begin{longtable}{|m{3cm}|m{3cm}|m{3cm}|}
\hline
\cellcolor{lightgray} X & \cellcolor{lightgray} $Y_{numeric}$ & \cellcolor{lightgray} $Y_{analitic}$\\
\hline
0 & 1 & 1\\
\hline
0.1 & -nan & 0.0899202\\
\hline
0.2 & -nan & 0.18885\\
\hline
0.3 & -nan & 0.285938\\
\hline
0.4 & -nan & 0.38017\\
\hline
0.5 & -nan & 0.470603\\
\hline
0.6 & -nan & 0.556333\\
\hline
0.7 & -nan & 0.636506\\
\hline
0.8 & -nan & 0.710318\\
\hline
0.9 & -nan & 0.777033\\
\hline
1 & -nan & 0.835984\\
\hline
\end{longtable}

Среднее отклонение от аналитического решения: nan

Максимальное отклонение от аналитического решения: 0

\section{Приближающий полином}

Приближающий полином 3й степени: $-nan + nax^1 + nax^2 + nax^3$

\section{График}

\begin{tikzpicture}
\begin{axis}[
	xlabel={$x$},
	ylabel={$y$},
	xmin=0, xmax=1,
	xtick={0,0.2,0.4,0.6,0.8,1},
	legend pos=outer north east,
	ymajorgrids=true,
	grid style=dashed,
]

\addplot[
	color=blue,
	mark=square,
	mark size=0.5pt
]
coordinates {
(0,1)(0.1,-nan)(0.2,-nan)(0.3,-nan)(0.4,-nan)(0.5,-nan)(0.6,-nan)(0.7,-nan)(0.8,-nan)(0.9,-nan)(1,-nan)
};
\addlegendentry{Приближённое решение}

\addplot[
	domain=0:1,
	color=red,
	samples=100
]{(1 + 100 / (1 + 10000)) * exp(0 - 100 * x) - (100 / (1 + 10000)) * (cos(deg(x)) - 100 * sin(deg(x)))};
\addlegendentry{Аналитическое решение}

\end{axis}
\end{tikzpicture}

\end{document}