\documentclass[a4paper,14pt]{extarticle}
\usepackage[left=1.5cm,right=1.5cm,top=2cm,bottom=2cm]{geometry}
\usepackage[T2A]{fontenc}
\usepackage[utf8x]{inputenc}
\usepackage[russian]{babel}
\usepackage{amsmath}
\usepackage{graphicx}
\usepackage[table]{xcolor}
\usepackage{titlesec}
\usepackage{longtable}
\usepackage{tocloft}
\usepackage{pgfplots}
\pgfplotsset{width=10cm, compat=1.16}

\titleformat{\section}
{\normalfont\bfseries}{}{0pt}{\fontsize{16}{12}\selectfont}
\titlespacing*{\section}{\parindent}{5ex}{1.0ex}
\titleformat{\subsection}
{\normalfont\bfseries}{}{0pt}{\fontsize{14}{12}\selectfont}
\titlespacing*{\subsection}{\parindent}{0.5ex}{1ex}

\setcounter{secnumdepth}{0}
\setlength{\cftbeforesecskip}{1mm}
\setlength{\cftbeforesubsecskip}{1mm}
\renewcommand\cftsecdotsep{\cftdot}
\renewcommand\cftsubsecdotsep{\cftdot}

\title{Отчёт по решению системы ОДУ}
\author{LaTeX-версия}
\date{Fri Feb 17 11:33:05 2023}

\begin{document}

\maketitle

\tableofcontents
\pagebreak

\section{Задача}

$$
\begin{cases}
	y' - \sin(x)y + \sin(x)\cos(x) = 0\\
	y(-2) = 0.099\\
	x \in [-2, 3]
\end{cases}
$$

Порядок задачи: 1

Начальный размер шага: 0.2

\section{Метод решения}

Метод: Gauss\\
Порядок точности: 6\\
Способ: 1

\section{Таблица Бутчера}

\begin{table}[h]
\centering
\begin{tabular}{|c||c|c|c|}
\hline
0.112702 & 0.138889 & -0.0359767 & 0.00978944\\
\hline
0.5 & 0.300263 & 0.222222 & -0.0224854\\
\hline
0.887298 & 0.267988 & 0.480421 & 0.138889\\
\hline
0 & \cellcolor{lightgray} 0.277778 & \cellcolor{lightgray} 0.444444 & \cellcolor{lightgray} 0.277778\\
\hline
\end{tabular}
\end{table}

\section{Жёсткость}

Коэффициент жёсткости задачи: 1.2877\\
Задача не жёсткая

\section{Решение задачи}

\begin{longtable}{|m{3cm}|m{3cm}|m{3cm}|}
\hline
\cellcolor{lightgray} X & \cellcolor{lightgray} $Y_{numeric}$ & \cellcolor{lightgray} $Y_{analitic}$\\
\hline
-2 & 0.099 & 0.0999616\\
\hline
-1.8 & 0.0270672 & 0.0278814\\
\hline
-1.6 & -0.000279054 & 0.000430486\\
\hline
-1.4 & 0.0130399 & 0.0136597\\
\hline
-1.2 & 0.0578518 & 0.0583911\\
\hline
-1 & 0.122408 & 0.122874\\
\hline
-0.8 & 0.194526 & 0.19493\\
\hline
-0.6 & 0.26307 & 0.263424\\
\hline
-0.4 & 0.318839 & 0.319157\\
\hline
-0.2 & 0.355054 & 0.355353\\
\hline
-2.77556e-16 & 0.367587 & 0.367879\\
\hline
0.2 & 0.355055 & 0.355353\\
\hline
0.4 & 0.318844 & 0.319157\\
\hline
0.6 & 0.263083 & 0.263424\\
\hline
0.8 & 0.194549 & 0.19493\\
\hline
1 & 0.122432 & 0.122874\\
\hline
1.2 & 0.0578562 & 0.0583911\\
\hline
1.4 & 0.0129905 & 0.0136597\\
\hline
1.6 & -0.000422186 & 0.000430486\\
\hline
1.8 & 0.0267956 & 0.0278814\\
\hline
2 & 0.0986034 & 0.0999616\\
\hline
2.2 & 0.211136 & 0.212785\\
\hline
2.4 & 0.351154 & 0.353086\\
\hline
2.6 & 0.49675 & 0.498931\\
\hline
2.8 & 0.621083 & 0.623455\\
\hline
3 & 0.698737 & 0.701222\\
\hline
\end{longtable}

Среднее отклонение от аналитического решения: 0.000872004

Максимальное отклонение от аналитического решения: 0.00248452

\section{Приближающий полином}

Приближающий полином 3й степени: $0.241352 - 0.11x^1 - 0.05x^2 + 0.04x^3$

\section{График}

\begin{tikzpicture}
\begin{axis}[
	xlabel={$x$},
	ylabel={$y$},
	xmin=-2, xmax=3,
	xtick={-2,-1,0,1,2,3},
	legend pos=outer north east,
	ymajorgrids=true,
	grid style=dashed,
]

\addplot[
	color=blue,
	mark=square,
	mark size=0.5pt
]
coordinates {
(-2,0.099)(-1.8,0.0270672)(-1.6,-0.000279054)(-1.4,0.0130399)(-1.2,0.0578518)(-1,0.122408)(-0.8,0.194526)(-0.6,0.26307)(-0.4,0.318839)(-0.2,0.355054)(-2.77556e-16,0.367587)(0.2,0.355055)(0.4,0.318844)(0.6,0.263083)(0.8,0.194549)(1,0.122432)(1.2,0.0578562)(1.4,0.0129905)(1.6,-0.000422186)(1.8,0.0267956)(2,0.0986034)(2.2,0.211136)(2.4,0.351154)(2.6,0.49675)(2.8,0.621083)(3,0.698737)
};
\addlegendentry{Приближённое решение}

\addplot[
	domain=-2:3,
	color=red,
	samples=100
]{cos(deg(x)) - 1 + exp(0 - cos(deg(x)))};
\addlegendentry{Аналитическое решение}

\end{axis}
\end{tikzpicture}

\end{document}